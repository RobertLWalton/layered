% The Layered Low (L) Programming Language
%
% File:         layered-l-language.tex
% Author:       Bob Walton (walton@acm.org)
% Version:      1a
  
\documentclass[12pt]{article}

\usepackage{makeidx}
\usepackage{pictex}

\makeindex

\setlength{\oddsidemargin}{0in}
\setlength{\evensidemargin}{0in}
\setlength{\textwidth}{6.5in}
\raggedbottom

\setlength{\unitlength}{1in}

\pagestyle{headings}
\setlength{\parindent}{0.0in}
\setlength{\parskip}{1ex}

\setcounter{secnumdepth}{5}
\setcounter{tocdepth}{5}
\newcommand{\subsubsubsection}[1]{\paragraph[#1]{#1.}}
\newcommand{\subsubsubsubsection}[1]{\subparagraph[#1]{#1.}}

% Begin \tableofcontents surgery.

\newcount\AtCatcode
\AtCatcode=\catcode`@
\catcode `@=11	% @ is now a letter

\renewcommand{\contentsname}{}
\renewcommand\l@section{\@dottedtocline{1}{0.1em}{1.4em}}
\renewcommand\l@table{\@dottedtocline{1}{0.1em}{1.4em}}
\renewcommand\tableofcontents{%
    \begin{list}{}%
	     {\setlength{\itemsep}{0in}%
	      \setlength{\topsep}{0in}%
	      \setlength{\parsep}{1ex}%
	      \setlength{\labelwidth}{0in}%
	      \setlength{\baselineskip}{1.5ex}%
	      \setlength{\leftmargin}{1.0in}%
	      \setlength{\rightmargin}{1.0in}}%
    \item\@starttoc{toc}%
    \end{list}}
\renewcommand\listoftables{%
    \begin{list}{}%
	     {\setlength{\itemsep}{0in}%
	      \setlength{\topsep}{0in}%
	      \setlength{\parsep}{1ex}%
	      \setlength{\labelwidth}{0in}%
	      \setlength{\baselineskip}{1.5ex}%
	      \setlength{\leftmargin}{1.0in}%
	      \setlength{\rightmargin}{1.0in}%
	      }%
    \item\@starttoc{lot}%
    \end{list}}

\catcode `@=\AtCatcode	% @ is now restored

% End \tableofcontents surgery.

\newcommand{\CN}[2]%	Change Notice.
    {\hspace*{0in}\marginpar{\sloppy \raggedright \it \footnotesize
     $^{\mbox{#1}}$#2}}
    % Change notice.

\newcommand{\key}[1]{{\bf \em #1}\index{#1}}
\newcommand{\mkey}[2]{{\bf \em #1}\index{#1!#2}}
\newcommand{\skey}[2]{{\bf \em #1#2}\index{#1}}
\newcommand{\smkey}[3]{{\bf \em #1#2}\index{#1!#3}}
\newcommand{\ikey}[2]{{\bf \em #1}\index{#2}}
\newcommand{\ttkey}[1]{{\tt \bf #1}\index{#1@{\tt #1}}}
% < and > do not work for \tt \bf, hence:
\newcommand{\ttnbkey}[1]{{\tt #1}\index{#1@{\tt #1}}}
\newcommand{\ttmkey}[2]{{\tt \bf #1}\index{#1@{\tt #1}!#2}}
\newcommand{\ttmnbkey}[2]{{\tt #1}\index{#1@{\tt #1}!#2}}
\newcommand{\ttfkey}[2]{{\tt \bf #1}\index{#1@{\tt #1}!for #2@for {\tt #2}}}
\newcommand{\ttakey}[2]{{\tt \bf #1}\index{#2@{\tt #1}}}
\newcommand{\ttamkey}[3]{{\tt \bf #1}\index{#2@{\tt #1}!#3}}
\newcommand{\ttdkey}[1]{{\tt \bf .#1}\index{#1@{\tt .#1}}}
\newcommand{\ttdmkey}[2]{{\tt \bf .#1}\index{#1@{\tt .#1}!#2}}
\newcommand{\ttindex}[1]{\index{#1@{\tt #1}}}
\newcommand{\ttmindex}[2]{\index{#1@{\tt #1}!#2}}
\newcommand{\emkey}[1]{{\bf \em #1}\index{#1@{\em #1}}}
\newcommand{\emindex}[1]{\index{#1@{\em #1}}}

\newcommand{\secref}[1]{\ref{#1}$^{p\pageref{#1}}$}
\newcommand{\stepref}[1]{\ref{#1}$^{p\pageref{#1}}$}
\newcommand{\appref}[1]{\ref{#1}$^{p\pageref{#1}}$}
\newcommand{\figref}[1]{\ref{#1}$^{p\pageref{#1}}$}
\newcommand{\pagref}[1]{p\pageref{#1}}
\newcommand{\TBDref}[1]{??}

\newcommand{\EOL}{\penalty \exhyphenpenalty}

\newcount\TildeCatcode
\TildeCatcode=\catcode`\~
\catcode`~=12
\newcommand{\Tilde}{~}
\catcode`~=\TildeCatcode

\newcount\CircumflexCatcode
\CircumflexCatcode=\catcode`\^
\catcode`^=12
\newcommand{\Circumflex}{^}
\catcode`^=\CircumflexCatcode

\newcount\CurlyBraCatcode
\newcount\CurlyKetCatcode
\newcount\SquareBraCatcode
\newcount\SquareKetCatcode
\CurlyBraCatcode=\catcode`{
\CurlyKetCatcode=\catcode`}
\SquareBraCatcode=\catcode`[
\SquareKetCatcode=\catcode`]

\catcode`{=\SquareBraCatcode
\catcode`}=\SquareKetCatcode
\catcode`[=\CurlyBraCatcode
\catcode`]=\CurlyKetCatcode

\newcommand[\CurlyBra][{]
\newcommand[\CurlyKet][}]

\catcode`{=\CurlyBraCatcode
\catcode`}=\CurlyKetCatcode
\catcode`[=\SquareBraCatcode
\catcode`]=\SquareKetCatcode

\newcommand{\ttbrackets}{%
    \renewcommand{\{}{\CurlyBra}%
    \renewcommand{\}}{\CurlyKet}}

\newsavebox{\TILDEBOX}
\begin{lrbox}{\TILDEBOX}
\verb|~|
\end{lrbox}
\newcommand{\TILDE}{\usebox{\TILDEBOX}}

\newsavebox{\BACKSLASHBOX}
\begin{lrbox}{\BACKSLASHBOX}
\verb|\|
\end{lrbox}
\newcommand{\BACKSLASH}{\usebox{\BACKSLASHBOX}}

\newsavebox{\LEFTBRACKETBOX}
\begin{lrbox}{\LEFTBRACKETBOX}
\verb|{|
\end{lrbox}
\newcommand{\LEFTBRACKET}{\usebox{\LEFTBRACKETBOX}}

\newsavebox{\RIGHTBRACKETBOX}
\begin{lrbox}{\RIGHTBRACKETBOX}
\verb|}|
\end{lrbox}
\newcommand{\RIGHTBRACKET}{\usebox{\RIGHTBRACKETBOX}}

\newsavebox{\UNDERLINEBOX}
\begin{lrbox}{\UNDERLINEBOX}
\verb|_|
\end{lrbox}
\newcommand{\UNDERLINE}{\usebox{\UNDERLINEBOX}}

\newsavebox{\CIRCUMFLEXBOX}
\begin{lrbox}{\CIRCUMFLEXBOX}
\verb|^|
\end{lrbox}
\newcommand{\CIRCUMFLEX}{\usebox{\CIRCUMFLEXBOX}}

\newsavebox{\BARBOX}
\begin{lrbox}{\BARBOX}
\verb/|/
\end{lrbox}
\newcommand{\BAR}{\usebox{\BARBOX}}

\newsavebox{\LESSTHANBOX}
\begin{lrbox}{\LESSTHANBOX}
\verb/</
\end{lrbox}
\newcommand{\LESSTHAN}{\usebox{\LESSTHANBOX}}

\newsavebox{\GREATERTHANBOX}
\begin{lrbox}{\GREATERTHANBOX}
\verb/>/
\end{lrbox}
\newcommand{\GREATERTHAN}{\usebox{\GREATERTHANBOX}}

\newlength{\figurewidth}
\setlength{\figurewidth}{\textwidth}
\addtolength{\figurewidth}{-0.40in}

\newsavebox{\figurebox}

\newenvironment{boxedfigure}[1][!btp]%
	{\begin{figure*}[#1]
	 \begin{lrbox}{\figurebox}
	 \begin{minipage}{\figurewidth}

	 \vspace*{1ex}}%
	{
	 \vspace*{1ex}

	 \end{minipage}
	 \end{lrbox}
	 \begin{center}
	 \fbox{\hspace*{0.1in}\usebox{\figurebox}\hspace*{0.1in}}
	 \end{center}
	 \end{figure*}}

\newenvironment{indpar}[1][0.3in]%
	{\begin{list}{}%
		     {\setlength{\itemsep}{0in}%
		      \setlength{\topsep}{0in}%
		      \setlength{\parsep}{1ex}%
		      \setlength{\labelwidth}{#1}%
		      \setlength{\leftmargin}{#1}%
		      \addtolength{\leftmargin}{\labelsep}}%
	 \item}%
	{\end{list}}

\begin{document}
        
\begin{center}

{\Large
The Lower (L) Layered Programming Language \\[0.5ex]
(Draft 1a)}

\medskip

Robert L. Walton\footnote{This document is dedicated to the memory
of Professor Thomas Cheatham of Harvard University.}

November 2, 2012
 
\end{center}

{\small \tableofcontents}

\newpage

\section{Introduction}

This document describes the Lower Layer Programming Language, or
L-Language.  See the Introduction to the Layered
Programming Languages for basic syntax and for an overview of the related
Middle Layer M-Language and Higher Layer H-Language.

Here we will give brief overviews of some of the principle issues
addressed by the L-Language.


\subsection{Code Targeting}

The L-Language is intended to be a target language for compilers of
higher level languages.  As such it is optimized first to be an easy to 
use and adequate target language, and second to be reasonably easy to
compile into efficient assembly language code.

The L-Language is similar to the C programming language
but has additional features intended to give
more control over compilation and more type safety in the use, but
not the declaration, of data.  The L-Language depends upon its powerful
macro facility to allow users to define type safe data declaration
macros.

\subsection{Code Computation}

The L-Language permits code to be computed by executing programs, and then
compiled and run.  The computed code can be an entire function, or can
be embedded within a function.

A main reason for this is to support scientific and large data base
computing in which
code specific to a task must be computed and then optimized.

\subsection{Data Declaration}

The L-Language supports low level declaration of data and fairly high
level type safe optimized usage of declared data.  The L-Language also
supports macros written in the high level H-Language that extends the
L-Language, permitting very
capable macros to be written by users to give users high level
type-safe data declaration facilities tailored to specific kinds of data.

The thesis here is that a high level data declaration language that is
type safe and efficient for all kinds of data is not practical.
So instead tools are provided to create data declaration sublanguages that are
type safe and efficient for more limited kinds of data.

One way of explaining our approach is to say that we are taking the normal
division of programming into two layers, \key{system programming}
and \key{application programming}, and we are adding a third
layer in-between: \key{application system programming}.  This last
kind of programming uses basic and sometimes type unsafe tools to build
advanced and type safe tools for a particular application area, such
scientific programming, business accounting,
programming with particular data structures, and so forth.


\section{Memory}
\label{MEMORY}

We begin with an overview of L-Language memory, and then provide
details in the following subsections.

TBD

\subsection{Numbers}

\ikey{Numbers}{number} are the basic element of L-Language memory.
Numbers are sequences of bits, and each number is one of
four types, unsigned integer, signed integer, floating point number,
and address:

\begin{indpar}
\begin{tabular}{p{1in}@{~~~~~~}p{2in}@{~~~~~~}l}
\bf Name & \bf Kind & \bf Length \\[2ex]
\ttkey{uns8}		& unsigned integer & 8 bits \\
\ttkey{uns16}		& unsigned integer & 16 bits \\
\ttkey{uns32}		& unsigned integer & 32 bits \\
\ttkey{uns64}		& unsigned integer & 64 bits \\
\ttkey{uns128}		& unsigned integer & 128 bits
\end{tabular}

\begin{tabular}{p{1in}@{~~~~~~}p{2in}@{~~~~~~}l}
\ttkey{int8}		& signed integer & 8 bits \\
\ttkey{int16}		& signed integer & 16 bits \\
\ttkey{int32}		& signed integer & 32 bits \\
\ttkey{int64}		& signed integer & 64 bits \\
\ttkey{int128}		& signed integer & 128 bits
\end{tabular}

\begin{tabular}{p{1in}@{~~~~~~}p{2in}@{~~~~~~}l}
\ttkey{float16}		& IEEE floating point number & 16 bits \\
\ttkey{float32}		& IEEE floating point number & 32 bits \\
\ttkey{float64}		& IEEE floating point number & 64 bits \\
\ttkey{float128}	& IEEE floating point number & 128 bits
\end{tabular}

\begin{tabular}{p{1in}@{~~~~~~}p{2in}@{~~~~~~}l}
\ttkey{adr}		& address & 32 bits or 64 bits (see text) \\
\ttkey{unsadr}		& unsigned integer & size of address \\
\ttkey{intadr}		& signed integer & size of address \\
\end{tabular}
\end{indpar}

The \mkey{length}{of number}
of a number is the number of its bits.  Numbers can have
different lengths: for example, unsigned integers can have
lengths of 8, 16, 32, 64, or 128 bits.

Numbers are stored in random access memory (RAM).

An \key{unsigned integer} of length $L$ is a binary integer with
$L$ binary digits (\skey{bit}s) and range from $0$ to $2^L-1$.

A \key{signed integer} of length $L$ is a two's complement integer
of length $L$ and range from $-2^{L-1}$ to $+2^{L-1}-1$.
This represents the integer $I$ in the given range
by the unsigned $L$-bit integer equal to $I~\mbox{modulo}~2^L$.

A \key{floating point number} of length $L$ is a floating point number
represented according to the IEEE 754 standard.  The sizes of exponents
and mantissas for various floating point number sizes is as follows:

\begin{center}
\begin{tabular}{l@{~~~~~~}l@{~~~~~~}l@{~~~~~~}r@{~~~~~~}r}
                   &              &              &             & \bf Maximum \\
\bf Floating Point & \bf Exponent & \bf Mantissa & \bf Decimal & \bf Decimal\\
\bf Number Size    & \bf Size     & \bf Size     & \bf Digits  & \bf Exponent
\\[2ex]
16 bits & 5 bits & 10 bits & 3.31 & 4.51 \\
32 bits & 8 bits & 23 bits & 7.22 & 38.23 \\
64 bits & 11 bits & 52 bits & 15.95 & 307.95 \\
128 bits & 15 bits & 112 bits & 34.02 & 4931.77 \\
\end{tabular}
\end{center}

An \key{address} (one kind of `\key{pointer}') holds a RAM byte address.
A address is a 32-bit or 64-bit unsigned integer whose size
is determined by the target machine.  Some of high order bits may be
required to be all 0's or all 1's, depending upon
the target machine.  The \verb|unsadr| and \verb|intptr|
unsigned and signed integer types of the same size
as an address are provided for storing indices and offsets.

A reasonable assumption for 64-bit addresses is that only the low
order 48-bits of the address are actually used.  This assumption can
be used to put other information in the high order 16 bits
of a 64-bit number containing an address.  For example, an address
can be embedded in a 64-bit floating point NaN.  The L-Language does
\underline{not} depend upon this assumption, but does provide a builtin function
that takes as input a 64-bit integer and two small integers, $L$ and $S$,
and returns a 64-bit address containing
the byte address equal to the low order $L$ bits of the input integer
left shifted by $S$.  The output may have undefined high order bits if
the hardware ignores them when using the output to address memory.
For example, if the hardware ignores the high order 20 bits, and uses only the
low order 44 bits,
this function would just copy its input 64-bit integer to its output
if $L\geq 44$ and $S=0$.\footnote{The I86 64-bit architecture
uses only the low order 48 bits of an address, but \underline{requires}
the high order 17 bits to all be the same, either all 1's or all 0's.
However, as it is unlikely that there will ever be an allocated memory region
that includes address 0 in its interior, it makes no significant difference
whether we consider addresses to be unsigned or signed.}

Variables can be declared to be of non-address numeric type
by declarations of the form
\begin{center}
{\em type-name} {\em variable-name}
\end{center}
For example,
\begin{indpar}\begin{verbatim}
int32 i
uns128 u
float64 f
\end{verbatim}\end{indpar}

Variables can be declared to be of address type by declarations of the form
\begin{center}
{\tt adr} {\em variable-name} {\tt ->} {\em variable-declaration}
\end{center}
For example,
\begin{indpar}\begin{verbatim}
adr ip1 -> int32
adr ip2 -> int32 i2
adr ipp3 -> adr ip3 -> int32 i3
\end{verbatim}\end{indpar}
declare variables {\tt ip1} and {\tt ip2} that store
the address of an {\tt int32} variable, and a variable
{\tt ipp3} that stores the address of a variable that stores
the address of an {\tt int32} variable.
The variable pointed at by an address can be given a name;
e.g., {\tt i2} is the variable pointed at by the address stored
in {\tt ip2}.
The unary `{\tt *}' operator is used to reference the variable
pointed at by an address, so
here
`\verb|* ip1|',
`\verb|* ip2|',
`\verb|i2|',
`\verb|** ipp3|', `\verb|* ip3|',
and `\verb|i3|' reference
an {\tt int32} integer variable, while
`\verb|ip1|', `\verb|ip2|'
`\verb|* ipp3|', and `\verb|ip3|',
reference an {\tt adr} variable that stores the address of
an {\tt int32} variable.

Some of the names in address type variable declarations may be omitted,
as in
\begin{indpar}\begin{verbatim}
adr ip -> int32
adr    -> int32 i
\end{verbatim}\end{indpar}

In both these cases the value stored in the variable is the address
of a 32-bit integer.  However, {\tt ip} names the address and
{\tt i} names the integer.  The unary `{\tt \&}' operator is the inverse
of `{\tt *}', so `{\tt * ip}' can be used to name the
integer, and `{\tt \& i}' can be used to name the address.
Either of these names can be used to assign values, as in
\begin{indpar}\begin{verbatim}
* ip = 5
& i = ip
\end{verbatim}\end{indpar}

Variable declarations within a code block
(\secref{CODE-AND-FRAME-BLOCKS})
are actually short forms for the declaration
of constant stack addresses.  For example,
within a function code block
the variable declarations
\begin{indpar}\begin{verbatim}
int32 i
stack adr ip -> int32
\end{verbatim}\end{indpar}
are equivalent to
\begin{indpar}
{\tt constant stack adr =} {\em some-stack-address} {\tt -> int32 i} \\
{\tt constant stack adr =} {\em some-stack-address}
                           {\tt -> stack adr ip -> int32}
\end{indpar}

so that `{\tt ip = \& i}' is a legal assignment.  The compiler arranges
for appropriate stack addresses to be computed.  `{\tt \& i}' and
`{\tt \& ip}' are both `{\tt constant}' and `{\tt stack}',
meaning respectively that they cannot be changed and that
they have a lifetime guaranteed to last only as long as function
code block is executing.

Note that if `{\tt stack adr ip}' were changed to
`{\tt adr ip}' the assignment `{\tt ip = \& i}' would no longer be
legal, because the lifetime of `{\tt \& i}' is shorter than the
indefinite lifetime promised by `{\tt adr ip}' unqualified by `{\tt stack}'.

\subsection{Subtypes}

A subtype of a parent type is a type whose values are formatted
and aligned in the same manner as values of the parent type and
whose values can be implicitly converted to or from the parent type by
unchecked code.  Given this, it is possible to define checked conversion
functions.

One use of a subtype is to make a promise that a value
is in a given range.  The following is an example of a subtype
that has the range {\tt 0 .. max index-1}:

\begin{indpar}\begin{verbatim}
define type index int = subtype of int32

external:
    constant int32 max index

inline index int .convert. ( int32 i ) unchecked:
    assert ( 0 <= i && i < max index )
    return i

inline int32 .convert. ( index int i ) unchecked:
    int32 j = i
    fact ( 0 <= j && j < max index )
    return j

float64 x[max index]
float64 getx ( index int i ) = x[i]
\end{verbatim}\end{indpar}

Here `{\tt x[i]}' will expand to code containing an implicit
conversion of {\tt i} from {\tt index int} to {\tt int32}
and also a check that {\tt i} is in the proper range to be
in index of {\tt x}, so the code expansion will effectively contain:
\begin{indpar}\begin{verbatim}
    fact ( 0 <= i && i < max index )
    assert ( 0 <= i && i < max index )
\end{verbatim}\end{indpar}
The `{\tt fact}' statement allows the `{\tt assert}' statement
to be optimized away.

A second use of a subtype is to reserve values of the parent type
for special meaning.  For example, an {\tt adr} address value
in the range from 0 through 4095 might be interpreted as an
integer and not
an address, under the assumption that low addresses are not
used by the memory system.  An example of a subtype implementing
this is:
\begin{indpar}\begin{verbatim}
define type special adr = subtype of adr
    // A special adr is either an adr or an integer
    // in the range [0,4096).

inline bool is int ( special adr a ) unchecked:
    unsadr x = a
    return x < 4096

inline bool is adr ( special adr a ) unchecked:
    unsadr x = a
    return x >= 4096

inline adr .convert. ( special adr a ) unchecked:
    unsadr x = a
    assert ( x >= 4096 )
    return a

inline unsadr .convert. ( special adr a ) unchecked:
    unsadr x = a
    assert ( x < 4096 )
    return x

inline special adr .convert. ( unsadr x ) unchecked:
    assert ( x < 4096 )
    adr a = x
    return a

inline special adr .convert. ( adr a ) unchecked:
    unsadr x = a
    assert ( x >= 4096 )
        // We could omit this assert if we knew that
        // every adr was >= 4096.
    return a

float64 x
stack special adr y = & x -> float64
stack special adr z = 23 -> float64
assert ( ! is int ( y ) )
assert ( is int ( z ) )
float64 x2 = * y   // OK
float64 z2 = * z   // Causes assert violation error!
\end{verbatim}\end{indpar}

\subsection{Random Access Memory (RAM) and Blocks}

\ikey{RAM}{Random Access Memory}\index{random access memory!RAM}
is a set of address/byte pairs.  Each 8-bit byte of RAM has an \key{address}
that is an unsigned integer.  It is possible for two addresses to refer to the
same byte, or for an address to refer to no byte.  An address that
refers to a byte is said to be \key{allocated}, and an address that refers
to no byte is said to be \key{deallocated}.  An allocated
byte that has two (or more)
distinct
addresses is said to be \key{shared}, and a byte with exactly one address is
\key{unshared}.
A deallocated byte is neither shared nor unshared.

A \key{RAM block}\index{block!of RAM} is a sequence of RAM bytes with
consecutively increasing addresses.
The \mkey{origin}{of block} of the block is the first address,
and the \mkey{length}{of block} of the block is the number of bytes.

A RAM block is \mkey{allocated}{block}
if the addresses of all its bytes are allocated,
and is \mkey{deallocated}{block} if the addresses of all of its bytes
are deallocated.
A RAM block is \mkey{unshared}{block} if all its bytes are unshared.
A RAM block is \mkey{shared}{block} if
it is one of a set of several blocks (sequences of consecutive addresses)
such that the $n$'th byte of each block in the set is the same.
Note that a RAM block can be neither allocated nor deallocated,
or neither shared nor unshared; e.g., some block bytes may be allocated
and some deallocated.  Also note that a shared byte must have two
distinct addresses, and this has nothing to do with whether the byte
is in two distinct blocks that overlap.

The address space is divided into \skey{page}s, which are RAM blocks that
have an implementation determined length that is a power of two,
e.g., 4096 byte pages, and an address that is an exact multiple of
this length.  There
are L-Language operations which call the operating system to
map an address page to physical memory, thus allocating a page
of RAM.  There are operations to deallocate a page, and to make
two pages be shared (map to the same bytes of RAM).

Pages are the units of allocation and sharing.  Each page of the
address space is either allocated or deallocated, and each allocated page is
either shared or unshared.  The only way for an arbitrary RAM block to
have allocated bytes is to overlap an allocated page, and similarly
for deallocated bytes, shared bytes, and unshared bytes.

A \key{segment} is a contiguous sequence of pages all of which have
the same allocation and sharing status.

The builtin RAM blocks in L-Language are link blocks,
frame blocks (\pagref{CODE-AND-FRAME-BLOCKS}),
code RAM blocks(\pagref{CODE-AND-FRAME-BLOCKS}), numbers, and segments.
All non-segment
blocks are contained within segments.  User defined (i.e., non-builtin)
blocks can be contained within segments.

Certain blocks are `\skey{independent block}s'.  These have the
property that no two distinct independent blocks can overlap.
Link blocks and frame blocks are independent.
Blocks allocated by user provided heap managment routines are
also typically independent.

User defined block data types may be declared by declarations of the form
\begin{center}
\ttkey{define type} {\em type-name} {\tt =}
     {\em type-flag}\,$^\star$
     {\tt (} {\em length}{\tt ,} {\em alignment}{\tt ,} {\em offset} {\tt )}
\\[2ex]
\begin{tabular}{rrl}
{\em type-flag} & ::= & {\tt floating} $|$ {\tt address } $|$ {\tt extendable}
                        \\
                & $|$ & {\tt locatable variable } $|$ {\tt locatable value} $|$
		        {\tt descriptor}
\\[1ex]
{\em length} & ::= & {\em natural-number} \\
{\em alignment} & ::= & {\em natural-number} \\
{\em offset} & ::= & {\em integer} \\
\end{tabular}
\end{center}

The {\em type-flags}, {\em length}, {\em alignment}, and {\em offset} are just
the information needed to allocate a variable of the type to memory and
control and optimize copying of the values of the type as call arguments
and return values.


The layout of a block of the defined type in memory is:

\begin{center}
\begin{picture}(4.0,1.6)
\put(2.0,1.5){\framebox(0.1,0.1){}}
\put(0.0,1.5){\makebox(1.5,0.1)[r]{origin}}
\put(1.6,1.55){\vector(1,0){0.3}}
\put(2.7,1.55){\vector(-1,0){0.5}}
\put(2.0,1.4){\framebox(0.1,0.1){}}
\put(2.0,1.13){\makebox(0.1,0.3){\vdots}}
\put(2.0,1.0){\framebox(0.1,0.1){}}
\put(2.0,0.9){\framebox(0.1,0.1){}}
\put(0.0,0.9){\makebox(1.5,0.1)[r]{address}}
\put(1.6,0.95){\vector(1,0){0.3}}
\put(2.5,0.95){\vector(-1,0){0.3}}
\put(2.4,1.25){\vector(0,1){0.3}}
\put(2.4,1.25){\vector(0,-1){0.3}}
\put(2.4,1.25){\line(1,0){0.6}}
\put(3.1,1.20){\makebox(0.5,0.1)[l]{\em offset}}
\put(2.0,0.8){\framebox(0.1,0.1){}}
\put(2.0,0.7){\framebox(0.1,0.1){}}

\put(2.0,0.43){\makebox(0.1,0.3){\vdots}}
\put(2.0,0.3){\framebox(0.1,0.1){}}
\put(2.0,0.2){\framebox(0.1,0.1){}}
\put(2.7,0.15){\vector(-1,0){0.5}}
\put(2.6,0.85){\vector(0,1){0.7}}
\put(2.6,0.85){\vector(0,-1){0.7}}
\put(2.6,0.85){\line(1,0){0.4}}
\put(3.1,0.80){\makebox(0.5,0.1)[l]{\em length}}
\end{picture}
\\[1ex]
address modulo {\em alignment} = 0
\end{center}

The \mkey{address}{of block} of the block is offset from the origin
of the block by the \mkey{offset}{of block} of block.  The address must be an
exact multiple of the \mkey{alignment}{of block} of the block.
The length of the block is an unsigned integer.  The alignment is an
unsigned integer equal to a power of two that defaults to {\tt 1}.
The offset is a signed integer that defaults to {\tt 0}.

A block can actually have bytes both before and after those described by
its type.  Thus the type specifies that every allocated block of the
type will contain allocated bytes at the addresses

\begin{center}
\begin{tabular}{l}
address - offset \\
address - offset + 1 \\
address - offset + 2 \\
\ldots\ldots\\
address - offset + length - 1
\end{tabular}
\end{center}

These are the \smkey{required byte}s{of a block} of the block.
In addition the block may contain \smkey{optional byte}s{of a block}
either before or after the required bytes.

Note that if the {\em offset} is negative, the block address does not
address a required byte, and may not address any block byte.  Similarly
if the {\em offset} is equal to or greater than the block {\em length}.

The {\em type-flags} in a type definition modify the way that blocks are
passed between a function caller and the function execution.
The possible flags are:

\begin{indpar}

\hspace*{-1em}\ttkey{floating}~~~~~Specifies
the value is an IEEE floating point number that can be moved
through floating point registers in standard hardware without
harm.  Note that some Not-a-Number (NaN) and denormalized IEEE values may be
modified by this process.

\hspace*{-1em}\ttkey{address}~~~~~Specifies
the value is a standard machine address (such as type {\tt adr})
and may be moved through address registers in standard hardware
without harm.  On some hardware, for example, addresses must be
in a certain range.  For example, on current (2012) I84\_64 hardware
an address is a 64 bit integer the top 17 bits of which must be identical.

\hspace*{-1em}\ttkey{extendable} Specifies
the value may contain optional bytes, as indicated above.
Such values cannot be copied between a caller and a called function
execution, but addresses pointing at them may be copied instead.

\hspace*{-1em}\ttkey{descriptor} Specifies there is only one possible
value of the type, and any variable of that value will be initialized
to that value.  This value is necessarily determined by the type, and
in general describes the type: see \secref{NAMES-TYPES-VARIABLES-FUNCTIONS}.

\hspace*{-1em}\ttkey{locatable variable} Specifies
the value must be stored in the thread stack in a
locatable variable:
see TBD for details.
Such values cannot be moved from their locatable variable,
and so cannot be copied between a caller and a called function
execution, but addresses pointing at them may be copied instead.

\hspace*{-1em}\ttkey{locatable value} Specifies
the value must be stored in the stack in some locatable variable,
though if it so stored, it may also be stored elsewhere:
see TBD for details.
Such values can be copied between a caller and a called function
execution.  When copied from caller to called function, the called
function may assume the value is stored in a locatable variable
within the caller frame.  But when returned from a called function to a caller,
the value must be copied into locatable variable before any interrupt
occurs.

\end{indpar}

Some of the standard default type definitions are
\begin{indpar}\begin{verbatim}
define type int32 = (4,0,0)
define type float64 = floating (8,0,0)
// On some hardware:
    define type adr = address (8,0,0)
    define type unsadr = (8,0,0)
    define type insadr = (8,0,0)
// On other hardware:
    define type adr = address (4,0,0)
    define type unsadr = (4,0,0)
    define type insadr = (4,0,0)
\end{verbatim}\end{indpar}

Blocks contain numbers (including direct addresses) and subblocks.
Access to these is provided by displacement values.  For example,
\begin{indpar}\begin{verbatim}
define type type1 = (48,8)
unchecked:
    constant disp member1 = 0 -> uns16 @ type1
    constant disp member2 = 2 -> uns16 @ type1
    constant disp member3 = 4 -> uns32 @ type1
    constant disp member4 = 8 -> float32 @ type1
    constant disp member5 = 12 -> adr @ type1 -> type1
. . . . .
type1 x
x.member1 = 9
x.member2 = -19
x.member3 = 190
x.member4 = -0.01 * x.member3
x.member5 = & x
\end{verbatim}\end{indpar}

A type of the form
\begin{center}
{\em disp} {\tt ->} {\em type1} {\tt @} {\em type2}
\end{center}
denotes a displacement of a block of type {\em type1}
inside a block of type {\em type2}.  The displacement itself
is a signed integer with the same size and characteristics
as an {\tt intadr} signed integer.

A statement of the form
\begin{center}
{\em disp} {\em member-name} {\tt =} {\em value }
           {\tt ->} {\em type1} {\tt @} {\em type2}
\end{center}
assigns the {\em value} to the {\em member-name} variable
that has the above type.

Above `{\tt unchecked:}' introduces a block of unchecked
code, that is, code in which normal type checking rules
do not apply.
Using {\tt =} to assign a signed integer to a displacement
variable is an unchecked operation that can be used only
in unchecked code.

The type of the expression `{\tt x.member1}' is
\begin{center}
\tt constant stack adr -> uns16 V @ type1.member1
\end{center}
and \underline{not}
\begin{center}
\tt constant stack adr V -> uns16 @ type1.member1
\end{center}
In these types
the {\em variable-name} {\tt V} is used to indicate which
value, the integer or the address, is the value of the expression.
This is the value that will be read if the expression appears to
the right of {\tt =} and written if the expression appears to the
left.  This value is also called the \key{value position} inside
the address type, and we say that {\tt V} specifies the value
position inside the address type.
Any other {\em variable-name} besides {\tt V} could be used
to specify the value position in an address type.

Therefore the statement
\begin{center}
\tt x.member1 = 9
\end{center}
sets an {\tt uns16} integer variable, and
\underline{not} an {\tt adr} constant.

Notice that
the expression `{\tt x.member1}' has type
\begin{center}
\tt constant stack adr -> uns16 V @ type1.member1
\end{center}
and \underline{not}
\begin{center}
\tt constant stack adr -> uns16 V
\end{center}
The difference between the type `{\tt uns16 @ type1.member1}'
and the type `{\tt uns16}' is that the former specifies an
{\tt uns16} value that is inside a {\tt type1} value, and is
in fact at offset {\tt member1} inside that {\tt type1} value.
For targets of {\tt adr} data this extra information is
used by the compiler to determine when variables can be aliased.
For targets of {\tt disp} data, as in
\begin{center}
\tt constant disp member1 = 0 -> uns16 @ type1
\end{center}
this extra information is
used by the compiler to type check expressions such as
`{\tt x.member1}'.

\subsection{Independent Block Types}
\label{INDEPENDENT-BLOCK-TYPES}

Some RAM blocks are `\key{independent}'.  Two different independent blocks
are guarenteed to not overlap each other.

There are three basic kinds of independent blocks: static blocks,
stack blocks, and heap blocks.

\ikey{Static blocks}{static block}
are never deallocated or relocated.  They can be \skey{link block}s that
are allocated when a program is loaded, or they can be
user allocated blocks that are never deallocated.

\ikey{Stack blocks}{stack block}
are allocated within thread stack frames
(see \secref{CODE-AND-FRAME-BLOCKS}).
A stack block is allocated by a code block execution, and is
deallocated when that code block execution terminates.

\ikey{Heap blocks}{heap block}
are user allocated blocks that may be deallocated.
Usually they are garbage collectible.

All the heap blocks that are garbage collectible by the
same garbage collection algorithm are grouped together into
a `\key{heap}'.
For each heap there is a list of
static block locations that might hold pointers into
blocks in the heap.  And for each heap and each thread stack
there is a separate list of stack block locations
that might hold pointers into blocks in the heap.

Independent RAM blocks may be relocated (i.e., moved) in RAM.  This applies
even to link blocks.

\subsection{Code and Frame Blocks}
\label{CODE-AND-FRAME-BLOCKS}

A \key{code block} is a sequence of executable program statements
and code subblocks.  Some code blocks are subblocks nested inside other
code blocks, and some are not.

Code blocks are sequences of lexemes, and are \underline{not} RAM blocks.
Associated with each code block is a \key{code RAM block} which is
the code block translated into machine code.  The code RAM block of
a code subblock need not be part of the code RAM block of the containing
code block, but must be associated with the latter.
Code RAM blocks may be static, stack, or heap blocks.

A function is a code block whose execution is initiated by a
call statement.  The execution of the code block containing
the call statement is interrupted in order to execute the function
code block.  When the call returns, the interrupted execution
resumes.

Similarly the execution of a code subblock interrupts the execution
of its containing code block.

Code block executions are grouped into \skey{thread}s.
At any time, only one code
block may be executing statements within a given thread.

Each thread has a \key{stack}.
When a code block begins executing, a \key{frame}
is allocated to the stack of the thread containing
the code block execution.  This frame is a RAM block that holds
data for the code block execution.  Thus there is a
1-1 correspondence between code block executions and frame blocks.

A thread stack is a sequence of RAM blocks.
A frame has a fixed part which is allocated when its code block
execution begins and which has a size determined at compile time.
A frame may also have extensions which are allocated during its
code block execution.  Deallocation of all parts of the
frame occurs precisely when the frame's code block execution terminates.

Frame blocks do not overlap each other, static blocks, or heap blocks.
Therefore blocks allocated within frame blocks that do not overlap
other blocks within the same frame are independent.
Stack variables are independent blocks allocated to a frame or its extensions.
A variable whose size is determined
at compile time and which is sufficiently small will be allocated
in the fixed part of the frame.  For example, the declaration
\begin{center}
\tt int64 x;
\end{center}
allocates an independent block for {\tt x} in the frame associated
with the execution of a code block containing this declaration, and
the complete type of {\tt x} is
\begin{center}
\tt constant stack adr -> int64 x;
\end{center}
where {\tt \& x} is the address of {\tt x} within the fixed part of
the frame.  However a variable of large size, or a variable of size
not known at compile time, will be allocated to an extension of
the frame.  For example, the declarations
\begin{center}
\tt
\begin{tabular}{l}
define type my type(uns32 M) = (24+8*M,0,8) \\
. . . . . \\
constant uns32 n = ... \\
my type(n) y
\end{tabular}
\end{center}
where {\tt M} is a type function argument and {\tt n} an argument value
causes {\tt y} to be allocated as an independent block in an
extension of the frame, and the complete type of {\tt y} is
\begin{center}
\tt constant stack adr -> constant stack adr -> my type(n) y;
\end{center}
where {\tt \& \& y} is the address of the variable
{\tt \& y} that is allocated within the fixed part of
the frame, and the value of this variable is the address of {\tt y}.

\subsection{Names, Types, Variables, and Functions}
\label{NAMES-TYPES-VARIABLES-FUNCTIONS}

A `\key{name}' is used to name a variable, function call, type,
or qualifier.
All names are sequences of words, numbers, quoted strings,
and bracketted lists of arguments.

In certain contexts several names may appear in a row.
The name of a variable or function call may be immediately preceded by the
name of a type or by qualifier names, and the name of a type may
be immediately preceded by qualifier names.
In these contexts qualifier names and type names are recognized
because they have been declared in advance.

Names may be either `\mkey{complete}{name}' or `\mkey{prototype}{name}'.
In a complete name arguments are given as values.  In a prototype
name, the arguments are represented by argument type/name pairs
(with optional qualifiers).

Names may be printed and matched with each other.  Matching may
induce assignment of argument names in prototypes to values supplied
by complete names.  A prototype name may be matched with a complete
name but not with another prototype name.

For example,

\begin{indpar}
\verb|sin ( float64 x )|
matched with
\verb|sin ( 3.141529 )| \\
assigns
\verb|x = 3.141529|
\\[2ex]
if \verb|x| is of type \verb|int32| then \\
\verb|max ( type T, T v, T w )|
matched to
\verb|max ( x , 2000 )| \\
assigns 
\verb|T = int32|, \verb|v = x|,
and \verb|w = 2000|. 

\end{indpar}

Matching a prototype name to a complete name
is complex and allows omitted arguments, most
particularly type arguments such as {\tt T} in the example,
to be deduced via match generated assignments.
Implicit conversions may also be allowed, such as the conversion of
{\tt 2000} to type {\tt T} in the example.
On the other hand, matching two complete names merely involves checking for
equality of names after fully evaluating arguments.

Names have the syntax:

\hspace*{0.2in}
\begin{tabular}{rcl}
{\em name}
& ::= & {\em named-argument-list} {\em named-argument-list}$^\star$ \\
& $|$ & {\em named-argument-list}$^\star$ {\em simple-name} \\
{\em simple-name}
& ::= & {\em word} \{ {\em word} $|$ {\em number} \}$^\star$ \\
& $|$ & {\em quoted-string} \\
{\em named-argument-list}
& ::= & {\em simple-name} {\em argument-list}
                          {\em argument-list}$^\star$ \\
{\em argument-list}
& ::= & {\tt (} {\em argument} \{ {\tt ,} {\em argument} \}$^\star$ {\tt )} \\
& $|$ & {\tt [} {\em argument} \{ {\tt ,} {\em argument} \}$^\star$ {\tt ]} \\
& $|$ & ditto but with brackets other than {\tt (~)} or {\tt [~]} \\
&     & and/or separators other than {\tt ,} \\
{\em argument}
& ::= & {\em expression} $|$ {\em variable-declaration} \\
\end{tabular}

Types are declared by \skey{type declaration}s which have the
syntax:

\hspace*{0.2in}
\begin{tabular}{rcl}
{\em type-declaration}
& ::= & {\tt define type} {\em qualifier}$^\star$ {\em type-name} \\
& $|$ & {\tt define type} {\em qualifier}$^\star$ {\em type-name}
        {\tt =} {\em type-flag}$^\star$ {\em type-spec} \\
{\em type-name}
& ::= & {\em name } \\
{\em type-spec}
& ::= & {\em block-type-spec} $|$ {\em subtype-type-spec} \\
{\em block-type-spec}
& ::= & {\tt (} {\em length}{\tt ,}
	        {\em alignment}{\tt ,} {\em offset} {\tt )} \\
& $|$ & {\tt (} {\em length}{\tt ,} {\em alignment} {\tt )} \\
& $|$ & {\tt (} {\em length} {\tt )} \\
{\em subtype-type-spec}
& ::= & {\tt subtype of} {\em type-name} \\
{\em type-flag}
& ::= & {\tt floating} $|$ {\tt address} $|$ {\tt extendable} \\
& $|$ & {\tt locatable variable} $|$ {\tt locatable value} $|$
	{\tt descriptor}
\end{tabular}

Variables are declared by \skey{variable declaration}s which
have the syntax:

\hspace*{0.2in}
\begin{tabular}{rcl}
{\em variable-declaration}
& ::= & {\em simple-variable-declaration} \label{SIMPLE-VARIABLE-DECLARATION} \\
&     & ~~~~~~~~ \{ {\tt ->} {\em simple-variable-declaration} \}$^\star$ \\
{\em simple-variable-declaration}
& ::= & {\em qualifier}$^\star$ {\em type-name} {\em container}$^\star$ \\
& $|$ & {\em qualifier}$^\star$ {\em type-name} {\em container}$^\star$
        {\em variable-name} \\
& $|$ & {\em qualifier}$^\star$ {\em type-name} {\em container}$^\star$ \\
&     & ~~~~~~~~ {\em variable-name} {\tt =} {\em initializer} \\
& $|$ & {\em qualifier}$^\star$ {\em type-name} {\em container}$^\star$ \\
&     & ~~~~~~~~ {\em variable-name} {\tt ==} {\em equivalent} \\
{\em variable-name}
& ::= & {\em name} \\
{\em container}
& ::= & {\tt @} {\em qualifier}$^\star$ {\em type-name}
\end{tabular}

Functions are declared by \skey{function declaration}s which
have the syntax:

\hspace*{0.2in}
\begin{tabular}{rcl}
{\em function-declaration}
& ::= & {\em function-output-spec} {\em qualifier}$^\star${\em function-name} \\
{\em function-output-spec}
& ::= & {\em qualifier}$^\star${\em type-name} \\
& $|$ & {\em argument-list} \\
{\em function-name}
& ::= & {\em name} \\
\end{tabular}

Qualifier declarations are described later (\secref{QUALIFIERS}).
\ikey{Qualifiers}{qualifier}
have the syntax:

\hspace*{0.2in}
\begin{tabular}{rcl}
{\em qualifier-name}
& ::= & {\em name} \\
\end{tabular}

\subsection{Descriptors}
\label{DESCRIPTORS}

A \key{descriptor} is a run-time value that gives information
about a type.  A descriptor has a type that has the
`\ttkey{descriptor}' {\em type-flag}, and every
such type has the special property that there is only one
value possible of that type\,%
\footnote{This is a central idea in the Haskell programming language}.

The `\ttmkey{basic}{descriptor}'
descriptor includes the length, alignment, offset, and
flags of a type.  An example use is
\begin{indpar}\begin{verbatim}
basic(int32) b;
unsadr s    = b.length       // Length of int32 (4)
unsadr a    = b.alignment    // Alignment of int32 (4)
intadr o    = b.offset       // Offset of int32 (0)
type flags f = b.flags       // Flags of int32 (none)
\end{verbatim}\end{indpar}

If $T$ is a type, then {\tt basic(} $T$ {\tt )} is a descriptor type that has
a unique value computable by the compiler and loader, and this
value has members that give the basic parameters of the type $T$,
namely $T$'s length, alignment, offset, and flags.

It is possible to create new descriptors by defining a descriptor
type and providing for it a constructor that can be run to compute
the one and only value of that type.  This constructor can only take
omitted arguments, which are generally provided by the type itself.
If the descriptor type is defined as a subtype, a conversion function
also needs to be supplied.
For example:
\begin{indpar}\begin{verbatim}
define type constant length ( type T ) = descriptor subtype of unsadr
( constant length(T) s) .construct. ( type T ) unchecked:
    basic(T) b
    s = b.length
( unsadr r ) .convert. ( type T, length(T) s ) unchecked:
    r = s
\end{verbatim}\end{indpar}

Values of descriptor types must be declared to be `{\tt constant}'
(see \pagref{CONSTANT}).  Constructors of such values may not
be invoked with actual arguments; that is, all their arguments must
be omitted (and infered from the type of the output).

\subsection{Traceable Types}
\label{TRACEABLE-TYPES}

The traceable type flags specify that values of a given type are locatable if
they are stored in `{\tt static}' or `{\tt stack}' locations,
that is, in locations
with an address of `{\tt adr}' type that has either
the `{\tt static}' or `{\tt stack}' qualifier, so the location
is either static or in a code block frame.

\begin{indpar}

\hspace*{-1em}\ttkey{traceable location}~~~~~A stack or static
location of a `{\tt traceable location}' type
can be found by appropriate trace operations.

Note that traceable location types cannot be used for arguments
or return values of functions.  However, addresses of locations
of these types can be used.

Traceable location types
are typically used for open files, exception catch data, etc.

\hspace*{-1em}\ttkey{traceable value}~~~~~If a value of a
`{\tt traceable value}' type is stored in a stack or static
location, then at least one such location containing the value
can be found by appropriate trace operations.
If one tracable location contains the value, the value may be
stored in other untraceable stack or static locations.

Traceable value types can be used for arguments and return
values of functions.  When used as an argument, traceable values are stored
in a traceable location by the caller before the call is made.
When used as a return
value, traceable return values which are to be stored in a stack or static
location are stored in a caller traceable location as part of the
call return operation.

Traceable value types are typically used for data of interest to a garbage
collector.

\end{indpar}

Note that the traceable type flags are not allowed for types
of location that are not stack or static.

\subsection{Pointer Types}
\label{POINTER-TYPES}

The standard pointer type is {\tt adr}, which denotes an address.
But any type can be used as a pointer type in variable declarations.

The general syntax of declarations in involving pointer types is

\hspace*{0.2in}
\begin{tabular}{rcl}
{\em pointer-declaration}
& ::= & {\em pointer-variable-declaration} \\
&     & ~~~~~ {\tt ->} {\em target-variable-declaration} \\
{\em pointer-variable-declaration}
& ::= & {\em simple-variable-declaration} \\
{\em target-variable-declaration}
& ::= & {\em pointer-variable-declaration} \\
& $|$ & {\em simple-variable-declaration} \\
{\em simple-variable-declaration}
& ::= & see \pagref{SIMPLE-VARIABLE-DECLARATION} \\
\end{tabular}

A {\em pointer-declaration} can be used specify a \key{pointer type}.
When this is done,
a single arbitrary variable name is placed within the declaration
to indicate the `\key{value position}' of the type.  Thus in
the pointer types
\begin{indpar}\begin{verbatim}
adr V -> int32
adr -> int32 W
\end{verbatim}\end{indpar}
the first is the type of an address pointer variable pointing at an
integer target variable, and the second is the type of an integer
target variable pointed at by an address variable.
The variable names choosen, in this case {\tt V} and {\tt W},
have no significance; any variable names could be substituted here.


The operators `{\tt *}' and `{\tt \&}' are automatically defined
for pointer types and variables of these types.   These operators merely
move the value position in the types.  When applied to variables, these
operators permit reference to pointer and target variables
not explicitly named.  Thus
\begin{center}
\begin{tabular}{lllll}
\tt *  & converts type & {\em type1} {\tt V ->} {\em type2}
       & to            & {\em type1} {\tt ->} {\em type2} {\tt V} \\
\tt \& & converts type & {\em type1} {\tt ->} {\em type2} {\tt V}
       & to            & {\em type1} {\tt V ->} {\em type2} \\
\end{tabular}
\end{center}
and in
\begin{indpar}\begin{verbatim}
int32 z                // Implicitly: constant stack adr -> int32 z
stack adr y -> int32   // Implicitly: constant stack adr ->
                       //                stack adr y -> int32
y = & z                // & z is address of z
int32 w = * y          // * y is target of y
\end{verbatim}\end{indpar}

Whether something is a legitimate pointer type depends upon whether
it may be used to read or write values.  Reading is done by the
implicitly invoked `{\tt .get.}' function, and writing by the
implicitly invoked `{\tt .set.}' function.

For example:

\begin{indpar}\begin{verbatim}
int8 x -> float64 y
float64 z = y     // Translates to .set.(&z,.get.(.get.(&x)));
                  // or to simplify z = .get.(x) or z = .get.(&y);
                  // .get.(x) is probably undefined as x is int8.
y = z             // Translates to .set.(.get(&x),.get.(&z));
                  // or to simplify .set.(x,z);
                  // .set.(x,...) is probably undefined.
\end{verbatim}\end{indpar}

If {\tt x} has a type of the form `$T1${\tt ->}$T2$ {\tt x}',
then `{\tt x = ...}' translates to `{\tt .set.(\&x,...)}' and
`{\tt ... = ... x ...}' translates to `{\tt ... = ... .get.(\&x) ...}',
but we simplify by writing the untranslated versions in these cases.

The {\tt .get.} and {\tt .set.} functions are likely to be
undefined for {\tt int8} pointers.  But in special cases
it may be convenient to define them.  For example, if
{\tt int8} values are used as file descriptors to reference
open files, one might have:

\begin{indpar}\begin{verbatim}
define type open file = ...
open file .get. ( int8 V -> open file )

( int8 V -> open file ) open ( ... )

int8 x -> open file
x = open ( ... )
open file z = * x  // Translates to z = .get.(x)
\end{verbatim}\end{indpar}

Although this is technically correct, it does not make much sense
for an `{\tt open file}' to be passed directly as a value.  Rather
its address should be passed.  So if {\tt int8} values are to be
used as file descriptors, they should be convertable to {\tt adr}
values, as in the following improved example:

\begin{indpar}\begin{verbatim}
define type open file = ...
( adr V -> open file ) .convert. ( int8 V -> open file )
    // Declares conversion function which is not defined here.
unchecked:
    constant disp state = 0 -> uns16 @ open file
    . . . . .

int8 V -> open file open ( ... )
    // Declares function that opens a file.

int8 x -> open file
x = open ( ... )
uns16 y = x.state    // Translates to
                     // y = .get.((.convert.(x)).state))
\end{verbatim}\end{indpar}

Here the implicit~{\tt .convert.}~function is invoked automatically because
$X${\tt .state} is only defined if $X$ has type `{\tt adr -> open file}'.

We end this section with a more refined example.
Suppose `{\tt ptr}' is an indirect address, that is,
the address of the
real address of the variable pointed at.  Further suppose
the real address has the user defined `{\tt relocatable}'
lifetime qualifier, as described in
Section~\secref{LIFETIME-QUALIFIERS}, and the
location where the real address is stored has the `{\tt static}'
lifetime qualifier.
This could all be implemented by the code:

\begin{indpar}[0.2in]\begin{verbatim}
define type ptr = subtype of adr

inline relocatable adr .convert. ( ptr p ) unchecked:
    adr p2 = p -> relocatable adr; return * p2
inline ptr .convert. ( static adr p -> relocatable adr ) unchecked:
    return p

static:
    relocatable adr stub -> uns32
    // The full type of stub is:
    //     constant static adr -> relocatable adr stub -> uns32

ptr p = & stub -> uns32  // Translates to p = .convert.(&stub)
* p = 5                  // Translates to .set(.convert.(p),5)
uns32 y = * p            // Translates to y = .get.(.convert.(p))
ptr q = & stub -> uns64  // Error! No implicit conversion available.
\end{verbatim}\end{indpar}

A pointer value of type `$T1${\tt ->}$T2$' can be converted to type
`$T3${\tt ->}$T4$' by a `{\tt .convert.}' implicit conversion function
that converts any value of type $T1$ to a value of type $T3$, but
only if $T2$ is identical to $T4$.  Thus the assignment of {\tt p}
above is legal and the assignment of {\tt q} is not.


\subsection{Displacements}
\label{DISPLACMENTS}

Displacement types are just like pointer types in so far as declaration
and assignment is concerned, but differ in not having
{\tt .get} and {\tt .set} functions, but instead having
the {\tt "."} operator that combines an address and a displacement.

Builtin {\tt "."} operators combine {\tt disp} and {\tt adr} values.
Specifically, given values of types
\begin{indpar}
{\tt adr A ->} $T1$ \\
{\tt disp D ->} $T2$ {\tt @} $T1$
\end{indpar}
the expression {\tt A.D} adds the value of {\tt D} to the value of
{\tt A} to produce a value of type
\begin{indpar}
{\tt adr AD ->} $T2$ {\tt @} $T1$
\end{indpar}


\subsection{Tuples and Parameter Memory}
\label{TUPLES}

A \key{tuple} is a sequence of zero or more values, where the values
need not all have the same type.
Tuples are surrounded by brackets and their values are separated by
separators.  The kind of bracket and separator determine the
`\mkey{type}{of a tuple}' of the tuple.  For example, there can be
`{\tt (,)}'-tuples that are values bracketed with parentheses `{\tt (~)}'
and separated by commas `{\tt ,}', and there can be
`{\tt [;]}'-tuples that are values bracketed with square brackets `{\tt [~]}'
and separated by semi-colons `{\tt ;}'.

A tuple is \underline{not}
a first class datum; it cannot be stored in most kinds of memory.
Tuples are temporary data used to pass values to functions and
return values from functions.  A call to a function typically involves
two tuples, one holding the
\smkey{argument}s{of function} of the function
and another holding the \smkey{return value}s{of function} of the function.
However calls may involve many tuples or no tuples.

It is also possible for one code block to transfer to another.
This is like a call in which the first code block execution
terminates just as the second code block execution begins.
A tuple can be transferred along with control.

Nested tuples are \key{flattened} when they have the same
tuple type, that is, the same brackets and separators.  For example, 
`{\tt (1, 2, (3, 4), 5)}' is equivalent to `{\tt (1, 2, 3, 4, 5)}'.
As another example, if the function call `{\tt function1()}' returns
the tuple `{\tt (1, 2, 3)}', then the expression `{\tt (function1())}'
is equivalent to `{\tt ((1, 2, 3))}' which is equivalent to
`{\tt (1, 2, 3)}'.  Note that a singleton tuple has no separator,
but for purposes of tuple type matching `no separator' matches
every specific separator.

Tuples are stored in \key{parameter memory}, which is a kind of
memory used to pass function arguments and results.  It is implementation
dependent whether parts of parameter memory are actually RAM memory;
they may be registers, or some other kind of memory.
Different tuple values are independent blocks; they do not overlap
other tuple values or blocks not stored in parameter memory.

\subsection{Clusters}
\label{CLUSTERS}

A \key{cluster} is a group of related variables.  One variable of
the cluster is the \mkey{base variable}{of cluster} of the cluster, and the
other variables have names that are derived from the
name of the base variable using the syntax:


\begin{indpar}
{\em variable-name} ::=
    {\em base-name} {\em member-selector}$^\star$ \\[1ex]
{\em member-selector} \begin{tabular}[t]{@{}rl}
    ::= & {\tt .}{\em member-name} \\
    $|$ & {\tt [}{\em member-index-list}{\tt ]}
    \end{tabular} \\[1ex]
{\em member-index-list} \begin{tabular}[t]{@{}rl}
    ::= & {\em member-index} \\
    $|$ & {\em member-index} {\tt ,} {\em member-index-list}
    \end{tabular} \\[1ex]
{\em member-index} ::= {\em integer-constant-expression}
\end{indpar}

Thus a cluster
is like a structure, but it is a set of variables and not a
piece of memory.

More specifically, the members of the cluster can be named by
adding either a member name preceded by `{\tt .}' or a
`{\tt []}' bracketed list of
integer constant subscripts to
either the base name of the cluster or to another member name
of the cluster.  Two variable names with different base names
belong to different clusters.  Base names must refer to independent
blocks of memory (\secref{INDEPENDENT-BLOCK-TYPES}),
and are different if and only if the memory
blocks they refer are not at the same memory location
(and therefore being independent do not overlap).

Clusters can be passed to functions and returned from functions.
For example:
\begin{indpar}\begin{verbatim}
define type pointer pair = subtype of void
    // if pointer pair pp then
    //   pp.begin points at the first element
    //   pp.end points just AFTER the last element

// Declaration of out of line function to allocate
// a vector of n T's.
//
( pointer pair pp -> T,
  unchecked adr pp.begin -> T,
  unchecked adr pp.end -> T )
    cluster allocate ( type T, basic(T) descriptor, uns32 n )

// Prefix operator to dereference a pointer pair.
//
( adr -> T V ) inline "*"
        ( type T,
          pointer pair pp -> T,
          unchecked adr pp.begin -> T,
          unchecked adr pp.end   -> T )
    unchecked:

    if ( pp.begin < pp.end ) return * pp.begin
    else fatal error
        ( "Deferencing empty pointer pair." )

// Prefix operator to increment the begin pointer of a pointer
// pair.
//
( pointer pair pp2 == pp -> T ) inline "++"
        ( type T,
          basic(T) descriptor,
          pointer pair pp -> T,
          adr -> unchecked adr pp.begin -> T )
    unchecked:

    unsadr c = pp.begin;
    pp.begin = c + descriptor.length;

//
// Example usage:
//
void my function ( void ):

    // Vector of 2 int32's is allocated and the elements
    // are set equal to 100 and 101.
    //
    pointer pair pp = allocate ( 2 ) -> int32
    * pp = 100
    * ++ pp = 101
    . . . .

\end{verbatim}\end{indpar}

Several heretofore unmentioned language features appear in
this code.

TBD

The {\tt ==} operator in an argument or return value
specification specifies that its two sides must be identical.
One use is in `{\tt b == pp.begin}' which states that {\tt b}
and {\tt pp.begin} are identical.  When used as a return value,
as with the {\tt allocate} function,
the value {\tt b} is returned, and the compiler then creates
the variable {\tt pp.begin} and stores the value in it.
Here {\tt pp.begin} is not otherwise specified.
When used in an argument, as with the {\tt "*"} function,
the argument is {\tt b } is implicit and its value is set to
{\tt pp.begin}.

In a different use, `{\tt pp2 == pp}' used as a return value
specifies that {\tt pp2} is identical to the input argument {\tt pp}.
Here it is {\tt pp} that is known and {\tt pp2} that must be
discovered.  In addition,
because these are cluster base names known to the caller
of the function, this means that the caller also identifies
the members of the clusters involved.

If arbitrary code could create clusters, type safety could be
violated.  Therefore clusters can be created only by functions with
the `\ttkey{cluster}' qualifier, which create cluster members after
the fashion of the `{\tt allocate}' function in the example.

The above example could use a different implementation of {\tt "++"} as
follows:
\begin{indpar}\begin{verbatim}
// Unchecked prefix operator to increment an adr -> T value.
//
( adr -> adr a2 -> T ) inline unchecked "++"
        ( type T, basic(T) descriptor,
          caller adr -> adr a -> T )
    unchecked:

    unsadr b = a;
    a = b + descriptor.length;
    & a2 = & a

// Prefix operator to increment the begin pointer of a pointer
// pair.
//
( pointer pair pp2 == pp -> T ) inline "++"
        ( type T,
          pointer pair pp -> T,
          adr -> unchecked adr b == pp.begin -> T )
    unchecked:

    ++ b
\end{verbatim}\end{indpar}

Here an {\tt unchecked} {\tt "++"} prefix operator is first implemented
after the manner of the C programming language for {\tt adr~->~T} values,
and then this is used to implement the {\tt "++"} operator for pointer pairs.

\subsection{Arrays}
\label{ARRAYS}

An array is a cluster that can be used to access many variables whose
addresses are computable by adding an integer to a pointer.
The integer to be added is computed from other integers, usually
by a linear function, though quadratic functions can also be used.

In theory arrays do not have to be built into the language, but they
are because they are so widely used, and because loop code optimization
is done specially for arrays, making it necessary to standardize
the array interface.

They pointer type used as by an array is arbitrary, so long as it
has an unchecked {\tt "+"} operator that adds an integer to a pointer.

The base of an array cluster is a pointer at the element type.
If the base cannot be used directly as a pointer,
it should have an `{\tt unchecked}' access qualifier.

The members of an array cluster are:

\begin{indpar}

\begin{list}{}{}

\item[\ttdkey{dimensions}] Type: any unsigned integer type.
The number of dimensions (subscripts) of the array.

\item[\ttdkey{step}{$[i]$}]
Type: any integer type.
The multiplier
of the $i+1$'st subscript of the array for the purpose of determining
the address offset of the element in the array.
An integer (may be negative).
The unit of these multipliers is the size of the array element.

\item[\ttdkey{lower\_bound}{$[i]$}]
Type: any integer type.
The lower bound (smallest legal value)
of the $i+1$'st
subscript of the array.
An integer (may be negative).

\item[\ttdkey{upper\_bound}{$[i]$}]
Type: any integer type.
The upper bound (largest legal value)
of the $i+1$'st subscript of the array.
An integer (may be negative).

\end{list}

\end{indpar}

A number of operations that access array elements are normally defined.
The following is an example:
\begin{indpar}\begin{verbatim}
// Define an array as a subtype of adr.
//
define type array = subtype of adr

// Return an element of an array.
//
( adr -> T V ) "" ( type T, array a -> T, int32 i0, int32 i1,
                    a.dimension == 2,
                    int32 s0 == a.step[0],
                    int32 s1 == a.step[1],
                    int32 lb0 == a.lower_bound[0],
                    int32 lb1 == a.lower_bound[1],
                    int32 ub0 == a.upper_bound[0],
                    int32 ub1 == a.upper_bound[1] )
    unchecked:

    assert ( lb0 <= i0 && i0 <= ub0 )
    assert ( lb1 <= i1 && i1 <= ub1 )
    adr base = a -> T
    return base + s0 * i0 + s1 * i1
\end{verbatim}\end{indpar}

TBD

\begin{center}
{\em type-name} {\em variable-name}
	{\tt [} {\em dimension-spec}
	       \{ {\tt ,} {\em dimension-spec} \}$^\star$ {\tt ]}
\end{center}
where
\begin{indpar}
{\em dimension-spec} ::= {\em bound-spec}~~{\em step-spec-option} \\[1ex]
{\em bound-spec} ::= {\em length}
    $|$ {\em lower-bound}~~{\tt ..}~~{\em upper-bound} \\[1ex]
{\em step-spec} ::= {\tt ++} {\em step}
\end{indpar}
For example,
\begin{indpar}\begin{verbatim}
int32 iarray[5]
int32 adr piarray[1 .. 5] -> int32
adr -> int32 irarray[10,20]
adr -> int32 tirarray[10++1,20++10]
\end{verbatim}\end{indpar}

Here
\begin{itemize}
\item {\tt iarray[i]} is an {\tt int32} variable for {\tt i} =
{\tt 0}, {\tt 1}, {\tt 2}, {\tt 3}, and {\tt 4}.  The {\tt length}, which is
{\tt 5}, specifies the number of elements in the array's one dimension,
and array indexing begins at {\tt 0}.

\item {\tt piarray[i]} is a `direct address of {\tt int32}'
variable for {\tt i} =
{\tt 1}, {\tt 2}, {\tt 3}, {\tt 4}, and {\tt 5}.  The {\em lower-bound}
and {\em upper-bound} of the dimension index, {\tt 1} and {\tt 5} respectively,
are given.

\item {\tt \& irarray[i,j]}
is a `direct address of {\tt int32} variable', and
{\tt irarray[i,j]} is the {\tt int32} location so addressed,
for {\tt i} = {\tt 0}, {\tt 1}, \ldots, {\tt 9} and
for {\tt j} = {\tt 0}, {\tt 1}, \ldots, {\tt 19}, where the two
dimension {\em lengths} are given, respectively {\tt 10} and {\tt 20}.
The elements of the array are stored in {\tt 10*20} consecutive
memory locations that in memory order
have indices {\tt (0,0), (0,1), \ldots, (0,19),
(1,0), (1,1), \ldots, (9,19)}, with the last index {\tt j} varying
most rapidly.  This means that the implied step size of the first
dimension is {\tt 20} and of the second dimension is {\tt 1}.

\item {\tt tirarray[i,j]} is just like {\tt irarray[i,j]} but the
elements are stored in memory in transposed order:
{\tt (0,0), (1,0), \ldots, (9,0),
(0,1), (1,1), \ldots, (9,19)}, with the first index {\tt i} varying
most rapidly.  The step size of the first dimension is
given as {\tt 1} and of the second is given as {\tt 10}.

\end{itemize}

An array is a cluster such that

\begin{center}
\begin{tabular}[t]{lrl}
\multicolumn{3}{l}{
address of $array${\tt [}$i_0${\tt ,}$i_1${\tt ,}\ldots{\tt ,}$i_{n-1}${\tt ]}
= } \\\hspace*{2em}
&   & $array${\tt .base} \\
& + & $array${\tt .step[0]} * $i_0$ * {\em element-size}  \\
& + & $array${\tt .step[1]} * $i_1$ * {\em element-size}  \\
& + & \ldots\ldots \\
& + & $array${\tt .step[}$n-1${\tt ]} * $i_{n-1}$ * {\em element-size} \\
\end{tabular}

\end{center}

where

\begin{center}
\begin{tabular}{l}
$array${\tt .lower\_bound[0]} $\leq$ $i_0$ $\leq$ $array${\tt .upper\_bound[0]}
\\
$array${\tt .lower\_bound[1]} $\leq$ $i_1$ $\leq$ $array${\tt .upper\_bound[1]}
\\
\ldots\ldots
\\
$array${\tt .lower\_bound[}$n-1${\tt ]}
    $\leq$ $i_{n-1}$ $\leq$ $array${\tt .upper\_bound[}$n-1${\tt ]}
\end{tabular}

\end{center}

The members of an array cluster are called the
\smkey{parmeter}s{of array} of the array.
The parameters of an array $A$ whose element type is $T$ and whose
inheritable qualifiers are $Q$ are

\begin{indpar}

\begin{list}{}{}

\item[\ttdkey{first}] Type: `$Q$ $T$ {\tt *\&}'.
The first element of the array,
that is, the lowest addressed array element that can be accessed
using subscripts within the range indicated by the subscript bounds.
`\verb|& |$A$\verb|.first|' is {\tt NULL} if there are no
accessible elements (i.e., if some
upper bound is not at least as great as the corresponding lower bound).

\item[\ttdkey{length}] Type: `{\tt unsadr}'.
The total number of elements in the array.
The last used element of the array is
\begin{center}
{\tt (} $Q$ $T$ {\tt *\&)}
    $A${\tt .first @ (} $A${\tt .length - 1 ) * }{\em element-size}
\end{center}

This is an unchecked expression that could be written to access the
last array element.  The {\tt .length} is \verb|0| if there are no array
elements.

\item[\ttdkey{base}] Type:
`{\tt noaccess} $T$ {\tt * (unchecked || readonly) * \&}'.
The address of the {\tt [0,0,}\ldots{\tt ,0]}
element of the array.  This may not actually be inside the array,
as {\tt 0} subscripts may not be within bounds.

It is possible for unchecked code
to store into the {\tt .base} in order to relocate
the array.  When an array is passed to or returned from a function, it is
actually the {\tt .base} that is passed or returned.

\item[\ttdkey{dimensions}] Type: any unsigned integer type.
The number of dimensions (subscripts) of the array.

\item[\ttdkey{step}{$[i]$}]
Type: any integer type.
The multiplier
of the $i+1$'st subscript of the array for the purpose of determining
the address offset of the element in the array.
An integer (may be negative).
The unit of these multipliers is the size of the array element.

\item[\ttdkey{lower\_bound}{$[i]$}]
Type: any integer type.
The lower bound (smallest legal value)
of the $i+1$'st
subscript of the array.
An integer (may be negative).

\item[\ttdkey{upper\_bound}{$[i]$}]
Type: any integer type.
The upper bound (largest legal value)
of the $i+1$'st subscript of the array.
An integer (may be negative).

\end{list}

\end{indpar}

All the parameters except {\tt .first} and {\tt .length}
are given to describe an array.  The {\tt .first} and {\tt .length}
parameters are then computed from the other parameters.\footnote{
Note that these are functions and not C language members, and as
such only {\tt .first} returns an lvalue in the sense of the
C language.}

Given the example from above,
\begin{indpar}\begin{verbatim}
int32 iarray[5]
int32 * piarray[1 .. 5]
int32 * & irarray[10,20]
int32 * & tirarray[10++1,20++10]
\end{verbatim}\end{indpar}
we have
\begin{indpar}\begin{verbatim}
iarray.lower_bound[0] == 0
iarray.upper_bound[0] == 4
iarray.step[0] == 1
iarray.length == 5

piarray.lower_bound[0] == 1
piarray.upper_bound[0] == 5
piarray.step[0] == 1
piarray.length == 5

irarray.lower_bound[0] == 0
irarray.upper_bound[0] == 9
irarray.step[0] == 20
irarray.lower_bound[1] == 0
irarray.upper_bound[1] == 19
irarray.step[1] == 1
irarray.length == 200

tirarray.lower_bound[0] == 0
tirarray.upper_bound[0] == 9
tirarray.step[0] == 1
tirarray.lower_bound[1] == 0
tirarray.upper_bound[1] == 19
tirarray.step[1] == 10
tirarray.length == 200

\end{verbatim}\end{indpar}

When an array is allocated to the stack or to static memory,
the length of the array is just enough to include all the elements of the
array.  There can be unused elements.  Thus given
\begin{indpar}\begin{verbatim}
int32 x[1 .. 4 ++ 3]
\end{verbatim}\end{indpar}
then in memory the array {\tt x} is laid out as
\begin{center}
{\tt x[1]}, unused, unused,
{\tt x[2]}, unused, unused,
{\tt x[3]}, unused, unused,
{\tt x[4]}
\end{center}
and we have {\tt x.length == 10}, {\tt x.first == x[1]},
{\tt x.base == \&x[1] - 3}.

The parameters of an array can be any expression that evaluates
to an integer at the time the array is allocated.  For example,
\begin{indpar}\begin{verbatim}
uns32 constant number = ...
int64 constant first = ...
uns32 constant step = ...

int32 x[number]
int32 y[first .. first + number - 1 ++ step]
\end{verbatim}\end{indpar}

The variables used to determine the parameters
of an array must be `{\tt constant}' or
`{\tt readonly}' so that checked code cannot modify them.
More on this below.

\subsubsection{Array Function Parameters}
\label{ARRAY-FUNCTION-PARAMETERS}

An array can be passed to a function and returned as the value
of a function.  Some examples are:
\begin{indpar}\begin{verbatim}
void sort ( int64 vector[length], int32 length )
void transpose ( float64 array[lb .. ub ++ step1, lb .. ub ++ step2],
                 int64 lb, int64 ub, int64 step1, int64 step2 )
(float64 array[length,length]) unit ( uns32 length )
(string output[olength], uns32 olength) sort
        ( string input[ilength], uns32 ilength )
\end{verbatim}\end{indpar}

When an array is passed to a function, the parameters needed to
describe the array must be expressions computable from parameters
passed to the function.  When an array is returned by a function,
the parameters needed to describe the array must be expressions
computable from parameters either passed to or returned from the
function.  Note that integer function parameters are `{\tt constant}'
by default.

When a function that takes arrays as input parameters or computes
arrays as output results is called, the array parameters that are
function parameters or results may be implicitly specified.  Examples using
the function declarations above are:
\begin{indpar}\begin{verbatim}
int64 x[1 .. 100]
. . . compute x . . .
sort ( x[] )    // length implicitly passed.
float y[1 .. 5][1 .. 5]
. . . compute y . . .
transpose ( y[] )     // lb, ub, step1, step 2 implicitly passed.
float z[] = unit ( 10 )     // computes z and its parameters.
string s[10]
. . . compute s . . .
string t[] = sort ( s[] )     // ilength and olength implicitly passed.
                              // computes t and its parameters.
\end{verbatim}\end{indpar}

In order to indicate that implicit array parameter passing is
desired, the empty brackets `\ttkey{[]}' must be appended to the
array name.

Checked functions can only allocate arrays to static or stack
memory.  Unchecked functions can be written to allocate arrays
in a heap and return them.

The parameters of an array need not be constants: they can be
variables.  However, the variables must be qualified (see
\secref{QUALIFIERS}) in such a way that checked code cannot
modify them.  Unchecked functions may then be written to change the
array parameter variables and reconfigure the array: e.g., the
array may be expanded.  An example is:

\begin{indpar}\begin{verbatim}
// Implement my array:

define type my array = (32,8)
readonly unsadr *& .n ( my array *& @ a ) unchecked = a@0
readonly unsadr *& .max_n ( my array *& @ a ) unchecked = a@8
readonly unsadr *& .inc_max ( my array *& @ a ) unchecked = a@16
noaccess float64 (readonly *&b) [1 .. a.n] .void ( my array *& @ a )
    unchecked = a@24
(my array *& @ a) .constructor.() unchecked
{
    a.n = 0
    a.max_n = 0
    a.inc_max = 30
    a.void.base = NULL
}
void push ( my array *& a, float64 value ) unchecked
{
    if ( a.n >= a.max_n )
    {
        a.max_n += a.inc_max
        float64 * ap =
            malloc ( sizeof ( float64 ) * ( a.max_n ) )
        if ( a.void != NULL )
            memcpy ( ap, a.void.base,
                     a.n * sizeof ( float64 ) )
        mfree ( a.void.base )
        a.void.base = ap
    }
    ap[a.n] = value
    ++ a.n

}

// Use my array:

my array a
push ( a, 1.0 )
push ( a, 2.0 )
a[1] // Equals 1.0.
a[2] // Equals 2.0.
a[0] // Run time error.
a[3] // Run time error.
\end{verbatim}\end{indpar}

There are a number of language features introduced in the above code.

First, {\tt a@24} is of type `{\tt void *\&}' which is
converted to type `{\tt float64 (*\&) [1 .. a.n]}' by the
`{\tt unchecked}' code of the {\tt .void} function.

Second, an array value is just its {\tt .base}, which is
just a pointer.  Viewed as the
element of an object, {\tt a.void} is just a {\tt .base} pointer,
and just enough
space to store this pointer must be allocated to it.  Here
we treat pointers as being 8 byte long values that are
aligned on 8 byte boundaries, but as 4 byte pointers will
fit into 8 bytes, the code will work even with 4 byte pointers.

Third, unchecked code can store a pointer of type `$T$ {\tt *}'
into $X${\tt .base}.  The line `{\tt a.void.base = ap}' above
does just this.  Unchecked code can also read the base, as in
the line `{\tt memcpy ( ap, a.void.base, n )}', and implicitly
convert the base to a pointer to a `{\tt readwrite}' value.  Checked code
can also read the base, but the value returned will be given
the type `noaccess $T$ {\tt *}' since base pointers may not
point at real memory (e.g., given $X${\tt [1 ..~2]} then
$X${\tt .base} points to an unusable location just before
the first element).

Fourth, `{\tt unchecked}' code can use a pointer as if it were
a 1-dimensional array with step {\tt +1} and no bounds.  This is
done in the statement `{\tt ap[a.n] = value}' above.

Fifth, `{\tt .base}' when applied to an array,
and `{\tt .void}' applied to any expression,
can always be omitted, as long as the
result can be disambiguated.  Thus in the above
`{\tt a.void.base}' could be replaced by either
`{\tt a.void}' or just `{\tt a}'.  In the code at the end
after the functions, `{\tt a.void[...]}' is replaced by `{\tt a[...]}'.

Sixth, `\verb|float64 (*&b) [1 .. a.n]|' differs from
`\verb|float64 *&b [1 .. a.n]|' which is equivalent to
`\verb|float64 *& (b [1 .. a.n])|'.  The first subexpression
means that the value returned is a pointer to the location holding
the base of an array whose elements are {\tt float64} values.
The second subexpression means that the value returned is the base
of an array whose elements are pointers to `{\tt float64}' values
(and {\tt b[i]} refers to the `{\tt float64}' value pointed at by the
array element indexed by {\tt i}).
In the expression `\verb|float64| \verb|(readonly *&b)| \verb|[1 .. a.n]|'
it is the array {\tt .base} that is read-only, not the array elements.

Seventh, the code above contains both an implementation of
{\tt my array} and a use of {\tt my array}.  We remind the reader that
the implementation is intended to be written by macros which are called
by the end user and which ensure that
the unchecked code generated is in fact type-safe, while the code that
uses {\tt my array} is intended to be written by the end user and
should be type-safe in its own right.

\subsubsection{Copying Arrays}
\label{COPYING-ARRAYS}

If you want to copy entire arrays you have to apply the `{\tt *}'
operator to them.  Thus
\begin{indpar}\begin{verbatim}
declare alias qualifier same_data
same_data int32 x[1 .. 10]
same_data int32 y[1 .. 10] = x
// Now x and y are the same piece of memory.  This would not be
// legal without the alias qualifier (described later).
y[1] = ..., etc.
int32 z[1 .. 10]
* z = * y
// Now the elements of x (and y) have been copied to
// the elements of z.
\end{verbatim}\end{indpar}

\subsubsection{Subarrays}
\label{SUBARRAYS}

Subarrays can be computed by expressions of the form
\begin{center}
{\em array-expression}
	{\tt [} {\em selector-spec}
	       \{ {\tt ,} {\em selector-spec} \}$^\star$ {\tt ]}
\end{center}
where
\begin{indpar}
{\em selector-spec} ::=
    {\em lower-bound}~~{\tt ..}~~{\em upper-bound}~~{\em step-spec-option}
\end{indpar}

Here the bounds are valid subscripts for the given array, and the
steps are increments in these valid subscripts.

For example,
\begin{indpar}\begin{verbatim}
declare alias qualifier first_half, second_half
first_half second_half int32 x[1 .. 10]
first_half int32 x1[1 .. 5] = x[1 .. 5]
second_half int32 x2[1 .. 5] = x[6 .. 10]

declare alias qualifier odd_half, even_half
odd_half even_half int32 y[1 .. 10]
odd_half int32 y1[1 .. 5] = y[1 .. 9 ++ 2]
even_half int32 y2[1 .. 5] = x[2 .. 10 ++ 2]
\end{verbatim}\end{indpar}

Here {\tt x1} is the first half of {\tt x} and {\tt x2} is the second
half.  Similarly {\tt y1} is all the elements of {\tt y} with
odd subscripts, and {\tt y2} is all the elements with even
subscripts.

The `{\tt =}' operator used with arrays overlays and does not
copy.  It can only be used to initialize subarray names, and
only requires that the arrays involved have equal numbers of
subscripts in each dimension.

In somewhat more generality a subarray of an array may be
given a \key{linear view} defined by any linear map between
arbitrary subscripts and the subscripts of the array.  For
example:
\begin{indpar}\begin{verbatim}
declare alias same_data
same_data int32 x[1 .. 70]
same_data int32 y[1 .. 10, 1 .. 7] = ([i,j] ===> x[(i-1)*7 + j])
same_data int32 z[1 .. 7, 1 .. 10] = ([i,j] ===> y[j,i])
same_data int32 w[1 .. 6] = ([i] ===> y[i,i+1])
\end{verbatim}\end{indpar}

Here {\tt y} is a 2-dimensional array overlaid on the vector {\tt x},
{\tt z} is the transpose of {\tt y}, and {\tt w} is a subdiagonal of
{\tt y}.

\subsubsection{Array Maps}
\label{ARRAY-MAPS}

Special forms of array can be defined by adding parameters to the
array and defining special operators.  For example, a symmetric
array can be defined by:
\begin{indpar}\begin{verbatim}
int32 x[28]
uns32 x.size = 7
int32 *& x[int32 i, int32 j]
{
    assert ( 0 <= i < size && 0 <= j < size )
    return x[i*(i+1)/2 + j]
}
\end{verbatim}\end{indpar}

Here `{\tt .size}' is an extra member of the cluster {\tt x}
and `{\tt x[ ]}' is just a function that surrounds it arguments
with `{\tt [ ]}' instead of `{\tt ( )}'.

\subsection{Qualifiers}
\label{QUALIFIERS}

The L-Language controls the way memory is accessed via \skey{qualifier}s.
A qualifier is either a qualifier name followed by optional
parameters in parentheses, or is a parentheses surrounded
qualifier expression.
Qualifier parameters may be expressions written
in a sublanguage special to the type of qualifier.

Qualifiers may be applied to values or to locations of memory.  Some
types of qualifiers apply to values and some to locations.
For example, `{\tt stack}' is a value qualifier that says that
a value is not valid after the currently executing function returns,
`{\tt constant}' is a location qualifier that says that the
location will never ever be written into,
and `{\tt readlonly}' is a location qualifier that says that the
location cannot be written using the given address of the location.

For example, the
declarations
\begin{indpar}[0.5em]\begin{verbatim}
constant int32 x = 5
stack adr yp = & x -> readonly int32 y
\end{verbatim}\end{indpar}
promise that after initialization no one will ever
write into the location {\tt x} (which is {\tt constant}),
that {\tt yp} will be valid as long as the current function
execution exists (because it holds a {\tt stack} value), and that
{\tt yp} cannot be used to write to {\tt y} (which is {\tt readonly}).

A location qualifier is applied to the type of a value that is the
target of a pointer.  That is, they appear in types of the form
\begin{center}
{\em pointer-type} \verb|->| {\em location-qualifier} {\em value-type}
\end{center}

Location qualifiers actually affect operations on the pointer,
and not on the value pointed at.  Thus both `{\tt constant}' and
`{\tt readonly}' applied to the pointer target type
prevent the builtin {\tt .set.} operator from
being executed on the pointer itself.

Qualifiers may be given in type definitions, and apply to all
values and memory locations of the defined type.
For example, if {\tt Q} is a qualifier,
\begin{indpar}[0.5em]\begin{verbatim}
define type Q type1 = (48,8)
type1 x
adr yp -> type1
\end{verbatim}\end{indpar}
is equivalent to
\begin{indpar}[0.5em]\begin{verbatim}
define type type1 = (48,8)
Q type1 x
adr yp -> Q type1
\end{verbatim}\end{indpar}

Qualifiers may also be applied to code blocks.  Such are called
`\skey{code block qualifier}s'.  A qualifier just before the
{\tt =} or \LEFTBRACKET{} that introduces the function body in a
function definition applies to the code in the
body.
For example, if {\tt Q} is a qualifier,
\begin{indpar}[0.5em]\begin{verbatim}
int32 function1 () Q = ... 
int32 function2 () Q
{
   . . .
}
\end{verbatim}\end{indpar}
apply {\tt Q} to the body of the functions.

A qualifier just before the name of a function in a function definition
or declaration applies to calls to the function, and is said to be
a `\key{function access qualifier}'.  For example,
\begin{indpar}[0.5em]\begin{verbatim}
int32 Q function3 ()
( adr -> int32 V ) Q function4 ()
\end{verbatim}\end{indpar}

Some qualifiers
are both access and body qualifiers, and these
are put in the position of function access qualifier.

If in the above examples we assume that
{\tt Q} is a protection qualifier, then {\tt function3}
and {\tt function4} can only be called by function bodies (or code blocks, see below) that have
the {\tt Q} qualifier, and in particular these functions can be called
by {\tt function1} and {\tt function2}.

A qualifier immediately preceding a \verb|{ }| bracketed code block
inside a function applies to the code block.  For example,
\begin{indpar}[0.5em]\begin{verbatim}
int32 function5 ()
{
   . . .
   Q {
       . . . a code block . . .
   }
}
\end{verbatim}\end{indpar}
applies {\tt Q} to the code block.  If {\tt Q} is an access qualifier,
the code block can call functions with access qualifier {\tt Q}, even
though the body of {\tt function5} outside the code block cannot.

There are several kinds of qualifiers.  We summarize these here, and
give details in following sections.

\ikey{Lifetime qualifiers}{lifetime qualifier}
specify the lifetime of a value.  This in turn controls where
the value may be stored, e.g., a value of limited lifetime may
not be stored in a location whose value is declared to be of
unlimited lifetime.

There are three builtin lifetime qualifiers.
The `{\tt static}' qualifier specifies that the value
has unlimited lifetime.
The `{\tt stack}' qualifier specifies that the value
has a lifetime that lasts at least until the
current function execution terminates.
The `{\tt caller}' qualifier
specifies that the value has a lifetime that begins before the
called function starts to execute and lasts until
sometime after that function returns.

In addition user define `calling lifetime qualifiers' may be defined
that specify that lifetimes will continue until
an out-of-line function having or not having given
access qualifiers (see below) is called.

\ikey{Caching qualifiers}{caching qualifier} specify when memory locations
may be changed, and thence when they may be cached in registers.  The
`{\tt constant}' qualifier specifies that a location will never be changed,
and may be cached indefinitely in registers.
The `{\tt unique}' qualifier specifies that a location has no overlaps
with other locations, and,
in particular, that a location passed from a caller to
a called function does not overlap with any other location so passed
or with any location accessible from other data.
\ikey{Parallel qualifiers}{parallel qualifier} may be declared to
identify memory locations whose register \underline{and hardware}
caches are to be flushed by explicit cache flushing statements in the
L-Language.
These are used for memory locations shared between different hardware
processing units (e.g., between a central processing unit and an
IO device, or between two different central processing units).

\ikey{Access qualifiers}{access qualifier} may be declared
to control which functions can access a memory location or call other
functions.  Access qualifiers may or may not be
\skey{protection qualifier}s.
The builtin non-protection access qualifiers are
`{\tt readwrite}', `{\tt readonly}', `{\tt writeonly}', and `{\tt noaccess}'.
The builtin protection qualifier is `{\tt unchecked}'.
Protection qualifiers may also be defined by the user.

A memory location with protection qualifier $X$ can only be
accessed by code blocks that have qualifier $X$, and
a function with an access protection
qualifier $X$ can only be called by functions
whose bodies have qualifier $X$.  The builtin protection
qualifier `{\tt unchecked}' permits access to various
builtin operations that explicitly or implicitly involve type
conversion.

Memory locations can be assigned qualifier expressions that are logical
combinations of qualifier names and the operators `\verb/|/' and
`\verb/&/'.  For example, a memory location may be assigned the expression
`\verb/(/$X$\verb/|constant)/' where $X$ is a protection qualifier.
A function body
with qualifier $X$ will be able to access such a memory location to both
read and write it, but a function body without qualifier $X$ will treat the
location as having the `\verb|constant|' qualifier, will not be able
to write the location, and will assume that the location will
not be written by others.  In order for this to work properly, the functions
with qualifier $X$ that are permitted to write the location must be called
first, before functions without qualifier $X$ are allowed to access the
location.  Alternatively, `{\tt constant}' can be replaced by `{\tt readonly}',
which forbids writing the location but does \underline{not} promise that
other code will not write the location.

\ikey{Inline qualifiers}{inline qualifier} specify that functions called
from a particular block of code are to be inlined, or that a particular
block of code is to be out-of-line even if code surrounding it is
being inlined.  Inlining is necessary to get full optimization.

Qualifiers may also be defined that represent
\ikey{groups}{group!of qualifiers} of
qualifiers.

Qualifiers may also be \ikey{inherited}{qualifiers}.  For example, if
{\tt G} is a group qualifier,
\begin{indpar}\begin{verbatim}
G? readonly unsadr .n ( adr -> my array G? a ) unchecked = ...
\end{verbatim}\end{indpar}
causes any qualifier in group {\tt G} that is present for the
argument `{\tt a}' to be applied to the function result.
See \secref{QUALIFIER-INHERITANCE}.

\subsubsection{Lifetime Qualifiers}
\label{LIFETIME-QUALIFIERS}

Lifetime qualifiers determine when value lifetimes expire.
For example, the lifetime of the address of a stack variable allocated
in a function frame expires when the function execution terminates.

The `{\tt static}' lifetime qualifier that indicates a value has
an unlimited lifetime is applied by default to numeric values.
But it is possible to apply other lifetime
qualifiers.  For example, if an {\tt int32} value is used
as an open file designator, it may be given a limited lifetime.

A lifetime qualifier {\tt Q1} is said to be `\key{convertible}'
to a lifetime qualifier {\tt Q2} if {\tt Q1} promises a lifetime
at least as long as {\tt Q2} promises.

If {\tt Q} is a lifetime qualifier in the type
\begin{center}
\tt adr xp -> Q mytype x
\end{center}
then {\tt Q} is said to be a qualifier of the location {\tt x},
all values written into that location must have lifetime qualifiers that
are convertable to {\tt Q},
and all values read from that location will have qualifier {\tt Q}.


Three lifetime qualifiers are builtin:  `{\tt static}', `{\tt stack}',
and `{\tt caller}'.
Checked code can only allocate to static memory
or to the stack memory of the currently executing function, and addresses
of variables allocated by checked code are given the `{\tt static}' or
`{\tt stack}' qualifier accordingly.
Addresses of variables that are part of a function argument or return value
are given the `{\tt caller}' lifetime qualifier by default.


\begin{indpar}

\hspace*{-1em}\ttkey{static}~~~~~Specifies
the value lives forever.  Automatically applied to the addresses of
locations allocated at load time, and may be applied to the
addresses of locations allocated at program initialization time
and never deallocated.

The `{\tt static}' lifetime qualifier is convertible to any other
lifetime qualifier.  It is also the default lifetime qualifier
for non-pointer values.

\hspace*{-1em}\ttkey{stack}~~~~~Specifies the value lives as least as
as long as the current function execution.  Equivalent to
{\tt stack(0)}: see the following.

`{\tt static}' values are convertable to `{\tt stack}' values.

\hspace*{-1em}\ttkey{stack(n)}~~~~~
\label{STACK(N)}
Specifies the value lives at least as long as the execution of
the current code block of nested level
{\tt n} within the currently executing function.
Unnested code within the function is considered to be at nested level {\tt 0}.
{\tt stack(n)} is automatically applied
to the addresses of locations allocated to the
current function frame by code in a level {\tt n} block (such locations
are freed when the block of level {\tt n} terminates).

`{\tt stack(m)}' is convertable during the execution of
a block of level {\tt n}$>${\tt m} to `{\tt stack(n)}'.


A `{\tt stack(n)}' qualifier \underline{cannot}
be applied to function argument and return types.

\hspace*{-1em}\ttkey{caller}~~~~~Only
used in a function $F$'s argument or return value types to qualify a values
such as addresses passed to the function.
Specifies the value's lifetime begins sometime before a call to $F$
and continues until sometime after that call returns.

The `{\tt caller}' qualifier is like the {\tt stack} qualifier but
specifies that the value is valid at the moment the function begins
executing and at the moment the caller code returned to by
the function begins executing.

`{\tt static}', `{\tt stack(n)}', and {\tt caller} qualifiers
can be converted to the `{\tt caller}' qualifier when the latter is
part of the type of a value that is being passed as an argument or
a return value to a function being called by the code that does
the conversion.  However, 
`{\tt stack(n)}' and {\tt caller} qualifiers \underline{cannot}
be converted to the `{\tt caller}' qualifier when the latter is
part of the type of a value that is an argument or return
value of the currently executing function.  Note that in this
case a `{\tt caller}' qualifier cannot even be converted to
a `{\tt caller}' qualifier.


For example:
\begin{indpar}[0.5em]\begin{verbatim}
int32 function1 ( caller adr arg1 -> static adr -> int32 )
    // The argument is a pointer to a caller allocated location
    // holding a pointer to a statically allocated int32.
    // function1 can use `*arg1 = value' to return a static
    // adddress to an int32 value, and can also use
    // `**arg1 = value' to return an int32 value.
int32 function2 ( caller adr arg1 -> caller adr -> int32 )
    // The argument is a pointer to caller allocated location
    // holding a pointer to a caller allocated location holding
    // an int32.  function2 can use `**arg1 = value' to return
    // an int32 value, and can use `*arg1 = value' to return
    // a static address, but not a caller address.
int32 function3 ( static adr arg1 -> caller adr -> int32 )
    // Illegal; a static location cannot contain a value whose
    // lifetime may only last until some function returns.
\end{verbatim}\end{indpar}

The `{\tt caller}' lifetime qualifier is the default
lifetime qualifier for any function
argument or return value pointer value that has no
lifetime qualifier.  Thus in the above example the `{\tt caller}'
qualifiers could have been omitted.

The `{\tt caller}' lifetime qualifier may \underline{not} be used
except in function argument and return value declarations.

\end{indpar}

Users may define \skey{calling lifetime qualifier}s by using
a declaration of the form:
\begin{center}
{\tt declare} \begin{tabular}[t]{@{}l@{}}
              {\tt calling lifetime qualifier} {\em qualifier-name} \\
              {\tt with continuer} {\em qualifier-name} \\
	      {\tt with discontinuer} {\em qualifier-name} \\
	      {\tt with default} {\em qualifier-name}
              \end{tabular}
\end{center}
An example is:
\begin{center}
\begin{tabular}{l}
\tt declare calling lifetime qualifier relocatable \\
\tt ~~~~~~~~with continuer nonrelocating \\
\tt ~~~~~~~~with discontinuer relocating \\
\tt ~~~~~~~~with default relocating \\
\end{tabular}
\end{center}

If {\tt Q} is a calling lifetime qualifier, {\tt CQ} is its
continuer, and {\tt DQ} is its discontinuer,
then a value with qualifier {\tt Q} will have
a lifetime that is terminated by a call to an out-of-line function that has
the access qualifier {\tt DQ} or the execution of a code
block that has qualifier {\tt DQ}.

If an out-of-line function has the {\tt CQ} access qualifier,
the lifetime of a value with qualifier {\tt Q}
is not terminated by a call to the function.
Similarly if a code block has the {\tt CQ} qualifier,
the lifetime of the value is not terminated by the execution
of the code block.

The default qualifier is applied to out-of-line functions with no
explicit {\tt CQ} or {\tt DQ} access qualifier.
The {\tt CQ} qualifier is applied by default to both code blocks
and to inline functions with no explicit
{\tt CQ} or {\tt DQ} qualifier.

The {\tt DQ} and {\tt CQ} qualifiers are incompatible with each other.

The discontinuer lifetime qualifier {\tt DQ}
is also automatically declared to be
a protection qualifier(\pagref{PROTECTION-QUALIFIER}) so that
a block of code with continuer qualifier {\tt CQ} cannot call a function
or invoke a block of code with the discontinuer qualifier {\tt DQ}.

Thus in the example, a `{\tt relocatable}' value would have a lifetime until
an out-of-line function with no
`{\tt nonrelocating}' access qualifier is
called, or a code block with the `{\tt relocating}' qualifier is executed,
where this code block could be part of an inline function.

\subsubsection{Caching Qualifiers}
\label{CACHING-QUALIFIERS}

When a memory location is moved into a register, the register becomes
a \key{software cache}\index{cache!software}
of the memory location.  Thus memory locations
are cached in registers under the control of software, and not hardware.
Unfortunately this makes things very difficult for the compiler
(the ultimate fix is a radical change in computer architecture;
see Appendix~\secref{ALIASING-HARDWARE}).

Memory locations may also be placed in hardware caches.  In a hardware
system with memory shared among multiple central processing units,
each central processing unit may have its own hardware cache that is
\underline{not} automatically synchronized with shared memory or
the caches of other central processing units.  Special instructions
are used to synchronize these hardware caches with shared memory.

Caching qualifiers determine when to invalidate and when to write
software and hardware caches.

Two caching qualifiers are builtin: `{\tt constant}' and `{\tt unique}'.
Two kinds are declarable: alias qualifiers and parallel qualifiers.

\begin{indpar}

\hspace*{-1em}\ttkey{constant}\label{CONSTANT}~~~~~The
memory location is constant and will not be modified at all.

The memory location can be cached in registers without
restriction.

Memory locations declared `{\tt constant}' must be written during their
initialization.  This is managed by not declaring a location to be
`{\tt constant}' when it is passed as a result value to the function
that initializes the location.

The following code has some examples:
\begin{indpar}[0.5em]\begin{verbatim}
int32 function1 ( stack adr -> constant int32 arg1 )
int32 function2 ( stack adr -> readonly int32 arg1 )
int32 function3 ( stack adr -> int32 arg1 )
. . . . . . . . . .
constant int32 value1 = 5
   // Permitted:
   //   caller adr -> int32 .assign. ( 5 ) is called.
... my_function ( ... )
{
    value1 = 10 
       // Erroneous:
       //   caller adr -> constant int32 .assign. ( 10 ) is called.
    function1 ( value1 )   // Permitted; arg1 constant.
    function2 ( value1 )   // Permitted; arg1 readonly.
    function3 ( value1 )   // Erroneous; arg1 NOT constant
                           // or readonly.
    int32 value2 = 15
    function1 ( value2 )   // Erroneous.  Function1 might make
                           // a long term cache of value2 which
                           // might then be changed.
    function2 ( value2 )   // Permitted.  Function2 merely
                           // promises not to write value2.
    . . . . . . . . . .
\end{verbatim}\end{indpar}

Note that the `{\tt constant}' qualifier in an argument type is a promise
that \underline{no code} will \underline{ever} modify the location.  By contrast
a `{\tt readonly}' argument qualifier (like `{\tt const}' in C/C++)
is a promise only that the location will not be modified during
the course of the call.  The difference is that a called function
may construct a long-lived cache of a `{\tt constant}' location
but not of a `{\tt readonly}' location.

The `{\tt constant}' qualifier obeys the following special rules:
\begin{enumerate}
\item
`{\tt constant}' is removed from a location passed to a
function as the result location in the initializer statement
of the location.
\item
`{\tt constant}' is the default cache qualifier for locations of
function arguments that are passed by value and not by reference.
\end{enumerate}

The `{\tt constant}' qualifier is automatically applied to temporary
values.  Thus continuing the above example:
\begin{indpar}[0.5em]\begin{verbatim}
    function1 ( value1 + value2 )
        // Permitted; value1 + value2 is a `constant' int32
        // temporary value.
\end{verbatim}\end{indpar}


\hspace*{-1em}\ttkey{unique}~~~~~Function argument and return value
locations can be given the {\tt unique} qualifier if these locations
are being passed to the function from its caller.  These locations
cannot overlap any other location passed to the function or available
to the function via global data.  This means these locations must
either have been allocated by the caller, or passed as `{\tt unique}'
argument or return value locations to the caller from its caller.

A `{\tt unique}' memory location may be passed as an argument to
or return value of a function only if the function argument
or return value location is declared `{\tt unique}' within the
function declaration.  A stack memory location may be so passed
if no address that permits access to the location has been stored
anywhere except in the stack and the location is not passed as
more than one of the function argument or return value locations.

Addresses that permit a `{\tt unique}' memory location to be
accessed may only be stored in stack locations where they
can be tracked by the compiler.

A `{\tt unique}' memory location may be cached in a register.
This cache must be flushed when an out-of-line function is called
only if the address of the location is passed to the out-of-line
function as the address of a `{\tt unique}' function argument or
return value location.

\hspace*{-1em}\ikey{Alias Qualifiers}%
{alias qualifier}~~~~~Alias
qualifiers are declared by the programmer and are used to tag
memory locations that may be aliased with each other.

At compile time the L-Language converts location
addresses into a form that is the sum
of a \key{base address} and an integer \key{offset}.
In addition the location may have associated
alias qualifiers.  The L-Language assumes two locations do not overlap
unless they have a common alias qualifier, or unless the locations have
identical base addresses and their offsets indicate they overlap.

A base address is one of the following:

\begin{enumerate}
\item The address of a memory block allocated by the loader.
\item The address of a memory block allocated to the current function frame.
\item An address passed into the current function as the location
of an argument or return value.
\item An address returned from a function call or builtin operation,
      including addresses read from memory locations.
\end{enumerate}

For example:
\begin{indpar}[0.5em]\begin{verbatim}
void copy1 ( uns8 source[length],
             uns8 destination[length],
             uns32 length )
{
    uns32 i = 0
    loop: {
        if ( i >= length ) break loop
        destination[i] = source[i]
        ++ i
    }
}
declare alias qualifier same_data
void copy2 ( same_data uns8 source[length],
             same_data uns8 destination[length],
             uns32 length )
{
    uns32 i = 0
    loop: {
        if ( i >= length ) break loop
        destination[i] = source[i]
        ++ i
    }
}
uns8 x[16]
x[0] = 0
copy1 ( x[0 .. 14], x[1 .. 15], 15 )   // Erroneous.
copy2 ( x[0 .. 14], x[1 .. 15], 15 )   // Zeros x.
\end{verbatim}\end{indpar}

The call to {\tt copy1} is erroneous because
arguments that are not supposed to be aliased are in fact aliased.

Aliasing is checked as necessary by the compiler and at run time.
The run-time checks can be suppressed by compilation switches.

Simple alias qualifiers are just names.  Complex alias qualifiers
contain a \key{type overlap expression} as follows:

\begin{indpar}
{\em complex-alias-qualifier} ::=
    {\em alias-name} \verb|(| {\em type-overlap-expression} \verb|)| \\[1ex]
{\em type-overlap-expression} ::=
    {\em type-inclusion-sequence}
    \{ \verb/|/ {\em type-inclusion-sequence} \}$^\star$ \\[1ex]
{\em type-inclusion-sequence} ::=
    {\em type-factor}
    \{ \verb/>>/ {\em type-factor} \}$^\star$ \\[1ex]
{\em type-factor} ::= {\em type}
                  $|$ \verb|(| {\em type-overlap-expression} \verb|)|
\end{indpar}

A canonical {\em type-overlap-expression} is one with no parenthesized
{\em type-\EOL overlap-\EOL sub\-ex\-pres\-sions},
and any {\em type-\EOL overlap-\EOL expression} can be made
canonical by distributing \verb/|/ over \verb|>>| as in

\begin{indpar}
\ldots{} \verb|>>| \verb|(| $e_1$ \verb/|/ $e_2$ \verb/)/ \verb|>>| \ldots{}
~~$\Longrightarrow$~~
\ldots{} \verb|>>| $e_1$ \verb|>>| \ldots{} \verb/|/
\ldots{} \verb|>>| $e_2$ \verb|>>| \ldots{}
\end{indpar}

A {\em type-inclusion-sequence}
\begin{center}
$t_1$\verb|>>|$t_2$\verb|>>|$t_3$\verb|>>|\ldots\verb|>>|$t_n$
\end{center}
intuitively
describes a hierarchical data block structure: a date block $b_1$ of type
$t_1$ contains a subblock $b_2$ of type $t_2$ and this contains a
subblock $b_3$ of type $t_3$, etc., until we get to the smallest
data block $b_n$ of type $t_n$ which in turn contains the location
being qualified.  Furthermore, $b_1$ is allocated
to the stack memory, to static memory, or to a heap memory,
and is disjoint from every other data block so allocated
that has a type incompatible with $t_1$.
Similarly $b_2$ is a child of $b_1$ in the hierarchical
containment scheme, and is disjoint from every other child subblock
of $b_1$ that has a type incompatible with $t_2$.   Etc. through $b_n$
and $t_n$.

Formally two data types overlap if they are compatible, which means
that a location of one type may be viewed as having the other type.
Two {\em type-inclusion-sequences} overlap if one is an initial
segment of the other, where compatible types are viewed as being equal.
For example `\verb|mydata>>subdata1>>int|' and `\verb|mydata>>subdata1|'
overlap but `\verb|mydata>>subdata1>>int|' and `\verb|mydata>>subdata2>>int|'
do not, assuming that \verb|subdata1| and \verb|subdata2| are
incompatible types.

Note that `\verb|type1>>type2|' refers to a datum of \verb|type2| located
inside a datum of \verb|type1| \underline{without} any intervening
data; i.e., there is \underline{no} \verb|type12| such that
`{\tt type1\verb|>>|\EOL type12\verb|>>|\EOL type2}'
can also be used to describe the datum.

Intuitively a canonical {\em type-overlap-expression} $e_1$\verb/|/$e_2$
describes a data block that can be described by either of the
{\em type-inclusion-sequences} $e_1$ or $e_2$.  Formally,
two canonical {\em type-overlap-expressions} overlap if any
{\em type-including-sequence} from the first overlaps any
{\em type-including-sequence} from the second.

TBD: include fixed integer offsets as part of inclusion?

\hspace*{-1em}\ikey{Parallel Qualifiers}%
{parallel qualifier}~~~~~Parallel
qualifiers are declared by the programmer and are used to tag
memory locations that may be flushed from software and hardware
caches in order to permit separate central processing units
to communicate through shared memory.

There are three L-Language statements that flush caches.  Each gives
a list of parallel qualifiers that names all the locations accessible
at the point of the statement that are tagged with these qualifiers.  An
example is:
\begin{indpar}[0.5em]\begin{verbatim}
declare parallel qualifier pdata, pflag
pdata float64 x[3]
pflag bool done_flag = false
. . . . . . .
// Output data
x[0] = 35.87
x[1] = -2.90
parallel flush write pdata
// Set done_flag indicating data has been output
done_flag = true
parallel flush write pflag
// Wait for data to be consumed
// Consumer clears done_flag after consuming data
loop: {
    spin ( 5000 ) // 5000 nanosecond spin
    parallel flush read pflag
    if ( done_flag == false ) break loop
}
// Data has been consumed
\end{verbatim}\end{indpar}

\end{indpar}

The `{\tt parallel flush}' command flushes \underline{both}
software and hardware caches.

\subsubsection{Access Qualifiers}
\label{ACCESS-QUALIFIERS}

Five access qualifiers are builtin:
`{\tt readwrite}', `{\tt readonly}', `{\tt writeonly}', `{\tt noaccess}',
and `{\tt unchecked}'.
The last two are protection qualifiers;
additional protection qualifiers may be declared.

\begin{indpar}

\hspace*{-1em}\ttkey{readwrite}, \ttkey{readonly}, \ttkey{writeonly},
\ttkey{noaccess}~~~~~Specifies whether execution may read or write a location.

\hspace*{-1em}\ttkey{unchecked}~~~~~A builtin protection qualifier
that is required to access the builtin operators that
perform explicit or implicit
type conversions.

\hspace*{-1em}\ikey{Protection Qualifiers}{protection qualifier}%
\label{PROTECTION-QUALIFIER}~~~~~Protection
qualifiers may be declared by the programmer.
Protection qualifiers may have names of the form
\begin{center}
{\em protection-qualifier-name} \verb|(| {\em type-expression} \verb|)|
\end{center}

A code block may call a function only if every protection qualifier
of the function is also protection qualifier of the code block, or
the special `{\tt unchecked}' qualifier is a qualifier of the code block.
Similarly a code block may access a memory location only if every
protection qualifier of the location is also a protection qualifier
of the code block, or
the special `{\tt unchecked}' qualifier is a qualifier of the code block.

\end{indpar}

Often qualifier expressions containing protection qualifiers are
used as function and memory location access qualifiers.  For this
reason, examples of access qualifiers are given in the next section.

\subsubsection{Qualifier Expressions}
\label{QUALIFIER-EXPRESSIONS}

A \key{qualifier expression} is an expression composed of qualifiers,
the operators `\ttmkey{\BAR}{in qualifier expression}'
and `\ttmkey{\&}{in qualifier expression}', and parentheses.
Qualifier expressions can be used as function access
qualifiers and memory location qualifiers.

In order for a function to be callable
or memory location to be accessible by a code block,
all of the function's access qualifier expressions
or the location's qualifier expressions must evaluate to true
according to the following rules.  First, each protection qualifier
is given the value true if it qualifies the code block and false otherwise.
Second, every non-protection qualifier is given the value true if and only if
doing so is necessary and sufficient to make all
the subexpressions containing the non-protection
qualifier true.  This last rule is applied from left to right.
Any non-protection qualifiers
given the value true for any qualifier expression
then qualify the function call or location when it is accessed by
the code block, while
non-protection qualifiers given the value false are ignored.

For example, given the code
\begin{indpar}[0.5em]\begin{verbatim}
define type type1 = (24,4)
declare protection qualifier constructor(type1)
(constructor(type1)|constant) type1 my_data
void function1 ( type1 *& arg1 )
void function2 ( constant type1 *& arg1 )
void function3 ( type1 *& arg1 ) constructor(type1)
. . . . . . .
function1 ( my_data)    // Erroneous; my_data is constant for
                        // function1.
function2 ( my_data)    // Legal; my_data is constant for
                        // function2 and so is arg1.
function3 ( my_data)    // Legal; my_data is NOT constant for
                        // function3.
\end{verbatim}\end{indpar}

Note the left-to-right clause in the rule for evaluating non-protection
qualifiers.  For example, given the code:
\begin{indpar}[0.5em]\begin{verbatim}
define type type1 = (24,4)
declare protection qualifier A
((A&readonly)|constant) type1 my_data1
((A&readonly)|(A&readwrite)|constant) type1 my_data2
\end{verbatim}\end{indpar}

then to a function with protection qualifier {\tt A}, {\tt my\_data1}
is {\tt readonly} and \underline{not}
{\tt constant}, whereas to all other functions, {\tt my\_data1}
is {\tt constant} and \underline{not} {\tt readonly}.  {\tt My\_data2}
behaves just like {\tt my\_data1}, as the subexpression `{\tt (A\&readwrite)}'
can never have any effect.

The precise specification for evaluating qualifier expressions
is as follows:

\begin{enumerate}
\item
A protection qualifier evaluates to true or false according to whether
the invoking code block has or does not have the qualifier.
\item
Any other qualifier evaluates to true when it is evaluated.
When this happens the qualifier is asserted for the function access
or location qualified by the qualifier expression.  Note that evaluation
may skip over subexpressions and thereby not evaluate qualifiers
therein.
\item
Evaluation of the subexpression $x$\verb/|/$y$ evaluates $x$ first.
If $x$ evaluates to true, the subexpression evaluates to true and
evaluation of $y$ is skipped.  If $x$ evaluates to false, $y$ is
evaluated and its value becomes the value of the subexpression.
Note that \verb/|/ is left associative:
$x$\verb/|/$y$\verb/|/$z$ $\equiv$
($x$\verb/|/$y$)\verb/|/$z$.
\item
Evaluation of the subexpression $x$\verb/&/$y$ evaluates $x$ first.
If $x$ evaluates to false, the subexpression evaluates to false and
evaluation of $y$ is skipped.  If $x$ evaluates to true, $y$ is
evaluated and its value becomes the value of the subexpression.
Note that \verb/&/ is left associative:
$x$\verb/&/$y$\verb/&/$z$ $\equiv$
($x$\verb/&/$y$)\verb/&/$z$.
\end{enumerate}

\subsubsection{Inline Qualifiers}
\label{INLINE-QUALIFIERS}

To `\key{inline}' a function is to replace the code calling the function
with the code of the function body.
For optimal performance functions should be inlined, so that
is the default.

Inlining is controlled by two builtin qualifiers: `\ttkey{inline}' and
`\ttkey{outline}'.  These are applied to function bodies and other
code blocks.  An `{\tt inline}' code block is inlined and an
`{\tt outline}' block not inlined.  An `{\tt outline}' block may
be nested inside an `{\tt inline}' block, but an `{\tt inline}'
block may \underline{not} be nested inside an `{\tt outline}' block.

An example is:
\begin{indpar}[0.5em]\begin{verbatim}
void function1 ( int32 x ) inline { ... }
void function2 ( int32 x ) outline { ... }
void function3 ( int32 x ) inline:

    if ( x > 0 ):
        ... inlined code ...

    else outline:
        ... outlined code ...

\end{verbatim}\end{indpar}

Calls to {\tt function1} are inlined, those to
{\tt function2} are not, while those to {\tt function3}
are inlined except for the part executed when {\tt x<=0}
which is treated as an out-of-line function.

Recursive inlining is permitted provided it does not
lead to unbounded code.  For example:
\begin{indpar}[0.5em]\begin{verbatim}
int32 function1 ( int32 n ) inline { ... }:
int32 recurse ( int32 n ) inline:
    if ( n == 0 ) return 0
    else return function1 ( n ) + recurse ( n - 1 )

int32 function2 ( void ) outline:
    return recurse ( 5 )
        // Works, inlining recurse 6 times.

int32 function3 ( int32 n ) outline:
    return recurse ( n )
        // Fails by producing an unbounded amount of code.

\end{verbatim}\end{indpar}

Inlining recursion works because while inlining code
the compiler computes compile time constant expressions
and uses the results to eliminate conditionalized code.

It is possible to control inlining to some degree
by associating compile time constant integers with
the `{\tt inline}' and `{\tt outline}' qualifiers.  If
a code block with qualifier `\verb|outline(|$m$\verb|)|'
calls a function whose code block has
qualifier `\verb|inline(|$n$\verb|)|', then the function
code block is inlined if and only if $m\geq n$.  Thus a
higher $m$ associated with and `{\tt output}' code block inlines more
code into the out-of-line block, and a higher $n$ associated
with an `{\tt inline}' code block makes it less likely the
code block will be inlined.

For example:
\begin{indpar}[0.5em]\begin{verbatim}
int32 function1 ( int32 n ) inline(1) { ... }
int32 function2 ( int32 n ) inline(2) { ... }
int32 function3 ( int32 n ) outline(1):

    function1(...) // Inlined.
    function2(...) // NOT inlined.

int32 function4 ( int32 n ) outline(2):

    function1(...) // Inlined.
    function2(...) // Inlined.

\end{verbatim}\end{indpar}

The numbers associated with the `{\tt inline}' and
`{\tt outline}' qualifiers are called
\key{optimization indices}.   Optimization indeces
default to the value {\tt 1}; e.g., `{\tt inline}'
is equivalent to `{\tt inline(1)}'.

\subsubsection{Qualifier Transitivity and Defaults}
\label{QUALIFIER-TRANSITIVITY-AND-DEFAULTS}

A qualifier may be declared to be \key{transitive qualifier},
meaning that if the qualifier is given as a function access
qualifier it will automatically be attached as a code block
qualifier of the function body.

A qualifier may be declared to be \ikey{default function access qualifier},
meaning that it is applied as a function access qualifier by
default to all functions that do not have a given
\key{counter qualifier}.  The counter qualifier merely specifies
that the default does not apply to a particular function.

Usually it is protection qualifiers that are declared to be transitive
or default function access qualifiers.

\subsubsection{Qualifier Inheritance}
\label{QUALIFIER-INHERITANCE}

\subsection{Aliasing and Containers}
\label{ALIASING-AND-CONTAINERS}

For purposes of optimization it is important to be able to accurately
determine if two pointers may point at overlapping data.  If they
do, data loaded into a register using one pointer may become invalid
when the other pointer is used to write data.
When two pointers point at overlapping data, they are said to
`\mkey{aliased}{pointers}'.

The L language uses the concept of type non-overlap and container types
to track aliasing.  Consider the pointer definitions:
\begin{indpar}[0.5em]\begin{verbatim}
define type Tx ...
define type TxC ...
define type Ty ...
define type TyC ...
. . . . .
adr xp -> Tx @ TxC x
adr yp -> Ty @ TyC y
\end{verbatim}\end{indpar}
If it is known that data of types {\tt Tx} and {\tt Ty} cannot
overlap, then the pointers cannot be aliased.  But this is also true
if it is known that data of types {\tt Tx} and {\tt TyC} cannot overlap,
or data of types {\tt TxC} and {\tt Ty} cannot overlap, or data of
types {\tt TxC} and {\tt TyC} cannot overlap.

Type overlap is represented by two relations. The {\tt T1 <@ T2} relation
means that a datum of type {\tt T1} can be completely contained inside
a datum of type {\tt T2}.  This relation is assumed to be transitive and
reflexive, and it is possible for both {\tt T1 <@ T2} and {\tt T2 <@ T1}
without {\tt T1} and {\tt T2} being the same type.
The second
relation {\tt T1 \TILDE@ T2} means that a datum of type {\tt T1} may overlap
a datum of type {\tt T2} without either data being contained in the other.
This relation is assumed to be reflexive and symmetric but not transitive.

These relations are specified by including the following in the context
of a code block that needs to determine whether two pointers are aliased.
Whenever {\tt T1 @ T2} appears in the context, {\tt T1 <@ T2} is added
to the {\tt <@} relation.  Whenever a type {\tt T1} is defined to be a subtype
of the type {\tt T2}, then {\tt T1 <@ T2} is added to the {\tt <@} relation.
The last of adding to the {\tt <@} relation, and the only way of adding
to the {\tt \TILDE@} relation, is with statements of the form:
\begin{center}
{\tt declare overlap}
{\em type-name}
\{ {\tt @} $|$ {\tt <@} $|$ {\tt <@>} $|$ {\tt \TILDE@} \}
{\em type-name}
\end{center}

where here {\tt @} is taken to mean {\tt <@} and
{\tt T1 <@> T2} is taken to mean that both {\tt T1 <@ T2} and
{\tt T2 <@ T1}.

Then two types {\tt T1} and {\tt T2} are defined to be non-overlapping
if none of the following is given data or deducible from given data
using transitivity of {\tt <@} or symmetry or reflexivity of {\tt \TILDE@}:
{\tt T1 <@ T2},
{\tt T2 <@ T1},
{\tt T1 \TILDE@ T2}.

Pointer data types are treated a bit differently.  Pointers
of types {\tt T1P -> T1D} and {\tt T2P -> T2D} are defined to be
non-overlapping if either {\tt T1P} and {\tt T2P} are non-overlapping
or if {\tt T1P } and {\tt T2P} are identical and
{\tt T2D} and {\tt T2D} are non-overlapping.

[TBD: virtual containers]

\subsection{Memory Channels}
\label{MEMORY-CHANNELS}

A \key{memory channel} is a mechanism for accessing a set of blocks in RAM
that permits blocks to be announced substantially in advance of being
accessed.  Thus memory channels implement `\key{look ahead}' for
memory accesses.

A memory channel implements a \key{window}, which is a
structured set of elements each associated with a member of
some data set.  Each window element contains a
\key{block descriptor} that holds the address and length of the
memory block that contains the data associated with the element.
Block descriptors can also be marked as
\mkey{empty}{block descriptor}, meaning there is no block to be accessed.
The window has a \key{reference point}, and window elements are addressed
relative to this reference point.  There are shift operations that move
the reference point to a nearby window element.

Although we talk about blocks here, a block can be just a numeric array
element, and can be as small as a single bit.  Although we talk about
each element of a memory channel window having its own block descriptor,
an actual memory channel may use only block group descriptors, each of
which functions as a group of more than one individual element
block descriptor.

A memory channel is stored in a cluster.  As such it is mostly an
inline construction, though it can be passed to or returned from a
function, and the function can be all or partly out-of-line.

The most common type of memory channel has a window that appears to be
an array with \ttdmkey{dimensions}{of memory channel},
\ttdmkey{lower\_bound}{of memory channel}{\tt [}$i${\tt ]},
and \ttdmkey{upper\_bound}{of memory channel}{\tt [}$i${\tt ]}
being defined memory channel members.  Such are called
\key{array windows}.
If the memory channel cluster name is $M$, the window elements are
referred to by $M$\verb|[|$i_0$\verb|,|$i_1$\verb|,|\ldots\verb|]|,
with $M$\verb|[0,0,|\ldots\verb|]| being the \key{reference point}.

The reference point can be shifted along any of the window's
dimensions by the command
$M$\ttdkey{center}{\tt [}$i_0${\tt ,}$i_1${\tt ,}\ldots{\tt ]}, which shifts
the window so that what was
$M${\tt [}$i_0${\tt ,}$i_1${\tt ,}\ldots{\tt ]} becomes
$M${\tt [0,0,}\ldots{\tt ]}.

Creating memory channels and completely reseting their reference points
are specific to the type of memory channel, and are not covered in
this section.

For most kinds of memory channels, block descriptors are computed
automatically when channel is created, when the window is
shifted, or when the data of
a neighboring window element is arrives from memory.
Immediately after a block descriptor is created, a read-ahead of
the block is initiated.  This read-ahead overlaps computation that
does not use the block contents.

If a memory channel accesses arrays stored in memory,
the channel block descriptors can be computed from the array coordinates
of the reference point.  Other memory channels use the contents
of a block to compute the block descriptors of neighboring blocks
in the window.

An example of the latter is a binary tree memory channel.
Let $M$ be such a channel, and let `{\tt .L}' denote the left
child of a binary tree element, `{\tt .R}' the right child, and
`{\tt .P}' the parent.  Then $M\!$\verb|.L.R| denotes the right
child of the left child of the reference point, $M\!$\verb|.P.L|
denotes the left child of the parent of the reference point,
and $M\!$\verb|.P.L.center| moves the reference point to this last element.
The window of such a memory channel might contain the depth 2
subtree of the reference point plus that closest 4 ancestors of the
reference point if these have been visited.  When the reference
point is moved, as soon as the reference point element has been
read from memory, the descriptors for its children are built and
the read of the children is initiated in parallel with other
computation.  When the children arrive from memory, the descriptors
of their children are built and reads of the data pointed at
are initiated.\footnote{All this can actually be done with modern
hardware: code is executed to read the reference point children and initiate the
reads of their children, and a modern processor will automatically
save the code that
executes when a read of a reference point child completes and execute
other code in parallel until the read does complete.}

Some standard memory channel types are built into the L-Language.
Others can be defined by users.

\section{To Do}

How can dynamically initialized locations be static.

Indirect address protocol.
\label{INDIRECT-ADDRESS-PROTOCOL}

Functions.
\label{FUNCTIONS}

Threads.
\label{THREADS}

Function calls should be able to serve as lvalues so
\verb|x[i]=z| can update gc flags.

\appendix

\section{Aliasing Hardware}
\label{ALIASING-HARDWARE}

The ultimate solution to the aliasing problem is new hardware.
At its simplest, registers, which currently hold a datum,
are replaced by triples of registers which hold a datum,
an address, and selection codes.  The register datum equals the value
of the memory location at the register address.  The selection
codes determine which part of this memory location is read or written
when the register is read or written.  If any memory location is
changed, the address of the location is checked against all the
register addresses, and if any match, the corresponding register
data are changed.

This is, however, not sufficient, because sometimes one register
address is a function of another register's datum.  For example,
consider the unchecked code:
\begin{indpar}[0.5em]\begin{verbatim}
struct S { ...; int32 m; ... }
S * * x
S * *& y = * x
int32 *& z = y->m
\end{verbatim}\end{indpar}
If we consider {\tt x}, {\tt y}, and {\tt z} to be registers,
the address of {\tt y} equals the value of {\tt x}, and the
address of {\tt z} equals the value of {\tt y} plus the offset of
{\tt m} in {\tt S}.

If the value of {\tt x} changes, this changes the address of {\tt y},
which may change the datum of {\tt y} and that may change the value
of {\tt y}.  If the value of {\tt y} changes, this changes the address
of {\tt z}, which may change the datum and value of {\tt z}.

The way we accommodate this is to use the selection codes of {\tt y}
to specify that the address of {\tt y} contains the value of {\tt x}
as an additive component, so
that if the value of {\tt x} is changed by adding $\Delta${\tt x}
then the address of {\tt y} should be changed by adding $\Delta${\tt x}.
And similarly the selection codes
of {\tt z} specify that the address of {\tt z}
contains the value of {\tt y} as an additive component.

So why should we bother with automatically updating
additive inclusions of one value in the
address of another value, and not bother with other expressions.
The reason is that expressions such as
`\verb|(*x)->m|' are likely to be reused frequently in code (actually,
in automatically generated code) and
therefore need to be cached, whereas an expressions of the form
`\verb|c*d|' will be reused comparatively rarely code
and therefore are not worth special hardware.


\bibliographystyle{plain}
\bibliography{layered-l-language}

\printindex

\end{document}
