% The Layered Low (L) Programming Language
%
% File:         layered-l-language.tex
% Author:       Bob Walton (walton@acm.org)
% Version:      1a
  
\documentclass[12pt]{article}

\usepackage[T1]{fontenc}
\usepackage{times}
\usepackage{makeidx}
\usepackage{pictex}

\makeindex

\setlength{\oddsidemargin}{0in}
\setlength{\evensidemargin}{0in}
\setlength{\textwidth}{6.5in}
\raggedbottom

\setlength{\unitlength}{1in}

\pagestyle{headings}
\setlength{\parindent}{0.0in}
\setlength{\parskip}{1ex}

\setcounter{secnumdepth}{5}
\setcounter{tocdepth}{5}
\newcommand{\subsubsubsection}[1]{\paragraph[#1]{#1.}}
\newcommand{\subsubsubsubsection}[1]{\subparagraph[#1]{#1.}}

% Begin \tableofcontents surgery.

\newcount\AtCatcode
\AtCatcode=\catcode`@
\catcode `@=11	% @ is now a letter

\renewcommand{\contentsname}{}
\renewcommand\l@section{\@dottedtocline{1}{0.1em}{1.4em}}
\renewcommand\l@table{\@dottedtocline{1}{0.1em}{1.4em}}
\renewcommand\tableofcontents{%
    \begin{list}{}%
	     {\setlength{\itemsep}{0in}%
	      \setlength{\topsep}{0in}%
	      \setlength{\parsep}{1ex}%
	      \setlength{\labelwidth}{0in}%
	      \setlength{\baselineskip}{1.5ex}%
	      \setlength{\leftmargin}{1.0in}%
	      \setlength{\rightmargin}{1.0in}}%
    \item\@starttoc{toc}%
    \end{list}}
\renewcommand\listoftables{%
    \begin{list}{}%
	     {\setlength{\itemsep}{0in}%
	      \setlength{\topsep}{0in}%
	      \setlength{\parsep}{1ex}%
	      \setlength{\labelwidth}{0in}%
	      \setlength{\baselineskip}{1.5ex}%
	      \setlength{\leftmargin}{1.0in}%
	      \setlength{\rightmargin}{1.0in}%
	      }%
    \item\@starttoc{lot}%
    \end{list}}

\catcode `@=\AtCatcode	% @ is now restored

% End \tableofcontents surgery.

\newcommand{\CN}[2]%	Change Notice.
    {\hspace*{0in}\marginpar{\sloppy \raggedright \it \footnotesize
     $^{\mbox{#1}}$#2}}
    % Change notice.

\newcommand{\TT}[1]{{\tt \bfseries #1}}

\newcommand{\key}[1]{{\bf \em #1}\index{#1}}
\newcommand{\mkey}[2]{{\bf \em #1}\index{#1!#2}}
\newcommand{\skey}[2]{{\bf \em #1#2}\index{#1}}
\newcommand{\smkey}[3]{{\bf \em #1#2}\index{#1!#3}}
\newcommand{\ikey}[2]{{\bf \em #1}\index{#2}}
\newcommand{\ttkey}[1]{\TT{#1}\index{#1@{\tt #1}}}
\newcommand{\tttkey}[1]{\TT{<#1>}\index{#1@{\tt <#1>}}}
\newcommand{\ttikey}[2]{\TT{#1}\index{#2@{\tt #2}}}
\newcommand{\ttmkey}[2]{\TT{#1}\index{#1@{\tt #1}!#2}}

\newcommand{\ttdkey}[1]{\TT{.#1}\index{#1@{\tt .#1}}}
\newcommand{\ttdmkey}[2]{\TT{.#1}\index{#1@{\tt .#1}!#2}}
\newcommand{\ttindex}[1]{\index{#1@{\tt #1}}}
\newcommand{\ttmindex}[2]{\index{#1@{\tt #1}!#2}}
\newcommand{\emkey}[1]{{\bf \em #1}\index{#1@{\em #1}}}
\newcommand{\emlkey}[2]{{\bf \em #1#2}\index{#1@{\em #1}!#2@{\em #2}}}
\newcommand{\emskey}[2]{{\bf \em #1#2}\index{#1@{\em #1}}}
\newcommand{\emmkey}[2]{{\bf \em #1}\index{#1@{\em #1}!#2}}
\newcommand{\emsmkey}[3]{{\bf \em #1#2}\index{#1@{\em #1}!#3}}
\newcommand{\emindex}[1]{\index{#1@{\em #1}}}

\newcommand{\itemref}[1]{\ref{#1}$^{p\pageref{#1}}$}
\newcommand{\pagref}[1]{p\pageref{#1}}
\newcommand{\pagnote}[1]{\,\textsuperscript{p\pageref{#1}}}

\newcommand{\EOL}{\penalty \exhyphenpenalty}

\newcommand{\STAR}{{\Large $^\star$}}
\newcommand{\PLUS}[1][]{{$^{+#1}$}}
\newcommand{\QMARK}{{$^{\,\mbox{\footnotesize ?}}$}}
\newcommand{\OPEN}{{$\{$}}
\newcommand{\CLOSE}{{$\}$}}


\newlength{\figurewidth}
\setlength{\figurewidth}{\textwidth}
\addtolength{\figurewidth}{-0.40in}

\newsavebox{\figurebox}

\newenvironment{boxedfigure}[1][!btp]%
	{\begin{figure*}[#1]
	 \begin{lrbox}{\figurebox}
	 \begin{minipage}{\figurewidth}

	 \vspace*{1ex}}%
	{
	 \vspace*{1ex}

	 \end{minipage}
	 \end{lrbox}
	 \begin{center}
	 \fbox{\hspace*{0.1in}\usebox{\figurebox}\hspace*{0.1in}}
	 \end{center}
	 \end{figure*}}

\newenvironment{indpar}[1][0.3in]%
	{\begin{list}{}%
		     {\setlength{\itemsep}{0in}%
		      \setlength{\topsep}{0in}%
		      \setlength{\parsep}{1ex}%
		      \setlength{\labelwidth}{#1}%
		      \setlength{\leftmargin}{#1}%
		      \addtolength{\leftmargin}{\labelsep}}%
	 \item}%
	{\end{list}}

\newenvironment{itemlist}[1][0.5in]%
	{\begin{list}{}{\setlength{\labelwidth}{#1}%
	                \setlength{\leftmargin}{#1}%
		        \addtolength{\leftmargin}{\labelsep}}}%
	{\end{list}}

\begin{document}
        
\begin{center}

{\Large
The Lower (L) Layered Programming Language \\[0.5ex]
(Draft 1a)}

\medskip

Robert L. Walton\footnote{This document is dedicated to the memory
of Professor Thomas Cheatham of Harvard University.}

July 25, 2014
 
\end{center}

{\small \tableofcontents}

\newpage

\section{Introduction}

This document describes the Lower Layer Programming Language, or
L-Language.  See the Introduction to the Layered
Programming Languages for basic syntax and for an overview of the related
Middle Layer M-Language and Higher Layer H-Language.

The L-Language is intended to be a target language for compilers of
higher level languages.  As such it is optimized first to be an easy to 
use as a target language, and second to be easy to
compile into reasonably efficient debuggable assembly language code.

The L-Language is similar to the C programming language
differs in the following ways:

\begin{enumerate}

\item
L-Language facilitates isolating untype-safe code into small inline functions.

\item
All L-Language data types are builtin number types or subtypes of
builtin number types.  All data managed directly by L-Language consists
of single numbers allocated to the stack.

Other memory is handled by code
encapsulated in small untype-safe inline functions by the programmer.
Memory references other than those to single stack allocated numbers
are treated in a manner reminiscent of input/output.

\item
The type `\TT{type}' is first class, is a subtype of integers, and
is assigned a unique constant address value for each type.  Constant
argument values, such as the value of a `\TT{type}' valued
argument, can be used to select overloaded functions.

\item
All L-Language functions are inline.  Programmer written untype-safe
functions are used to implement out-of-line function calls.

\item
L-Language associates formatting functions
with data types for use in displaying and in some cases inputting
data during debugging.

\item
L-Language uses a single-assignment style of program structure
to encourage
efficient code that leaves an audit trail for debugging in the stack.
Partial evaluation of expressions and inline code is supported without
compromising debuggability.

\item
L-Language uses the Layered Language Module Structure to manage
multi-file source code in a well-ordered and type-safe manner.


\end{enumerate}


\section{Types}
\label{TYPES}

The basic builtin types are the number and \TT{void} types.
All defined types are subtypes of these types.
There is one builtin define type, the `\TT{type}' type.

\subsection{Number Types}
\label{NUMBER-TYPES}

Numbers are sequences of bits, and each number is one of
four types, unsigned integer, signed integer, floating point number,
and address:

\begin{indpar}
\begin{tabular}{p{1in}@{~~~~~~}p{2in}@{~~~~~~}l}
\bf Name & \bf Kind & \bf Length \\[2ex]
\ttkey{uns8}		& unsigned integer & 8 bits \\
\ttkey{uns16}		& unsigned integer & 16 bits \\
\ttkey{uns32}		& unsigned integer & 32 bits \\
\ttkey{uns64}		& unsigned integer & 64 bits \\
\ttkey{uns128}		& unsigned integer & 128 bits
\end{tabular}

\begin{tabular}{p{1in}@{~~~~~~}p{2in}@{~~~~~~}l}
\ttkey{int8}		& signed integer & 8 bits \\
\ttkey{int16}		& signed integer & 16 bits \\
\ttkey{int32}		& signed integer & 32 bits \\
\ttkey{int64}		& signed integer & 64 bits \\
\ttkey{int128}		& signed integer & 128 bits
\end{tabular}

\begin{tabular}{p{1in}@{~~~~~~}p{2in}@{~~~~~~}l}
\ttkey{float16}		& IEEE floating point number & 16 bits \\
\ttkey{float32}		& IEEE floating point number & 32 bits \\
\ttkey{float64}		& IEEE floating point number & 64 bits \\
\ttkey{float128}	& IEEE floating point number & 128 bits
\end{tabular}

\begin{tabular}{p{1in}@{~~~~~~}p{2in}@{~~~~~~}l}
\ttkey{unsadr}		& unsigned integer & size of address \\
\ttkey{intadr}		& signed integer & size of address \\
\ttkey{adr}		& signed or unsigned integer & size of address \\
\end{tabular}
\end{indpar}

These are the \key{built in number types}:
\begin{indpar}
{\em builtin-number-type-name}
    \begin{tabular}[t]{@{}cl}
    ::= & \TT{uns8} $|$ \TT{uns16} $|$ \TT{uns32} $|$ \TT{uns64}
                    $|$ \TT{uns128} \\
    $|$ & \TT{int8} $|$ \TT{int16} $|$ \TT{int32} $|$ \TT{int64}
                    $|$ \TT{int128} \\
    $|$ & \TT{float16} $|$ \TT{float32} $|$ \TT{float64} $|$ \TT{float128} \\
    $|$ & \TT{unsadr} $|$ \TT{intadr} $|$ \TT{adr} \\
    \end{tabular}
\end{indpar}

The \mkey{length}{of number}
of a number is the number of its bits.  Numbers can have
different lengths: for example, unsigned integers can have
lengths of 8, 16, 32, 64, or 128 bits.

Numbers are stored in registers or in random access memory (RAM).

An \key{unsigned integer} of length $L$ is a binary integer with
$L$ binary digits (\skey{bit}s) and range from $0$ to $2^L-1$.

A \key{signed integer} of length $L$ is a two's complement integer
of length $L$ and range from $-2^{L-1}$ to $+2^{L-1}-1$.
This represents the integer $I$ in the given range
by the unsigned $L$-bit integer equal to $I~\mbox{modulo}~2^L$.

A \key{floating point number} of length $L$ is a floating point number
represented according to the IEEE 754 standard.  The sizes of exponents
and mantissas for various floating point number sizes are as follows:

\begin{center}
\begin{tabular}{l@{~~~~~~}l@{~~~~~~}l@{~~~~~~}r@{~~~~~~}r}
                   &              &              &             & \bf Maximum \\
\bf Floating Point & \bf Exponent & \bf Mantissa & \bf Decimal & \bf Decimal\\
\bf Number Size    & \bf Size     & \bf Size     & \bf Digits  & \bf Exponent
\\[2ex]
16 bits & 5 bits & 10 bits & 3.31 & 4.51 \\
32 bits & 8 bits & 23 bits & 7.22 & 38.23 \\
64 bits & 11 bits & 52 bits & 15.95 & 307.95 \\
128 bits & 15 bits & 112 bits & 34.02 & 4931.77 \\
\end{tabular}
\end{center}

An \key{address} holds a RAM byte address.
A address is a 32-bit or 64-bit unsigned integer whose size
is determined by the target machine.  Some of high order bits may be
required to be all 0's or all 1's, depending upon
the target machine.  The \TT{unsadr} and \TT{intadr}
unsigned and signed integer types of the same size
as an address are provided for storing indices and offsets.

A reasonable assumption for 64-bit addresses is that only the low
order 48-bits of the address are actually used.  This assumption can
be used to put other information in the high order 16 bits
of a 64-bit number containing an address.  For example, an address
can be embedded in a 64-bit floating point NaN.  The L-Language does
\underline{not} depend upon this assumption, but does provide a builtin function
that takes as input a 64-bit integer and two small integers, $L$ and $S$,
and returns a 64-bit address containing
the byte address equal to the low order $L$ bits of the input integer
left shifted by $S$.  The output may have undefined high order bits if
the hardware ignores them when using the output to address memory.
For example, if the hardware ignores the high order 20 bits, and uses only the
low order 44 bits,
this function would just copy its input 64-bit integer to its output
if $L\geq 44$ and $S=0$.\footnote{The I86 64-bit architecture
uses only the low order 48 bits of an address, but \underline{requires}
the high order 17 bits to all be the same, either all 1's or all 0's.
However, as it is unlikely that there will ever be an allocated memory region
that includes address 0 in its interior, it makes no significant difference
whether we consider addresses to be unsigned or signed.}

The following \underline{implicit} conversion operators are defined:
\begin{center}

\begin{tabular}{rcl}
\hspace*{2.0in} & & \hspace*{2.0in} \\[-2ex]
\TT{uns}$x$ & \TT{<{}<++} & \TT{uns}$y${\em -value} \\
\TT{int}$x$ & \TT{<{}<++} & \TT{uns}$y${\em -value} \\
\TT{float}$x$ & \TT{<{}<++} & \TT{uns}$y${\em -value} \\
\TT{int}$x$ & \TT{<{}<++} & \TT{int}$y${\em -value} \\
\TT{float}$x$ & \TT{<{}<++} & \TT{int}$y${\em -value} \\
\TT{float}$x$ & \TT{<{}<++} & \TT{float}$y${\em -value} \\
\end{tabular}

where $x>y$

\end{center}

For all these implicit conversions no information is lost from
the value when it is converted to the new type.

The following \underline{unchecked} conversion operator is
defined:
\begin{center}

$t1${\em -type-value} ~~ \TT{**>{}>} ~~ $t2$

for any two builtin number types $t1$ and $t2$

\end{center}

Here the results are undefined if the value cannot be expressed
as type $t2$.


\subsection{The Void Type}
\label{THE-VOID-TYPE}

A \ttkey{void} type variable has no value.  No value can be
assigned to such a variable.  Nevertheless such variables have
uses, for example, as base variable in clusters (\pagref{CLUSTERS}).
When used as the base variable of a cluster, an assignment statement
may appear to be assigning a value to a \TT{void} variable, but
in fact it is assigning values to cluster members.


\subsection{Defined Types}
\label{DEFINED-TYPES}

A statement of the following form defines a type:
\begin{center}
\TT{type} {\em defined-type-name} \TT{is} {\em builtin-type-name}
\end{center}
Here the builtin type is called the \key{underlying type} of the
\key{defined type}.

A value of a defined type may be converted to or from its underlying
type by the \underline{unchecked} \TT{**>{}>} operators:
\begin{center}
\begin{tabular}{rcl}
{\em defined-type-value} & \TT{**>{}>} & {\em builtin-type-name} \\
{\em builtin-type-value} & \TT{**>{}>} & {\em defined-type-name} \\
\end{tabular}
\end{center}
These operators do not change the actual value, but merely its
compile-time type.


\subsection{Type Values}
\label{TYPE-VALUES}

Each type is associated with a unique unsigned integer
value of type `\TT{type}'.  This value is known at both
run time and compile time.

Functions can be overloaded based on the number and types
of their arguments and also on the values of arguments that
are known at compile time.  Thus different functions with
the common prototype
\begin{indpar}\begin{verbatim}
(unsadr length) = size of ( type T )
\end{verbatim}\end{indpar}
may be defined for different values of \TT{T}.  
This is done automatically for the following functions:

\begin{indpar}[0.5in]
\hspace*{-0.3in}{\tt (unsadr size) = \ttkey{size of}
			( type T )} \\
Returns \TT{size}, the number of bytes in a value of type \TT{T}.

\hspace*{-0.3in}{\tt (unsadr alignment) = \ttkey{alignment of}
			( type T )} \\
Returns \TT{alignment}, a strictly positive integer.
The address of a value of type \TT{T} in RAM should
optimally be a multiple of the \TT{alignment}.
The alignment of builtin types is usually their size.

\hspace*{-0.3in}{\tt (type B) = \ttkey{base of}
			( type T )} \\
Returns the parent type \TT{B} of the type \TT{T}.  For defined
types this is the associated builtin type.  For builtin types, it is
\TT{T} itself.
\end{indpar}

\section{Assignment Statements}
\label{ASSIGNMENT-STATEMENTS}

In the following {\em expressions} produce a list of one
or more values of well defined types.

Functions have stack frames that contain \underline{all} their variables.
These variables are allocated and their values are computed by
assignment statements:

\begin{indpar}
\emkey{value-assignment-statement}
    ::= {\em variable-list} \TT{=} {\em expression}
\\[0.5ex]
\emkey{variable-list}
    \begin{tabular}[t]{@{}rl}
    ::= & {\em type-name} {\em variable-name} \\
    $|$ & \TT{(} {\em type-name} {\em variable-name}
	   \{ \TT{,} {\em type-name} {\em variable-name} \}\STAR{} \TT{)} \\
    \end{tabular}
\end{indpar}

This allocates new variables of the given {\em variable-names} and types
to the stack and stores values from the {\em expression} in the variables.
The {\em expression} produces a list of values, and this list must have
at least as many elements as there are variables.  Each variable is
assigned its value in order, and this value must be implicitly
convertable to the variable's type.  If the {\em expression} produces
too many values, the excess values at the end of the value list will be
discarded.

The {\em expression} may only input values that are constants or are
the values of other variables in the same function stack frame (including
arguments passed to the function).  In particular,
no input can come from RAM memory outside the function stack frame.

Implicitly when a variable value is set this way, the value is first
computed in a register, then stored in the stack, and for some time
afterwards the value is left in the register which becomes a cache
on the stack value.  This is hidden from the programmer.

If a variable value is copied from another variable, usually the
new variable simply becomes a compile-time alias for the old variable.
No executable copy code is created, no new register is allocated, and no new
value is pushed into the stack.

Similarly if a variable value can be computed from previous variable
values by in-line code, 
its value may not be pushed into the stack.  Instead its
value can be recovered at any time by repeating the computation
of the variable from the other values.

Two variables in the same block of code may not have the same
name.  Thus variables are only assigned values once.

However, if an assignment statement uses the word `\TT{next}' in
place of the type of a variable being assigned a value, and
if the variable previously exists, a new variable with the same
name and type as the previous variable is allocated.

Note that `\TT{next}' behaves differently for loop interation
variables (see below) than it does for other variables.

\subsection{Block Assignment Statements}
\label{BlOCK-ASSIGNMENT-STATEMENTS}

Another variant of the assignment statement is:

\begin{indpar}
\emkey{block-assignment-statement}
    ::= \begin{tabular}[t]{l}
        {\em variable-list} \TT{=:} \\
	~~~~ {\em executable-statement}\STAR{}
	\end{tabular}
\end{indpar}

The {\em executable-statements} in the block can include
{\em assignment-statements} in which the {\em type-names}
of variables in the {\em executable-statement's} {\em variable-list}
are omitted if these variables have the same name as a variable
in the {\em block-assignment-statement's} {\em variable-list}.
These {\em executable-statements} assign values to the variables
in the {\em block-assignment-statement's} {\em variable-list}.
Furthermore, every variable in the
{\em block-assignment-statement's} {\em variable-list} must be
assigned a value by one of the {\em executable-statements}.
Also the values of the variables in the {\em variable-list} cannot
be input to any of the {\em executable-statements}.

Variables allocated in the block are not visible outside the
block.

\subsection{Conditional Assignment Statements}
\label{CONDITIONAL-ASSIGNMENT-STATEMENTS}

Conditional assignment statements are similar:

\begin{indpar}
\emkey{conditional-assignment-statement}
    ::= \begin{tabular}[t]{l}
        {\em variable-list} \TT{= if:} \\
	~~~~ {\em condition-expression}\TT{:} \\
	~~~~~~~~~ {\em executable-statement}\STAR{} \\
	~~~~ {\em condition-expression}\TT{:} \\
	~~~~~~~~~ {\em executable-statement}\STAR{} \\
	~~~~ \ldots\ldots\ldots\ldots\ldots\ldots\ldots\ldots\ldots \\
	~~~~ \TT{else:} \\
	~~~~~~~~~ {\em executable-statement}\STAR{} \\
	\end{tabular}
\end{indpar}

In this case the {\em condition-expressions} are evaluated in order
until either one evaluates to true or the `\TT{else}' condition is
reached, and then the {\em executable-statements} subject to the
first true condition (a.k.a the \key{subblock} of that condition)
are executed.  Each separate subblock of
{\em executable-statements} must set \underline{every} variable
in the {\em condition-assignment-statement's} {\em variable-list},
with exceptions for defaults described next.

In addition to an `\TT{else:}' subblock,
a {\em conditional-assignment-statement} may have one or
more `\ttkey{default}\TT{:}' subblocks.  Each assigns one or more
variables in the {\em variable-list}.  A default subblock
executes after another subblock executes if and only if all of the
{\em variable-list} variables set by the default subblock have
\underline{not} been set yet.  This means that multiple subblocks
may execute; if this happens, the order of execution is the order
in which the subblocks appear in the program, but variables allocated
by one subblock are not visible outside that subblock.

\subsection{Loop Assignment Statements}
\label{LOOP-ASSIGNMENT-STATEMENTS}

Lastly loop assignment statements are similar:

\begin{indpar}
\emkey{loop-assignment-statement}
    ::= \begin{tabular}[t]{l}
        {\em result-variable-list}
	    \TT{= loop} {\em iteration-variable-list}\TT{:} \\
	~~~~ \TT{initially}\TT{:} \\
	~~~~~~~~~ {\em executable-statement}\STAR{} \\
	~~~~ \TT{while} {\em condition-expression}\TT{:} \\
	~~~~~~~~~ {\em executable-statement}\STAR{} \\
	~~~~ \TT{finally}\TT{:} \\
	~~~~~~~~~ {\em executable-statement}\STAR{} \\
	\end{tabular}
\end{indpar}

This is somewhat like the conditional assignment statement
except the {\em executable-statements} are repeated.  The
{\em executable-statements} qualified by `\TT{initially}'
only execute on the first repetition, and they must
set all the variables in the {\em iteration-variable-list}.
The {\em executable-statements} qualified by `\TT{while}
{\em condition-expression}' only execute if the
{\em condition-expression} evaluates to true, and they
for every variable \TT{v} in the {\em iteration-variable-list}
they must set the variable `\TT{next v}', as if it were in
this list.  The `\TT{next v}' variable is in fact the iteration
variable in the next iteration of the loop.  Lastly, when
the {\em condition-expression} evaluates to false, the
{\em executable-statements} qualified by `\TT{finally}'
execute, and they must set all the variables in the
{\em result-variable-list}.  At this point the loop stops
executing.

Loops are always in effect unravelled so multiple copies of
the loop exist at one time.  `\TT{next v}' refers to \TT{v} in the
next iteration of the loop, which exists simultaneously with
the current iteration.  Instead of setting \TT{v} in the
\TT{initially} statements and \TT{next v} in the \TT{while}
statements, it is permissible to set both \TT{v} and \TT{next v}
in the \TT{initially} statements and \TT{next next v} in the
\TT{while} statements.

There are always at least 4 iterations of the loop with variables
in the current function frame stack: the current iteration,
the previous iteration (if any), and the next two iterations.
Then the third iteration of the loop needs to start, the first
iteration may be garbage collected, as it were, and so forth.

\section{Functions}
\label{FUNCTIONS}

All functions are inline.  An inline function can call out-of-line
code in either a classical or non-classical manner.

Function definitions have the form:

\begin{indpar}
\emkey{function-definition}
    ::= \begin{tabular}[t]{l}
        {\em function-prototype}\TT{:} \\
	~~~~ {\em executable-statement}\STAR{} \\
	\end{tabular}
\\[1ex]
\emkey{function-prototype}
    \begin{tabular}[t]{@{}rl}
    ::= & {\em result-variable-list} \TT{=} {\em function-name}
	    {\em argument-variable-list}\QMARK{}\TT{:} \\
    $|$ & {\em function-name} {\em argument-variable-list}\QMARK{}\TT{:} \\
    $|$ & {\em function-name} {\em argument-variable-list}\QMARK{}
            \TT{=} {\em input-expression}\TT{:} \\
    \end{tabular}
\end{indpar}

A function call with a {\em result-variable-list}
is in effect replaced by a {\em block-assignment-statement}
with the {\em function-definition's} {\em result-variable-list} becoming the 
{\em block-assignment-statement's} {\em variable-list} and the
{\em function-definition's} {\em executable-statements} becoming the 
{\em block-assignment-statement's} {\em executable-statements}.
Actual argument values from the function call are assigned to variables in
the definition {\em argument-variable-list}.  The values produced in
the {\em result-variable-list's} variables become the values of the
function call when it is used as part of an {\em expression}.

A function call without any {\em result-variable-list} or
{\em input-expression} is just like a function call with an empty
{\em result-variable-list}.

A function call with an {\em input-expression} is just syntactic
sugar for a function call with neither a {\em result-variable-list}
or {\em input-expression}.  Specifically
\begin{center}
{\tt F(a1,a2,...)~=~(e1,e2,...)} \\
is syntactic sugar for \\
{\tt F(a1,a2,...,e1,e2,...)} \\
\end{center}



\subsection{Clusters}
\label{CLUSTERS}

A \key{cluster} is a group of related variables.  One variable of
the cluster is the \mkey{base variable}{of cluster} of the cluster, and the
other variables have names that are derived from the
name of the base variable using the syntax:


\begin{indpar}
{\em variable-name} ::=
    {\em base-name} {\em member-selector}$^\star$ \\[1ex]
{\em member-selector} \begin{tabular}[t]{@{}rl}
    ::= & {\tt .}{\em member-name} \\
    $|$ & {\tt [}{\em member-index-list}{\tt ]}
    \end{tabular} \\[1ex]
{\em member-index-list} \begin{tabular}[t]{@{}rl}
    ::= & {\em member-index} \\
    $|$ & {\em member-index} {\tt ,} {\em member-index-list}
    \end{tabular} \\[1ex]
{\em member-index} ::= {\em integer-constant-expression}
\end{indpar}

Thus a cluster
is like a structure, but it is a set of variables and not a
piece of memory.

More specifically, the members of the cluster can be named by
adding either a member name preceded by `{\tt .}' or a
`{\tt []}' bracketed list of
integer constant subscripts to
either the base variable name of the cluster or to another member name
of the cluster.  Two variable names with different base variable names
belong to different clusters.

The main feature of clusters is that members are passed to or returned
from functions implicitly when their the base variable is named
explicitly.  For example:
\begin{indpar}\begin{verbatim}
type pointer pair is void
    // if pointer pair pp then
    //   unsadr pp.begin points at the first element
    //   unsadr pp.end points just AFTER the last element
    //   type pp.type is type of element

// Declaration of function to allocate a vector of n T's.
//
( mem pointer pair pp,
  unsadr pp.begin, unsadr pp.end, type pp.type ) =
        mem allocate ( type T, uns32 n ):
    pp.type = T
    unsadr length = size of ( T ) * n
    pp.begin = allocate ( length )
    pp.end = pp.begin + length


// Prefix operator to dereference a pointer pair to read
// memory.
//
( mem pp.type out, unsadr out.adr  ) =
        mem "*" ( mem pointer pair pp,
                  type pp.type, unsadr pp.begin,
                  unsadr pp.end )
    out.adr = if:
        pp.begin < pp.end:
            out.adr = pp.begin
        else:
            fatal error
                ( "Deferencing empty pointer pair." )

// Prefix operator to dereference a pointer pair to
// write memory.
//
mem "*" ( mem pointer pair pp,
              unsadr pp.begin, unsadr pp.end,
              type pp.type ) =
        ( pp.type in ):
    if:
        pp.begin < pp.end:
            in --> pp.begin
        else:
            fatal error
                ( "Deferencing empty pointer pair." )

// Prefix operator to increment the begin pointer of a
// pointer pair.  Note the result is a NEW pointer pair.
//
( mem pointer pair pp2,
  unsadr pp2.begin, unsadr pp2.end, type pp2.type ) =
        mem "++" ( pointer pair pp,
                   type pp.type,
                   unsadr pp.begin, unsadr pp.end ):
      pp2.begin = pp.begin + size of ( T )
      pp2.end = pp.end
      pp2.type = pp.type

// Example usage:
//
    ... out-of-line function header ...

    // Vector of 2 int32's is allocated and the elements
    // are set equal to 100 and 101.
    //
    mem pointer pair pp = allocate ( int32, 2 )
    * pp = 100
    next mem pointer pair pp = ++ pp
    * pp = 101
    . . . .
    // Vector of 1000 float64's is summed.
    //
    mem pointer pair ppf = allocate ( float64, 1000 )
    ... set 1000 elements ...
    float64 sum = loop ( float64 partial sum,
                         mem pointer pair ppf2,
                         unsadr ppf2.begin,
                         unsadr ppf2.end,
                         unsadr ppf2.type ):
        initially:
            partial sum = 0
            ppf2.begin = pp.begin
            ppf2.end = ppf2.end
            ppf2.type = ppf.type
        while ppf2.begin < ppf2.end
            next partial sum = partial sum + * ppf2
            next ppf2 = ++ ppf2
        finally:
            sum = partial sum
\end{verbatim}\end{indpar}

Cluster member names may be used as prototype parameter
names, with the base of these names also being a prototype
parameter.  This specifies that the parameters are related
by being in the same cluster.

When this is done for arguments, the cluster base must
also be an argument, and the cluster member values
will be derived from the base, and \underline{must}
be omitted in calls.

When this is done for results, the cluster base must similarly
be a result, and the values of the designated
members of the cluster are set from the results.
Again the cluster members \underline{must} be omitted in calls.

A variable of \TT{void} type has no actual value
and must not be assigned one.  Such variable are only useful
as cluster bases or cluster members which have submembers.

In the above example `\TT{mem}' is a protection qualifier
which can be applied to both variables and functions.
A qualifier applied to a cluster base automatically is applied
to the cluster members.

The `\TT{mem}' qualifier has the special property that given a variable
\TT{v} of type \TT{T} and qualifier \TT{mem}, then if \TT{v} is allocated,
and \TT{v.adr} of type \TT{unsadr} is allocated and
assigned a value, the base \TT{v} will also be automatically allocated
and assigned as its value the value read from RAM address \TT{v.adr}.
It is an error in this case to explicitly assign a value to \TT{v}.


\section{Memory Channels}
\label{MEMORY-CHANNELS}

OLD STUFF: REVISE.

A \key{memory channel} is a mechanism for accessing a set of blocks in RAM
that permits blocks to be announced substantially in advance of being
accessed.  Thus memory channels implement `\key{look ahead}' for
memory accesses.

A memory channel implements a \key{window}, which is a
structured set of elements each associated with a member of
some data set.  Each window element contains a
\key{block descriptor} that holds the address and length of the
memory block that contains the data associated with the element.
Block descriptors can also be marked as
\mkey{empty}{block descriptor}, meaning there is no block to be accessed.
The window has a \key{reference point}, and window elements are addressed
relative to this reference point.  There are shift operations that move
the reference point to a nearby window element.

Although we talk about blocks here, a block can be just a numeric array
element, and can be as small as a single bit.  Although we talk about
each element of a memory channel window having its own block descriptor,
an actual memory channel may use only block group descriptors, each of
which functions as a group of more than one individual element
block descriptor.

A memory channel is stored in a cluster.  As such it is mostly an
inline construction, though it can be passed to or returned from a
function, and the function can be all or partly out-of-line.

The most common type of memory channel has a window that appears to be
an array with \ttikey{.di\-men\-sions}{dimensions!of memory channel},
\ttdmkey{lower\_bound}{of memory channel}{\tt [}$i${\tt ]},
and \ttdmkey{upper\_bound}{of memory channel}{\tt [}$i${\tt ]}
being memory channel members.  Such are called
\key{array windows}.
If the memory channel cluster name is $M$, the window elements are
referred to by $M$\verb|[|$i_0$\verb|,|$i_1$\verb|,|\ldots\verb|]|,
with $M$\verb|[0,0,|\ldots\verb|]| being the \key{reference point}.

The reference point can be shifted along any of the window's
dimensions by the command
\begin{center}
$M$\ttdkey{center}{\tt [}$i_0${\tt ,}$i_1${\tt ,}\ldots{\tt ]}
\end{center}
This shifts
the window so that what was
$M${\tt [}$i_0${\tt ,}$i_1${\tt ,}\ldots{\tt ]} becomes
$M${\tt [0,0,}\ldots{\tt ]}.

Creating memory channels and completely reseting their reference points
are specific to the type of memory channel, and are not covered in
this section.

For most kinds of memory channels, block descriptors are computed
automatically when channel is created, when the window is
shifted, or when the data of
a neighboring window element is arrives from memory.
Immediately after a block descriptor is created, a read-ahead of
the block is initiated.  This read-ahead overlaps computation that
does not use the block contents.

If a memory channel accesses arrays stored in memory,
the channel block descriptors can be computed from the array coordinates
of the reference point.  Other memory channels use the contents
of a block to compute the block descriptors of neighboring blocks
in the window.

An example of the latter is a binary tree memory channel.
Let $M$ be such a channel, and let `{\tt .L}' denote the left
child of a binary tree element, `{\tt .R}' the right child, and
`{\tt .P}' the parent.  Then $M\!$\verb|.L.R| denotes the right
child of the left child of the reference point, $M\!$\verb|.P.L|
denotes the left child of the parent of the reference point,
and $M\!$\verb|.P.L.center| moves the reference point to this last element.
The window of such a memory channel might contain the depth 2
subtree of the reference point plus that closest 4 ancestors of the
reference point if these have been visited.  When the reference
point is moved, as soon as the reference point element has been
read from memory, the descriptors for its children are built and
the read of the children is initiated in parallel with other
computation.  When the children arrive from memory, the descriptors
of their children are built and reads of the data pointed at
are initiated.\footnote{All this can actually be done with modern
hardware: code is executed to read the reference point children and initiate the
reads of their children, and a modern processor will automatically
save the code that
executes when a read of a reference point child completes and execute
other code in parallel until the read does complete.}

Some standard memory channel types are built into the L-Language.
Others can be defined by users.

\section{To Do}

How can dynamically initialized locations be static.

Indirect address protocol.
\label{INDIRECT-ADDRESS-PROTOCOL}

Threads.
\label{THREADS}

Function calls should be able to serve as lvalues so
\verb|x[i]=z| can update gc flags.

\appendix

\section{Aliasing Hardware}
\label{ALIASING-HARDWARE}

The ultimate solution to the aliasing problem is new hardware.
At its simplest, registers, which currently hold a datum,
are replaced by triples of registers which hold a datum,
an address, and selection codes.  The register datum equals the value
of the memory location at the register address.  The selection
codes determine which part of this memory location is read or written
when the register is read or written.  If any memory location is
changed, the address of the location is checked against all the
register addresses, and if any match, the corresponding register
data are changed.

This is, however, not sufficient, because sometimes one register
address is a function of another register's datum.  For example,
consider the unchecked code:
\begin{indpar}[0.5em]\begin{verbatim}
struct S { ...; int32 m; ... }
S * * x
S * *& y = * x
int32 *& z = y->m
\end{verbatim}\end{indpar}
If we consider {\tt x}, {\tt y}, and {\tt z} to be registers,
the address of {\tt y} equals the value of {\tt x}, and the
address of {\tt z} equals the value of {\tt y} plus the offset of
{\tt m} in {\tt S}.

If the value of {\tt x} changes, this changes the address of {\tt y},
which may change the datum of {\tt y} and that may change the value
of {\tt y}.  If the value of {\tt y} changes, this changes the address
of {\tt z}, which may change the datum and value of {\tt z}.

The way we accommodate this is to use the selection codes of {\tt y}
to specify that the address of {\tt y} contains the value of {\tt x}
as an additive component, so
that if the value of {\tt x} is changed by adding $\Delta${\tt x}
then the address of {\tt y} should be changed by adding $\Delta${\tt x}.
And similarly the selection codes
of {\tt z} specify that the address of {\tt z}
contains the value of {\tt y} as an additive component.

So why should we bother with automatically updating
additive inclusions of one value in the
address of another value, and not bother with other expressions.
The reason is that expressions such as
`\verb|(*x)->m|' are likely to be reused frequently in code (actually,
in automatically generated code) and
therefore need to be cached, whereas an expressions of the form
`\verb|c*d|' will be reused comparatively rarely code
and therefore are not worth special hardware.


\bibliographystyle{plain}
\bibliography{layered-l-2012-language}

\printindex

\end{document}

