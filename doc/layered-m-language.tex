% The Layered Middle (M) Programming Language
%
% File:         layered-m-language.tex
% Author:       Bob Walton (walton@deas.harvard.edu)
% Version:      1a
  
\documentclass[12pt]{article}

\usepackage{makeidx}
\usepackage{pictex}

\makeindex

\setlength{\oddsidemargin}{0in}
\setlength{\evensidemargin}{0in}
\setlength{\textwidth}{6.5in}
\raggedbottom

\setlength{\unitlength}{1in}

\pagestyle{headings}
\setlength{\parindent}{0.0in}
\setlength{\parskip}{1ex}

\setcounter{secnumdepth}{5}
\setcounter{tocdepth}{5}
\newcommand{\subsubsubsection}[1]{\paragraph[#1]{#1.}}
\newcommand{\subsubsubsubsection}[1]{\subparagraph[#1]{#1.}}

% Begin \tableofcontents surgery.

\newcount\AtCatcode
\AtCatcode=\catcode`@
\catcode `@=11	% @ is now a letter

\renewcommand{\contentsname}{}
\renewcommand\l@section{\@dottedtocline{1}{0.1em}{1.4em}}
\renewcommand\l@table{\@dottedtocline{1}{0.1em}{1.4em}}
\renewcommand\tableofcontents{%
    \begin{list}{}%
	     {\setlength{\itemsep}{0in}%
	      \setlength{\topsep}{0in}%
	      \setlength{\parsep}{1ex}%
	      \setlength{\labelwidth}{0in}%
	      \setlength{\baselineskip}{1.5ex}%
	      \setlength{\leftmargin}{1.0in}%
	      \setlength{\rightmargin}{1.0in}}%
    \item\@starttoc{toc}%
    \end{list}}
\renewcommand\listoftables{%
    \begin{list}{}%
	     {\setlength{\itemsep}{0in}%
	      \setlength{\topsep}{0in}%
	      \setlength{\parsep}{1ex}%
	      \setlength{\labelwidth}{0in}%
	      \setlength{\baselineskip}{1.5ex}%
	      \setlength{\leftmargin}{1.0in}%
	      \setlength{\rightmargin}{1.0in}%
	      }%
    \item\@starttoc{lot}%
    \end{list}}

\catcode `@=\AtCatcode	% @ is now restored

% End \tableofcontents surgery.

\newcommand{\CN}[2]%	Change Notice.
    {\hspace*{0in}\marginpar{\sloppy \raggedright \it \footnotesize
     $^{\mbox{#1}}$#2}}
    % Change notice.

\newcommand{\key}[1]{{\bf \em #1}\index{#1}}
\newcommand{\mkey}[2]{{\bf \em #1}\index{#1!#2}}
\newcommand{\skey}[2]{{\bf \em #1#2}\index{#1}}
\newcommand{\ikey}[2]{{\bf \em #1}\index{#2}}
\newcommand{\ttkey}[1]{{\tt \bf #1}\index{#1@{\tt #1}}}
% < and > do not work for \tt \bf, hence:
\newcommand{\ttnbkey}[1]{{\tt #1}\index{#1@{\tt #1}}}
\newcommand{\ttmkey}[2]{{\tt \bf #1}\index{#1@{\tt #1}!#2}}
\newcommand{\ttmnbkey}[2]{{\tt #1}\index{#1@{\tt #1}!#2}}
\newcommand{\ttfkey}[2]{{\tt \bf #1}\index{#1@{\tt #1}!for #2@for {\tt #2}}}
\newcommand{\ttakey}[2]{{\tt \bf #1}\index{#2@{\tt #1}}}
\newcommand{\ttamkey}[3]{{\tt \bf #1}\index{#2@{\tt #1}!#3}}
\newcommand{\ttdkey}[1]{{\tt \bf .#1}\index{#1@{\tt .#1}}}
\newcommand{\ttdmkey}[2]{{\tt \bf .#1}\index{#1@{\tt .#1}!#2}}
\newcommand{\ttindex}[1]{\index{#1@{\tt #1}}}
\newcommand{\ttmindex}[2]{\index{#1@{\tt #1}!#2}}
\newcommand{\ttikey}[2]{{\tt \bf #1}\index{#1@{\tt #1}!#2}}
\newcommand{\emkey}[1]{{\bf \em #1}\index{#1@{\em #1}}}
\newcommand{\emindex}[1]{\index{#1@{\em #1}}}

\newcommand{\secref}[1]{\ref{#1}$^{p\pageref{#1}}$}
\newcommand{\stepref}[1]{\ref{#1}$^{p\pageref{#1}}$}
\newcommand{\appref}[1]{\ref{#1}$^{p\pageref{#1}}$}
\newcommand{\figref}[1]{\ref{#1}$^{p\pageref{#1}}$}
\newcommand{\pagref}[1]{p\pageref{#1}}

\newcommand{\EOL}{\penalty \exhyphenpenalty}

\newcount\TildeCatcode
\TildeCatcode=\catcode`\~
\catcode`~=12
\newcommand{\Tilde}{~}
\catcode`~=\TildeCatcode

\newcount\CircumflexCatcode
\CircumflexCatcode=\catcode`\^
\catcode`^=12
\newcommand{\Circumflex}{^}
\catcode`^=\CircumflexCatcode

\newcount\CurlyBraCatcode
\newcount\CurlyKetCatcode
\newcount\SquareBraCatcode
\newcount\SquareKetCatcode
\CurlyBraCatcode=\catcode`{
\CurlyKetCatcode=\catcode`}
\SquareBraCatcode=\catcode`[
\SquareKetCatcode=\catcode`]

\catcode`{=\SquareBraCatcode
\catcode`}=\SquareKetCatcode
\catcode`[=\CurlyBraCatcode
\catcode`]=\CurlyKetCatcode

\newcommand[\CurlyBra][{]
\newcommand[\CurlyKet][}]

\catcode`{=\CurlyBraCatcode
\catcode`}=\CurlyKetCatcode
\catcode`[=\SquareBraCatcode
\catcode`]=\SquareKetCatcode

\newcommand{\ttbrackets}{%
    \renewcommand{\{}{\CurlyBra}%
    \renewcommand{\}}{\CurlyKet}}

\newsavebox{\TILDEBOX}
\begin{lrbox}{\TILDEBOX}
\verb|~|
\end{lrbox}
\newcommand{\TILDE}{\usebox{\TILDEBOX}}

\newsavebox{\BACKSLASHBOX}
\begin{lrbox}{\BACKSLASHBOX}
\verb|\|
\end{lrbox}
\newcommand{\BACKSLASH}{\usebox{\BACKSLASHBOX}}

\newsavebox{\LEFTBRACKETBOX}
\begin{lrbox}{\LEFTBRACKETBOX}
\verb|{|
\end{lrbox}
\newcommand{\LEFTBRACKET}{\usebox{\LEFTBRACKETBOX}}

\newsavebox{\RIGHTBRACKETBOX}
\begin{lrbox}{\RIGHTBRACKETBOX}
\verb|}|
\end{lrbox}
\newcommand{\RIGHTBRACKET}{\usebox{\RIGHTBRACKETBOX}}

\newsavebox{\UNDERLINEBOX}
\begin{lrbox}{\UNDERLINEBOX}
\verb|_|
\end{lrbox}
\newcommand{\UNDERLINE}{\usebox{\UNDERLINEBOX}}

\newsavebox{\CIRCUMFLEXBOX}
\begin{lrbox}{\CIRCUMFLEXBOX}
\verb|^|
\end{lrbox}
\newcommand{\CIRCUMFLEX}{\usebox{\CIRCUMFLEXBOX}}

\newsavebox{\BARBOX}
\begin{lrbox}{\BARBOX}
\verb/|/
\end{lrbox}
\newcommand{\BAR}{\usebox{\BARBOX}}

\newsavebox{\LESSTHANBOX}
\begin{lrbox}{\LESSTHANBOX}
\verb/</
\end{lrbox}
\newcommand{\LESSTHAN}{\usebox{\LESSTHANBOX}}

\newsavebox{\GREATERTHANBOX}
\begin{lrbox}{\GREATERTHANBOX}
\verb/>/
\end{lrbox}
\newcommand{\GREATERTHAN}{\usebox{\GREATERTHANBOX}}

\newlength{\figurewidth}
\setlength{\figurewidth}{\textwidth}
\addtolength{\figurewidth}{-0.40in}

\newsavebox{\figurebox}

\newenvironment{boxedfigure}[1][!btp]%
	{\begin{figure*}[#1]
	 \begin{lrbox}{\figurebox}
	 \begin{minipage}{\figurewidth}

	 \vspace*{1ex}}%
	{
	 \vspace*{1ex}

	 \end{minipage}
	 \end{lrbox}
	 \begin{center}
	 \fbox{\hspace*{0.1in}\usebox{\figurebox}\hspace*{0.1in}}
	 \end{center}
	 \end{figure*}}

\newenvironment{indpar}[1][0.3in]%
	{\begin{list}{}%
		     {\setlength{\itemsep}{0in}%
		      \setlength{\topsep}{0in}%
		      \setlength{\parsep}{1ex}%
		      \setlength{\labelwidth}{#1}%
		      \setlength{\leftmargin}{#1}%
		      \addtolength{\leftmargin}{\labelsep}}%
	 \item}%
	{\end{list}}

\begin{document}
        
\begin{center}

{\Large
The Layered Middle (M) Programming Language \\[0.5ex]
(Draft 1a)}

\medskip

Robert L. Walton\footnote{This document is dedicated to the memory
of Professor Thomas Cheatham of Harvard University.}

February 14, 2007
 
\end{center}

{\small
\tableofcontents 
}

\newpage

\section{Introduction}

This document describes the Middle Layer Programming Language, or
M-Language.  See the Introduction to the Layered
Programming Languages for an a description of the syntax of
the M-Language and an overview of the related
Lower Layer L-Language and Higher Layer H-Language.


\section{Memory}
\label{MEMORY}

We begin with an overview of M-language memory, and then provide
details in the following subsections.

\subsection{Blocks}
\label{BLOCKS}

An M-language memory \key{block} is a sequence of bits.
Blocks can be contained in other blocks or can overlap.

A block contains a subblock called the \ikey{base}{of block}
of the block.  For many blocks the base and the block are the same.
For blocks that can grow, the base is the part of the block that
is always there, and which contains information telling how much
the block has grown.

When a block is stored in memory, it has a \ikey{bit address}{of block}.
The bit address plus some constant offset is the bit address of the first
bit of the base of the block.

Blocks have the following attributes:

\begin{indpar}\begin{tabular}{p{1.0in}p{4.0in}}
\ttikey{size}{of block}
    & Number of bits in the base of the block.
\end{tabular}\end{indpar}
\begin{indpar}\begin{tabular}{p{1.0in}p{4.0in}}
\ttikey{growth}{of block}
    & \ttikey{up}{block growth} and/or \ttikey{down}{block growth} or neither;
      Both {\tt up} and {\tt down} mean the block may be larger than its base.
      {\tt up} means there may be extra bits after the
      base, and {\tt down} means there may be extra bits before the base.
      If the {\tt growth} of a block is neither {\tt up} nor {\tt down},
      the block is the same as its base.
\end{tabular}\end{indpar}
\begin{indpar}\begin{tabular}{p{1.0in}p{4.0in}}
\ttikey{offset}{of block}
    & Offset in bits of the bit address of the first bit of the block
      base from the bit address of the block, when the block is stored
      in RAM memory.
\end{tabular}\end{indpar}
\begin{indpar}\begin{tabular}{p{1.0in}p{4.0in}}
\ttikey{alignment}{of block}
    & Exact divisor of bit address of the first bit of the base of
      the block, when the block is stored in RAM memory.
\end{tabular}\end{indpar}
\begin{indpar}\begin{tabular}{p{1.0in}p{4.0in}}
\ttikey{mobility}{of block}
    & \ttikey{fixed}{block mobility},
      \ttikey{movable}{block mobility},
      or \ttikey{free}{block mobility}.
      A {\tt fixed} block is stored in RAM memory at a fixed address.
      A {\tt movable} block is stored in RAM memory at an address that
      may be either fixed or may be changed according to the
      rules for managing heavy weight addresses below (\secref{ADDRESSES}).
      A {\tt free} block may be stored in register or RAM memory
      and moved at any time.
\end{tabular}\end{indpar}
\begin{indpar}\begin{tabular}{p{1.0in}p{4.0in}}
\ttikey{operability}{of block}
    & \ttikey{integer}{block operability},
      \ttikey{float}{block operability},
      \ttikey{address}{block operability}, or
      \ttikey{container}{block operability}.
      The {\tt operability} of a block determines the kind of
      register the block is stored in, if the block is moved to
      a register.
\end{tabular}\end{indpar}
\begin{indpar}\begin{tabular}{p{1.0in}p{4.0in}}
\ttikey{status}{of block}
    & \ttikey{constant}{block status} or
      \ttikey{variable}{block status}.  A {\tt constant}
      block cannot be change, while a {\tt variable} block can be.
\end{tabular}\end{indpar}

This is just enough information to allow the M-language implementation
to allocate blocks to memory and move blocks in memory.

\subsection{Addresses}
\label{ADDRESSES}

Both blocks and addresses are classified as `light weight'
or `heavy weight'.

A \key{heavy weight block} is a movable block that
has a reference pointer stored at a fixed location which
points at the block and which changes when the block moves.

A \key{heavy weight address} has
two parts: the address of a reference pointer, and an offset within
the heavy weight block that is pointed at by the reference pointer.

A \key{light weight block} is a free block,
a fixed block allocated to a fixed address,
or a subblock of another block.  It can be a subblock of a
heavy weight block.
A heavy weight address is used to address a light weight subblock
of a heavy weight block.

A \key{light weight address} is just the address of a fixed, non-movable
(light weight) block.

The entire heavy weight block can be
addressed by a light weight address pointing at the reference pointer
of the block, but an addressing indirection is required to address
the block.  Such an address, which is just a heavy weight address
without any offset part, is called a \key{middle weight address}.
A heavy weight block that is never addressed by a heavy
weight address, but only by middle weight addresses, is called a
\key{middle weight block}.

Reference pointers can all be elements of a vector or other data
structure, in which case an index can be used in place of the address of 
a reference pointer.

Alternatively reference pointers can be allocated as the first thing in
their movable heavyweight block, in which case they must be addressed
and not identified by index.  In this case, the entire heavy weight
block must be allocated as a subblock of a fixed block, so the location
of the reference pointer will be fixed.  When the heavy weight block
is moved, the reference pointer in its old location is changed to point
to the new location.  At this point the block has two reference pointers,
one in its old location, and one at the beginning of its new location,
but both have the same address.  Standard garbage collection techniques
can be used to remove all references to the old reference pointer to
the old block can be deallocated.  In greater generality, any reference
pointer may be moved by allocating a duplicate and then eliminating
all references to the original reference pointer.

A heavy weight block can be moved almost anytime by
changing its reference pointer.  The block can be deallocated
almost anytime by moving it to unimplemented virtual memory.

A heavy weight block can be a stack that grows up or down, and
subblocks can be allocated to or deallocated from the growing
boundary of a stack.  Fixed blocks can also be stacks; but they
must have memory above or below them reserved for allocation
of subblocks.

\subsection{Block Groups}
\label{BLOCK-GROUPS}

A \key{block group} is a set of blocks in RAM memroy which point at each other,
with one particular block in the set being designated as the
\ikey{root}{of block group} of the block group.  All the other blocks
must in the group must be reachable by following pointers from the root.
The pointers may be addresses or may be indeces into some data structure.

A block group can be copied in RAM memory by copying each of the group's
blocks and adjusting the pointers in the copies to point at the copies
of their original target blocks.

The \ikey{address}{of block group} is the address of its root.

Block groups can point at other block groups; that is, can contain pointers
to the roots of other block groups.  When a block group is copied, these
pointers can be adjusted to point copies of their original targets, if these
copies exist.  Otherwise, if there is no copy of the target block group of
a pointer, then a \key{stub} for the target may be allocated and the
copied pointer adjusted to point at this target stub.  The stub must be
such that if the original target block group is ever copied, it will be
located in memory where the stub is located, and will replace the stub.
This is made easier if the stub is a heavy weight block, as its replacment
may be much bigger than it is.  In general, the stub is a block group
and may have more than one block.

\subsection{Values}
\label{VALUES}

A value is a sequence of blocks.  The last block is
the proper value.  The initial subsequence of all blocks except the last
is the `type' of the value, and is itself a value.
The type descibes the format of the value proper, and is used
to select function code when the value is given as an argument
to a function.  Some example types are 
`32-bit signed integer' and `64-bit floating point number'.
The type must specify the attributes of the value proper; e.g.,
the block size of the value proper.

As a special case, an empty sequence is taken as equal to
a particular value that is a type which specifies that the
value proper is a type and is constant.

A type defines a view of a value proper.  A value proper can
have several different views, i.e., can be the value proper
of several different values with different types.

Although a type is a value, it is constrained in many ways.
First, it must exist both at compile time and at run time.
Therefore it must be possible to copy a type to an external medium
and read it back into memory.
Second, types must be suitable for type-inference, a process in
which parts of the type are discovered and filled in, so
types must be able to exhibit partial information as well
as complete information.  Type inference is a search algorithm
that adds information to types, and may backup up the state
of a type to some previous state.  Asside from that actions
of type inference, types may not change their value.

A variable is a place in memory that holds a value.  A constant is just
a variable whose value may not be changed.

Normally at run time a variable is a sequence of just two blocks,
the first being a constant type, and the second the value proper,
which may or may not be constant, as is indicated by the type.
Normally during type inference at compile time a variable is a sequence of three
blocks, the first begin a constant type that indicates that the
second is a type that is not constant and can be changed, and the
third block being the value proper.  If the type (second block)
indicates, the third block can be constant.  If the type (second block)
does not have enough information, the value proper may not exist
(e.g., if the type does not have enough information to determine the
block size of the value proper).

However, there is no prohibition
against values having types that are not constant at run time, though
the available operations on such values are few and slow.  Such
values are well suited to compute inputs to delayed compilation
(see~\secref{EXPRESSIONS}).

A component of a value V is a value C that can be computed from V
by a designated `read function'.  The read function may take
agruments in addition to V, and these are called indices.
The components of a variable are just a components of the variable's value.
A component of a variable can itself be a variable if there is a
designated `write function' that changes the value returned for the
component.  Read and write functions for a component must obey
some rules: for example, two successive reads must return the same
value, and a read immediately following a write must return the value
written, when the value and index arguments given to successive functions
calls are the same.  In addition to read and write functions a component
may have update, accumulate, and test functions.

The different components of a variable obey different protocols.
The {\tt rw} protocol says that a component is readable and writable
at any time.  {\tt w/r} says that all writes occur before all reads.
{\tt w/a/r} says that writes occur first, then accumulates, then reads,
where accumulates are operations that may be exchanged, such as incrementing
the variable component by various amounts.

Types and type inference are not described in detail until
\secref{EXPRESSIONS}.
Components and protocols are not describe in detail until
\secref{PROTOCOLS}.  The other concepts just introduced are
defined in detail in the following subsections.

\section{Expressions}
\label{EXPRESSIONS}

The task of an expression is to compute values from other values.
This is straightforward except for issues in parsing, overloading,
implicit variable creation, implicit typing, implicit conversion,
and protection.

Parsing involves inserting implied parentheses and then rewriting
expressions, particularly those involving operators.  Overloading
involves picking which function to call given many functions with the
same name.  Implicit variable creation involves creating variables to
hold values output by one function and input by another.\footnote{
What we call `implicit variables' are usually called `temporary variables'.
We make the implicit variable creation process explicit for two reasons:
first, we have a more complex situation in which one function may
output many results, as is explained a bit further
in the text, and second, M-language debuggers have a mode of
operation in which instead of displaying location within the program
code, which can be a problematic concept given all the rewriting that
is done, they display the status of ordinary and implicit variables, making
it important to explain implicit variables carefully to novice programmers.}
Implicit typing involves assigning types to variables, such as implicit
variables, that were given no type or only a partial type in the code.
Implicit conversion involves inserting conversion operators to change the
type of a value.
Protection involves ensuring that code in a restricted protection domain
cannot be called or referenced from another protection domain without
explicit permission.

Parsing begins by identifying operators in an expression.  By using
a precedence level assigned to each operator/fixity pair (operators
can have three fixities: prefix, infix, or postfix), implied parentheses
are inserted.  The M-language differs from other computer languages
in not using associativity rules in doing this, so that, for
instance, `\verb|(0 < x <= y)|' has \underline{no} implied parentheses
inserted, and can be rewritten later to be the equivalent of
`\verb|(0 < x AND x <= y)|'.  After inserting implied parentheses,
expressions are rewritten using macros.  In the example just given,
`\verb|<|' and `\verb|<=|' have the same precedence level and invoke
the same rewrite macro.

Program code begins as text, a sequence of characters.  It is first
scanned to become a sequence of lexemes (words and numbers and
punctuation marks), and then parsed.  The lexemes are tagged with
their location within the text, and with the subexpression they
belong to after insertion of implied parentheses.  These tags
are carried through the rewriting of expressions, and may be copied
to new lexemes introduced during rewriting, so that eventually
a debugger can associate operations executed when the program runs with
particular lexemes and subexpressions in the program code text.

Parsing and rewriting use data constructs and operations provided
by the H-Language, the higher level layered language.  This language
provides, for example, for strings used to encode lexemes and
lists used to encode expressions.  Macros are user written H-language code
executed by the M-language compiler.

Overloading, implicit variable creation, implicit typing, and implicit
conversion are all
done together.  Each function definition has a component function called
the typer which takes as input a partial description of the types
and movement modes of the arguments of a call to the function
and any available values of compile time constant arguments
input to the function,
and produces as output possible completions of this information which
would allow the function definition to be used.  The movement modes
of an argument are `in' for an input argument, `out' for an output
argument, and `inout' for an argument that is both input and output.
The result of a function is treated here as an argument.
The typer succeeds or fails when it is given enough information to
ensure that the function definition can or cannot be used, and if it
is not given enough information, the typer proposes additions to the
information it is given, if it can.  Note that if all inputs to a function
are compile time constants, it may happen that the typer can produce
compile time constants for all outputs, and thus eliminate the need to execute
the function at run time.

Specification of argument movement modes determines what implicit
variables are needed.  Specification of argument types determines
implicit types of both implicit variables and of explicit variables
that are declared without types (or with incomplete types).

In addition to typers for explicitly named functions,
typers for implicitly named conversion functions are executed.
Conversion functions convert values of one type to values of a
different type: e.g., convert 32-bit integers to 64-bit floating
point numbers.  Implicit conversions are inserted into the value
flow wherever necessary and possible so that all the needed
types can be computed.

All this is a search that specifies implicit type information,
movement modes, implicit conversions, and compile time constants.
This search may have
multiple successful answers.  If it does the result is deemed
ambiguous and a compile error.  The user who provides multiple
definitions for a given function may elect to given them a priority
order, so that lower priority definitions will not be examined by
the search if a higher priority definition can be used, and ambiguity
may be avoided.  Similarly implied conversions have priorities, with not using
an implied conversion being given higher priority
than using any implicit conversion.

If the search fails to find any answer within a reasonable time, compilation
fails.  If this happens, the coder can explicitly specify as many more types,
variables, and conversions as is necessary to get the search to complete within
a reasonable time.  Here time is measured by counting certain operations
performed by the search, so the measure of reasonable time will be hardware
independent.

Typer functions can be implicitly declared using M-language function
declarations, or can be explicitly coded by the user.  The later option
is of course more flexible.

After all types are computed and all overloaded function definitions
have been selected, the function definitions generate function code.
The generated code may just be a fixed block of M-language code,
or it may be generated by calling a generation component function of the
function definition.  Such a generation function is written in the H-language
and returns the code for the function, given the types and movement modes
of its arguments, and any compile time constant argument values.

The M-language also supports delayed compilation of both code blocks and
functions.   In either case some set of arguments or variables or types
is specified to be the set of parameters input to a code generator
that generates the block or function code.
The generated code may be a constant
block of M-Language code that has implied types or uses parameters as
compile time constants, or the generated code may be
M-Language code output by a user supplied H-Language code generator function.
Every time the function or block executes the current values of its
parameters may be input to the code generator to generate the code to be
executed.  The result is cached according to the
parameter values, so that when the function or block is invoked again,
if the parameters are the same the compiled code may be taken from the
cache and not recompiled.

\section{Protocols}
\label{PROTOCOLS}


\bibliographystyle{plain}
\bibliography{layered-m-language}

\printindex

\end{document}

