% The Layered Middle (M) Programming Language
%
% File:         layered-m-language.tex
% Author:       Bob Walton (walton@deas.harvard.edu)
% Version:      1a
  
\documentclass[12pt]{article}

\usepackage{makeidx}
\usepackage{pictex}

\makeindex

\setlength{\oddsidemargin}{0in}
\setlength{\evensidemargin}{0in}
\setlength{\textwidth}{6.5in}
\raggedbottom

\setlength{\unitlength}{1in}

\pagestyle{headings}
\setlength{\parindent}{0.0in}
\setlength{\parskip}{1ex}

\setcounter{secnumdepth}{5}
\setcounter{tocdepth}{5}
\newcommand{\subsubsubsection}[1]{\paragraph[#1]{#1.}}
\newcommand{\subsubsubsubsection}[1]{\subparagraph[#1]{#1.}}

% Begin \tableofcontents surgery.

\newcount\AtCatcode
\AtCatcode=\catcode`@
\catcode `@=11	% @ is now a letter

\renewcommand{\contentsname}{}
\renewcommand\l@section{\@dottedtocline{1}{0.1em}{1.4em}}
\renewcommand\l@table{\@dottedtocline{1}{0.1em}{1.4em}}
\renewcommand\tableofcontents{%
    \begin{list}{}%
	     {\setlength{\itemsep}{0in}%
	      \setlength{\topsep}{0in}%
	      \setlength{\parsep}{1ex}%
	      \setlength{\labelwidth}{0in}%
	      \setlength{\baselineskip}{1.5ex}%
	      \setlength{\leftmargin}{1.0in}%
	      \setlength{\rightmargin}{1.0in}}%
    \item\@starttoc{toc}%
    \end{list}}
\renewcommand\listoftables{%
    \begin{list}{}%
	     {\setlength{\itemsep}{0in}%
	      \setlength{\topsep}{0in}%
	      \setlength{\parsep}{1ex}%
	      \setlength{\labelwidth}{0in}%
	      \setlength{\baselineskip}{1.5ex}%
	      \setlength{\leftmargin}{1.0in}%
	      \setlength{\rightmargin}{1.0in}%
	      }%
    \item\@starttoc{lot}%
    \end{list}}

\catcode `@=\AtCatcode	% @ is now restored

% End \tableofcontents surgery.

\newcommand{\CN}[2]%	Change Notice.
    {\hspace*{0in}\marginpar{\sloppy \raggedright \it \footnotesize
     $^{\mbox{#1}}$#2}}
    % Change notice.

\newcommand{\key}[1]{{\bf \em #1}\index{#1}}
\newcommand{\mkey}[2]{{\bf \em #1}\index{#1!#2}}
\newcommand{\skey}[2]{{\bf \em #1#2}\index{#1}}
\newcommand{\ikey}[2]{{\bf \em #1}\index{#2}}
\newcommand{\ttkey}[1]{{\tt \bf #1}\index{#1@{\tt #1}}}
% < and > do not work for \tt \bf, hence:
\newcommand{\ttnbkey}[1]{{\tt #1}\index{#1@{\tt #1}}}
\newcommand{\ttmkey}[2]{{\tt \bf #1}\index{#1@{\tt #1}!#2}}
\newcommand{\ttmnbkey}[2]{{\tt #1}\index{#1@{\tt #1}!#2}}
\newcommand{\ttfkey}[2]{{\tt \bf #1}\index{#1@{\tt #1}!for #2@for {\tt #2}}}
\newcommand{\ttakey}[2]{{\tt \bf #1}\index{#2@{\tt #1}}}
\newcommand{\ttamkey}[3]{{\tt \bf #1}\index{#2@{\tt #1}!#3}}
\newcommand{\ttdkey}[1]{{\tt \bf .#1}\index{#1@{\tt .#1}}}
\newcommand{\ttdmkey}[2]{{\tt \bf .#1}\index{#1@{\tt .#1}!#2}}
\newcommand{\ttindex}[1]{\index{#1@{\tt #1}}}
\newcommand{\ttmindex}[2]{\index{#1@{\tt #1}!#2}}
\newcommand{\emkey}[1]{{\bf \em #1}\index{#1@{\em #1}}}
\newcommand{\emindex}[1]{\index{#1@{\em #1}}}

\newcommand{\secref}[1]{\ref{#1}$^{p\pageref{#1}}$}
\newcommand{\stepref}[1]{\ref{#1}$^{p\pageref{#1}}$}
\newcommand{\appref}[1]{\ref{#1}$^{p\pageref{#1}}$}
\newcommand{\figref}[1]{\ref{#1}$^{p\pageref{#1}}$}
\newcommand{\pagref}[1]{p\pageref{#1}}

\newcommand{\EOL}{\penalty \exhyphenpenalty}

\newcount\TildeCatcode
\TildeCatcode=\catcode`\~
\catcode`~=12
\newcommand{\Tilde}{~}
\catcode`~=\TildeCatcode

\newcount\CircumflexCatcode
\CircumflexCatcode=\catcode`\^
\catcode`^=12
\newcommand{\Circumflex}{^}
\catcode`^=\CircumflexCatcode

\newcount\CurlyBraCatcode
\newcount\CurlyKetCatcode
\newcount\SquareBraCatcode
\newcount\SquareKetCatcode
\CurlyBraCatcode=\catcode`{
\CurlyKetCatcode=\catcode`}
\SquareBraCatcode=\catcode`[
\SquareKetCatcode=\catcode`]

\catcode`{=\SquareBraCatcode
\catcode`}=\SquareKetCatcode
\catcode`[=\CurlyBraCatcode
\catcode`]=\CurlyKetCatcode

\newcommand[\CurlyBra][{]
\newcommand[\CurlyKet][}]

\catcode`{=\CurlyBraCatcode
\catcode`}=\CurlyKetCatcode
\catcode`[=\SquareBraCatcode
\catcode`]=\SquareKetCatcode

\newcommand{\ttbrackets}{%
    \renewcommand{\{}{\CurlyBra}%
    \renewcommand{\}}{\CurlyKet}}

\newsavebox{\TILDEBOX}
\begin{lrbox}{\TILDEBOX}
\verb|~|
\end{lrbox}
\newcommand{\TILDE}{\usebox{\TILDEBOX}}

\newsavebox{\BACKSLASHBOX}
\begin{lrbox}{\BACKSLASHBOX}
\verb|\|
\end{lrbox}
\newcommand{\BACKSLASH}{\usebox{\BACKSLASHBOX}}

\newsavebox{\LEFTBRACKETBOX}
\begin{lrbox}{\LEFTBRACKETBOX}
\verb|{|
\end{lrbox}
\newcommand{\LEFTBRACKET}{\usebox{\LEFTBRACKETBOX}}

\newsavebox{\RIGHTBRACKETBOX}
\begin{lrbox}{\RIGHTBRACKETBOX}
\verb|}|
\end{lrbox}
\newcommand{\RIGHTBRACKET}{\usebox{\RIGHTBRACKETBOX}}

\newsavebox{\UNDERLINEBOX}
\begin{lrbox}{\UNDERLINEBOX}
\verb|_|
\end{lrbox}
\newcommand{\UNDERLINE}{\usebox{\UNDERLINEBOX}}

\newsavebox{\CIRCUMFLEXBOX}
\begin{lrbox}{\CIRCUMFLEXBOX}
\verb|^|
\end{lrbox}
\newcommand{\CIRCUMFLEX}{\usebox{\CIRCUMFLEXBOX}}

\newsavebox{\BARBOX}
\begin{lrbox}{\BARBOX}
\verb/|/
\end{lrbox}
\newcommand{\BAR}{\usebox{\BARBOX}}

\newsavebox{\LESSTHANBOX}
\begin{lrbox}{\LESSTHANBOX}
\verb/</
\end{lrbox}
\newcommand{\LESSTHAN}{\usebox{\LESSTHANBOX}}

\newsavebox{\GREATERTHANBOX}
\begin{lrbox}{\GREATERTHANBOX}
\verb/>/
\end{lrbox}
\newcommand{\GREATERTHAN}{\usebox{\GREATERTHANBOX}}

\newlength{\figurewidth}
\setlength{\figurewidth}{\textwidth}
\addtolength{\figurewidth}{-0.40in}

\newsavebox{\figurebox}

\newenvironment{boxedfigure}[1][!btp]%
	{\begin{figure*}[#1]
	 \begin{lrbox}{\figurebox}
	 \begin{minipage}{\figurewidth}

	 \vspace*{1ex}}%
	{
	 \vspace*{1ex}

	 \end{minipage}
	 \end{lrbox}
	 \begin{center}
	 \fbox{\hspace*{0.1in}\usebox{\figurebox}\hspace*{0.1in}}
	 \end{center}
	 \end{figure*}}

\newenvironment{indpar}[1][0.3in]%
	{\begin{list}{}%
		     {\setlength{\itemsep}{0in}%
		      \setlength{\topsep}{0in}%
		      \setlength{\parsep}{1ex}%
		      \setlength{\labelwidth}{#1}%
		      \setlength{\leftmargin}{#1}%
		      \addtolength{\leftmargin}{\labelsep}}%
	 \item}%
	{\end{list}}

\begin{document}
        
\begin{center}

{\Large
The Layered Middle (M) Programming Language \\[0.5ex]
(Draft 1a)}

\medskip

Robert L. Walton\footnote{This document is dedicated to the memory
of Professor Thomas Cheatham of Harvard University.}

December 6, 2006
 
\end{center}

{\small
\tableofcontents 
}

\newpage

\section{Introduction}

This document describes the Middle Layer Programming Language, or
M-Language.  See the Introduction to the Layered
Programming Languages for an overview of the related
Lower Layer L-Language and Higher Layer H-Language.


\section{Memory}
\label{MEMORY}

We begin with an overview of M-language memory, and then provide
details in the following subsections.

M-language memory consists of blocks which are sequences of bits.
Blocks can be contained in other blocks or can overlap.  Blocks
have the following attributes:

\begin{center}
\begin{tabular}{ll}
{\tt size}		& number of bits \\
{\tt growth}		& {\tt up}, {\tt down}, {\tt none} \\
{\tt alignment}		& exact divisor of bit address \\
{\tt offset}		& offset in bits of start of block from
			  bit address of block \\
{\tt mobility}		& {\tt free}, {\tt fixed}, {\tt movable} \\
{\tt operability}	& {\tt integer}, {\tt float}, {\tt address},
			  {\tt container} \\
{\tt status}		& {\tt constant}, {\tt variable} \\
\end{tabular}
\end{center}

This is just enough information to allow the M-language implementation
to allocate blocks to memory and move blocks in memory.  A {\tt growth}
of `{\tt up}' means the {\tt size} is the minimum size and the actual size
is larger with the block extending above its address.  Similarly for
`{\tt down}' with the block extending below its address.  The {\tt alignment}
and {\tt offset} only affects blocks when they are stored in memory.
{\tt Free} blocks have no fixed address and can be moved around in
memory; {\tt fixed} blocks have a fixed unchanging address; and
{\tt movable} blocks have an address that can be changed in a restricted
way (see below).  {\tt Operability} determines the kind of register
a block is copied into, when a block is placed in a register.  {\tt Status}
allows a block to be marked as {\tt constant} permitting it to be in
several places at once.

Both blocks and addresses are classified as `light weight'
or `heavy weight'.

A heavy weight block is a movable block that
has a reference pointer stored at a fixed location which
points at the block and which changes when the block moves.
A heavy weight address has
two parts: the address of a reference pointer, and an offset within
the heavy weight block that is pointed at by the reference pointer.

A lightweight block is a free block,
a fixed block allocated to a fixed address,
or a subblock of another block.  It can be a subblock of a
heavy weight block.
A heavy weight address is used to address a light weight subblock
of a heavy weight block.
A light weight address is just the address of a fixed, non-movable
(light weight) block.

The entire heavy weight block can be
addressed by a light weight address pointing at the reference pointer
of the block, but an addresssing indirection is required to address
the block.  Such an address, which is just a heavy weight address
without any offset part, is called a `middle weight' address.
A heavy weight block that is never addressed by a heavy
weight address, but only by middle weight addresses, is called a
`middle weight block'.

Reference pointers can all be elements of a vector, in which case
a vector index can be used in place of the address of 
a reference pointer.  Alternatively reference pointers can be allocated
as the first thing in their movable heavyweight block, in which
case they must be addressed and not identified by index.  In this last
case, the entire heavy weight block must be allocated as a
subblock of a fixed block, so the location of the
reference pointer will be fixed.
However, a reference pointer can be garbage collected by keeping
two copies, the old and the new, with identical contents, and
then replacing all addresses or indices of the old pointer by
the address or index of the new pointer.

A heavy weight block can be moved almost anytime by
changing its reference pointer.  The block can be deallocated
almost anytime by moving it to unimplemented virtual memory.

A heavy weight block can be a stack that grows up or down, and
subblocks can be allocated to or deallocated from the growing
boundary of a stack.  Other blocks can also be stacks; but they
must have memory above or below them reserved for allocation
of subblocks.

A value is a sequence of constant free blocks.  The initial subsequence
of all blocks up until the last is called the `type' of the value,
and the blocks in this type are called `descriptors'.  The first descriptor
contains an encoding that specifies
the `major type' of the value.  Some major types are
32-bit signed integers, 64-bit floating point numbers, structures with
a given list of components, etc.  If a second descriptor exists it
generally encodes subtype information, such as bounds for an integer
or subscript bounds for an array.

The value that is an empty sequence is a special case that is equivalent
to a major type descriptor that specifies that a value is a major type
descriptor.

A variable is a place that holds a value that may be changed.
However, when changing a variable's value, only its last block may be changed:
the type may not be.  So a variable has a fixed type,
and is a place to store the last block of a value of that type.
A variable may be fixed, movable, or
free according to whether the variable has a fixed, movable, or free
block in which it stores the last block of its current value.
Regardless of whether a variable is fixed, movable, or free, the
descriptors of the variable's type are constant free blocks.

The compiler may or may not have to know something about the type
of a variable in order to compile code that uses the variable.
Type information is used in overloading at compile time, but as
long as the information is sufficient to resolve overloading and
implied conversions, compilation will be successful.
See~\secref{EXPRESSIONS}.

A component of a value V is a value C that can be computed from V
by a function.  The components
of a variable are just a components of the variable's value.
Components of variables can themselves be variable if there is a
function that changes the value that will be returned for the component.

The different components of a variable obey different protocols.
The {\tt rw} protocol says that a component is readable and writable
at any time.  {\tt w/r} says that all writes occur before all reads.
{\tt w/a/r} says that writes occur first, then accumulates, then reads,
where accumulates are operations that may be exchanged, such as incrementing
the variable component by various amounts.

Components and protocols are not describe in detail until
\secref{EXPRESSIONS}.  The other concepts just introduced are
defined in detail in the following subsections.

\section{Expressions}
\label{EXPRESSIONS}

The task of an expression is to compute values from other values.
This is straightforward except for issues in parsing, overloading,
implicit variable creation, implicit typing, implicit conversion,
and protection.

Parsing involves inserting implied parentheses and then rewriting
expressions, particularly those involving operators.  Overloading
involves picking which function to call given many functions with the
same name.  Implicit variable creation involves creating variables to
hold values output by one function and input by another.\footnote{
What we call `implicit variables' are usually called `temporary variables'.
We make the implicit variable creation process explicit for two reasons:
first, we have a more complex situation in which one function may
output many results, as is explained a bit further
in the text, and second, M-language debuggers have a mode of
operation in which instead of displaying location within the program
code, which can be a problematic concept given all the rewriting that
is done, they display the status of ordinary and implicit variables, making
it important to explain implicit variables carefully to novice programmers.}
Implicit typing involves assigning types to variables, such as implicit
variables, that were given no type or only a partial type in the code.
Implicit conversion involves inserting conversion operators to change the
type of a value.
Protection involves ensuring that code in a restricted protection domain
cannot be called or referenced from another protection domain without
explicit permission.

Parsing begins by identifying operators in an expression.  By using
a precedence level assigned to each operator/fixity pair (operators
can have three fixities: prefix, infix, or postfix), implied parentheses
are inserted.  The M-language differs from other computer languages
in not using associativity rules in doing this, so that, for
instance, `\verb|(0 < x <= y)|' has \underline{no} implied parentheses
inserted, and can be rewritten later to be the equivalent of
`\verb|(0 < x AND x <= y)|'.  After inserting implied parentheses,
expressions are rewritten using macros.  In the example just given,
`\verb|<|' and `\verb|<=|' have the same precedence level and invoke
the same rewrite macro.

Program code begins as text, a sequence of characters.  It is first
scanned to become a sequence of lexemes (words and numbers and
punctuation marks), and then parsed.  The lexemes are tagged with
their location within the text, and with the subexpression they
belong to after insertion of implied parentheses.  These tags
are carried through the rewriting of expressions, and may be copied
to new lexemes introduced during rewriting, so that eventually
a debugger can associate operations executed when the program runs with
particular lexemes and subexpressions in the program code text.

Parsing and rewriting use data constructs and operations provided
by the H-Language, the higher level layered language.  This language
provides, for example, for strings used to encode lexemes and
lists used to encode expressions.  Macros are user written H-language code
executed by the M-language compiler.

Overloading, implicit variable creation, implicit typing, and implicit
conversion are all
done together.  Each function definition has a component function called
the typer which takes as input a partial description of the types
and movement modes of the arguments of a call to the function,
and when available and useful the values of constant arguments input
to the function,
and produces as output possible completions of this information which
would allow the function definition to be used.  The movement modes
of an argument are `in' for an input argument, `out' for an output
argument, and `inout' for an argument that is both input and output.
The result of a function is treated here as an argument.
The typer succeeds or fails when it is given enough information to
ensure that the function definition can or cannot be used, and if it
is not given enough information, the typer proposes additions to the
information it is given, if it can.

Specification of argument movement modes determines what implicit
variables are needed.  Specification of argument types determines
implicit types of both implicit variables and of explicit variables
that are declared without types (or with incomplete types).

In addition to typers for explicitly named functions,
typers for implicitly named conversion functions are executed.
Conversion functions convert values of one type to values of a
different type: e.g., convert 32-bit integers to 64-bit floating
point numbers.  Implicit conversions are inserted into the value
flow whereever necessary and possible so that all the needed
types can be computed.

All this is a search that specifies implicit type information,
movement modes, implicit conversions, and sometimes compile
time constants.  The search may have
multiple successful answers.  In this case the result is deemed
ambiguous and a compile error.  The user who provides multiple
definitions for a given function may elect to given them a priority
order, so that lower priority definitions will not be examined by
the search if a higher priority definition can be used, and ambiguity
may be avoided.  Similarly conversions have priorities, with not using
an implied conversion being given higher priority
than using any implicit conversion.

If the search fails to find any answer within a reasonable time, compilation
fails.  If this happens, the coder can explicitly specify as many more types,
variables, and conversions as is necessary to get the search to complete within
a reasonable time.

Typer functions can be implicitly declared using M-language function
declarations, or can be explicitly coded by the user.  The later option
is of course more flexible.

After all types are computed and all overloaded function definitions
have been selected, the function definitions generate function code.
The generated code may just be a fixed block of M-language code,
or it may be generated by calling a generation component function of the
function definition.  Such a generation function is written in the H-language
and returns the code for the function, given the types and movement modes
of its arguments.

The M-language also supports delayed compilation of both code blocks and
functions.   In either case some set of arguments or variables or types
is deemed as the set of parameters to a code generator which determine
the code output by the code generator.  The generated code may be a constant
block of M-Language code that has implied types or can use constant
parameters to optimize the compiled result, or the generated code may be
M-Language code output by a user supplied H-Language code generator function.
Every time the function or block executes the current values of its
parameters are input to the code generator to generate the code to be
executed.  The result is cached according to the
parameter values, so that when the function or block is invoked again,
if the parameters are the same the compiled code may be taken from the
cache and not recompiled.


\bibliographystyle{plain}
\bibliography{layered-introduction}

\printindex

\end{document}

