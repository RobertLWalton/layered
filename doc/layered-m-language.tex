% The Layered Middle (M) Programming Language
%
% File:         layered-m-language.tex
% Author:       Bob Walton (walton@deas.harvard.edu)
% Version:      1a
  
\documentclass[12pt]{article}

\usepackage{makeidx}
\usepackage{pictex}

\makeindex

\setlength{\oddsidemargin}{0in}
\setlength{\evensidemargin}{0in}
\setlength{\textwidth}{6.5in}
\raggedbottom

\setlength{\unitlength}{1in}

\pagestyle{headings}
\setlength{\parindent}{0.0in}
\setlength{\parskip}{1ex}

\setcounter{secnumdepth}{5}
\setcounter{tocdepth}{5}
\newcommand{\subsubsubsection}[1]{\paragraph[#1]{#1.}}
\newcommand{\subsubsubsubsection}[1]{\subparagraph[#1]{#1.}}

% Begin \tableofcontents surgery.

\newcount\AtCatcode
\AtCatcode=\catcode`@
\catcode `@=11	% @ is now a letter

\renewcommand{\contentsname}{}
\renewcommand\l@section{\@dottedtocline{1}{0.1em}{1.4em}}
\renewcommand\l@table{\@dottedtocline{1}{0.1em}{1.4em}}
\renewcommand\tableofcontents{%
    \begin{list}{}%
	     {\setlength{\itemsep}{0in}%
	      \setlength{\topsep}{0in}%
	      \setlength{\parsep}{1ex}%
	      \setlength{\labelwidth}{0in}%
	      \setlength{\baselineskip}{1.5ex}%
	      \setlength{\leftmargin}{1.0in}%
	      \setlength{\rightmargin}{1.0in}}%
    \item\@starttoc{toc}%
    \end{list}}
\renewcommand\listoftables{%
    \begin{list}{}%
	     {\setlength{\itemsep}{0in}%
	      \setlength{\topsep}{0in}%
	      \setlength{\parsep}{1ex}%
	      \setlength{\labelwidth}{0in}%
	      \setlength{\baselineskip}{1.5ex}%
	      \setlength{\leftmargin}{1.0in}%
	      \setlength{\rightmargin}{1.0in}%
	      }%
    \item\@starttoc{lot}%
    \end{list}}

\catcode `@=\AtCatcode	% @ is now restored

% End \tableofcontents surgery.

\newcommand{\CN}[2]%	Change Notice.
    {\hspace*{0in}\marginpar{\sloppy \raggedright \it \footnotesize
     $^{\mbox{#1}}$#2}}
    % Change notice.

\newcommand{\key}[1]{{\bf \em #1}\index{#1}}
\newcommand{\mkey}[2]{{\bf \em #1}\index{#1!#2}}
\newcommand{\skey}[2]{{\bf \em #1#2}\index{#1}}
\newcommand{\ikey}[2]{{\bf \em #1}\index{#2}}
\newcommand{\ttkey}[1]{{\tt \bf #1}\index{#1@{\tt #1}}}
% < and > do not work for \tt \bf, hence:
\newcommand{\ttnbkey}[1]{{\tt #1}\index{#1@{\tt #1}}}
\newcommand{\ttmkey}[2]{{\tt \bf #1}\index{#1@{\tt #1}!#2}}
\newcommand{\ttmnbkey}[2]{{\tt #1}\index{#1@{\tt #1}!#2}}
\newcommand{\ttfkey}[2]{{\tt \bf #1}\index{#1@{\tt #1}!for #2@for {\tt #2}}}
\newcommand{\ttakey}[2]{{\tt \bf #1}\index{#2@{\tt #1}}}
\newcommand{\ttamkey}[3]{{\tt \bf #1}\index{#2@{\tt #1}!#3}}
\newcommand{\ttdkey}[1]{{\tt \bf .#1}\index{#1@{\tt .#1}}}
\newcommand{\ttdmkey}[2]{{\tt \bf .#1}\index{#1@{\tt .#1}!#2}}
\newcommand{\ttindex}[1]{\index{#1@{\tt #1}}}
\newcommand{\ttmindex}[2]{\index{#1@{\tt #1}!#2}}
\newcommand{\ttikey}[2]{{\tt \bf #1}\index{#1@{\tt #1}!#2}}
\newcommand{\emkey}[1]{{\bf \em #1}\index{#1@{\em #1}}}
\newcommand{\emindex}[1]{\index{#1@{\em #1}}}

\newcommand{\secref}[1]{\ref{#1}$^{p\pageref{#1}}$}
\newcommand{\stepref}[1]{\ref{#1}$^{p\pageref{#1}}$}
\newcommand{\appref}[1]{\ref{#1}$^{p\pageref{#1}}$}
\newcommand{\figref}[1]{\ref{#1}$^{p\pageref{#1}}$}
\newcommand{\pagref}[1]{p\pageref{#1}}

\newcommand{\EOL}{\penalty \exhyphenpenalty}

\newcount\TildeCatcode
\TildeCatcode=\catcode`\~
\catcode`~=12
\newcommand{\Tilde}{~}
\catcode`~=\TildeCatcode

\newcount\CircumflexCatcode
\CircumflexCatcode=\catcode`\^
\catcode`^=12
\newcommand{\Circumflex}{^}
\catcode`^=\CircumflexCatcode

\newcount\CurlyBraCatcode
\newcount\CurlyKetCatcode
\newcount\SquareBraCatcode
\newcount\SquareKetCatcode
\CurlyBraCatcode=\catcode`{
\CurlyKetCatcode=\catcode`}
\SquareBraCatcode=\catcode`[
\SquareKetCatcode=\catcode`]

\catcode`{=\SquareBraCatcode
\catcode`}=\SquareKetCatcode
\catcode`[=\CurlyBraCatcode
\catcode`]=\CurlyKetCatcode

\newcommand[\CurlyBra][{]
\newcommand[\CurlyKet][}]

\catcode`{=\CurlyBraCatcode
\catcode`}=\CurlyKetCatcode
\catcode`[=\SquareBraCatcode
\catcode`]=\SquareKetCatcode

\newcommand{\ttbrackets}{%
    \renewcommand{\{}{\CurlyBra}%
    \renewcommand{\}}{\CurlyKet}}

\newsavebox{\TILDEBOX}
\begin{lrbox}{\TILDEBOX}
\verb|~|
\end{lrbox}
\newcommand{\TILDE}{\usebox{\TILDEBOX}}

\newsavebox{\BACKSLASHBOX}
\begin{lrbox}{\BACKSLASHBOX}
\verb|\|
\end{lrbox}
\newcommand{\BACKSLASH}{\usebox{\BACKSLASHBOX}}

\newsavebox{\LEFTBRACKETBOX}
\begin{lrbox}{\LEFTBRACKETBOX}
\verb|{|
\end{lrbox}
\newcommand{\LEFTBRACKET}{\usebox{\LEFTBRACKETBOX}}

\newsavebox{\RIGHTBRACKETBOX}
\begin{lrbox}{\RIGHTBRACKETBOX}
\verb|}|
\end{lrbox}
\newcommand{\RIGHTBRACKET}{\usebox{\RIGHTBRACKETBOX}}

\newsavebox{\UNDERLINEBOX}
\begin{lrbox}{\UNDERLINEBOX}
\verb|_|
\end{lrbox}
\newcommand{\UNDERLINE}{\usebox{\UNDERLINEBOX}}

\newsavebox{\CIRCUMFLEXBOX}
\begin{lrbox}{\CIRCUMFLEXBOX}
\verb|^|
\end{lrbox}
\newcommand{\CIRCUMFLEX}{\usebox{\CIRCUMFLEXBOX}}

\newsavebox{\BARBOX}
\begin{lrbox}{\BARBOX}
\verb/|/
\end{lrbox}
\newcommand{\BAR}{\usebox{\BARBOX}}

\newsavebox{\LESSTHANBOX}
\begin{lrbox}{\LESSTHANBOX}
\verb/</
\end{lrbox}
\newcommand{\LESSTHAN}{\usebox{\LESSTHANBOX}}

\newsavebox{\GREATERTHANBOX}
\begin{lrbox}{\GREATERTHANBOX}
\verb/>/
\end{lrbox}
\newcommand{\GREATERTHAN}{\usebox{\GREATERTHANBOX}}

\newlength{\figurewidth}
\setlength{\figurewidth}{\textwidth}
\addtolength{\figurewidth}{-0.40in}

\newsavebox{\figurebox}

\newenvironment{boxedfigure}[1][!btp]%
	{\begin{figure*}[#1]
	 \begin{lrbox}{\figurebox}
	 \begin{minipage}{\figurewidth}

	 \vspace*{1ex}}%
	{
	 \vspace*{1ex}

	 \end{minipage}
	 \end{lrbox}
	 \begin{center}
	 \fbox{\hspace*{0.1in}\usebox{\figurebox}\hspace*{0.1in}}
	 \end{center}
	 \end{figure*}}

\newenvironment{indpar}[1][0.3in]%
	{\begin{list}{}%
		     {\setlength{\itemsep}{0in}%
		      \setlength{\topsep}{0in}%
		      \setlength{\parsep}{1ex}%
		      \setlength{\labelwidth}{#1}%
		      \setlength{\leftmargin}{#1}%
		      \addtolength{\leftmargin}{\labelsep}}%
	 \item}%
	{\end{list}}

\begin{document}
        
\begin{center}

{\Large
The Layered Middle (M) Programming Language \\[0.5ex]
(Draft 1a)}

\medskip

Robert L. Walton\footnote{This document is dedicated to the memory
of Professor Thomas Cheatham of Harvard University.}

March 16, 2007
 
\end{center}

{\small
\tableofcontents 
}

\newpage

\section{Introduction}

This document describes the Middle Layer Programming Language, or
M-Language.  See the Introduction to the Layered
Programming Languages for an a description of the syntax of
the M-Language and an overview of the related
Lower Layer L-Language and Higher Layer H-Language.


\section{Memory}
\label{MEMORY}

We begin with an overview of M-language memory, and then provide
details in the following subsections.

TBD

\subsection{Blocks}
\label{BLOCKS}

An M-language memory \key{block} is a sequence of bits.
Blocks can be contained in other blocks or can overlap.

A block contains a subblock called the \ikey{base}{of block}
of the block.  For many blocks the base and the block are the same.
For blocks that can grow, the base is the part of the block that
is always there.

When a block is stored in RAM (random access memory),
it has a \ikey{bit address}{of block}.  The block bit address plus
some constant offset is the bit address of the first
bit of the base of the block.

Blocks have the following attributes:

\begin{indpar}\begin{tabular}{p{1.0in}p{4.0in}}
\ttikey{size}{of block}
    & Number of bits in the base of the block.
\end{tabular}\end{indpar}
\begin{indpar}\begin{tabular}{p{1.0in}p{4.0in}}
\ttikey{growth}{of block}
    & \ttikey{up}{block growth} and/or \ttikey{down}{block growth} or neither;
      Both {\tt up} and {\tt down} mean the block may be larger than its base.
      {\tt up} means there may be extra bits after the
      base, and {\tt down} means there may be extra bits before the base.
      If the {\tt growth} of a block is neither {\tt up} nor {\tt down},
      the block is the same as its base.
\end{tabular}\end{indpar}
\begin{indpar}\begin{tabular}{p{1.0in}p{4.0in}}
\ttikey{offset}{of block}
    & Offset in bits of the bit address of the first bit of the block
      base from the bit address of the block, when the block is stored
      in RAM.
\end{tabular}\end{indpar}
\begin{indpar}\begin{tabular}{p{1.0in}p{4.0in}}
\ttikey{alignment}{of block}
    & A strictly positive integer, the exact divisor of the block bit address,
      when the block is stored in RAM.  Usually a power of 2, frequently
      at least 8.
\end{tabular}\end{indpar}
\begin{indpar}\begin{tabular}{p{1.0in}p{4.0in}}
\ttikey{mobility}{of block}
    & \ttikey{fixed}{block mobility},
      \ttikey{movable}{block mobility},
      or \ttikey{free}{block mobility}.
      A {\tt fixed} block is stored in RAM at a fixed address.
      A {\tt movable} block is stored in RAM at an address that
      may be either fixed or may be changed according to the
      rules for managing heavy weight addresses below (\secref{ADDRESSES}).
      A {\tt free} block may be stored in register or RAM
      and moved at any time.
\end{tabular}\end{indpar}
\begin{indpar}\begin{tabular}{p{1.0in}p{4.0in}}
\ttikey{operability}{of block}
    & \ttikey{integer}{block operability},
      \ttikey{float}{block operability},
      \ttikey{address}{block operability}, or
      \ttikey{container}{block operability}.
      The {\tt opera\-bility} of a block determines the kind of
      register the block is stored in, if the block is moved to
      a register.  The block must be {\tt free}.
\end{tabular}\end{indpar}
\begin{indpar}\begin{tabular}{p{1.0in}p{4.0in}}
\ttikey{status}{of block}
    & \ttikey{constant}{block status} or
      \ttikey{variable}{block status}.  A {\tt constant}
      block cannot be change, while a {\tt variable} block can be.
\end{tabular}\end{indpar}

This is just enough information to allow the M-language implementation
to allocate blocks to memory and move blocks in memory.

\subsection{Addresses}
\label{ADDRESSES}

Both blocks and addresses are classified as `light weight'
or `heavy weight'.

A \key{heavy weight block} is a movable block that
has a \key{forwarding address} stored at a fixed location that
always points at the block and that changes when the block moves.
The block is addressed indirectly through its forwarding
address, thus making it possible to move the block at almost
any time (see~\secref{BLOCK-MOVEMENT-PROTOCOL}).

A \key{heavy weight address} has
two parts: a pointer to a forwarding address, and an offset within
the heavy weight block that is pointed at by the forwarding address.

A \key{light weight block} is a free block,
a fixed block allocated to a fixed address,
or a subblock of another block.  It can be a subblock of a
heavy weight block.
A heavy weight address is used to address a light weight subblock
of a heavy weight block.

A \key{light weight address} is just the address of a fixed, non-movable
(light weight) block.  Light weight addresses are integers that have
size and alignment equal to 64, and that address bytes.  They must be
mulitiplied by 8 to give bit addresses.

A \key{pointer} is a list of constant free blocks
from which a block bit address can be
computed by some \key{pointer function}.
The block whose bit address is computed is the
\key{target} of the pointer.  A pointer can be a light weight address,
it can be an index into some data structure in which light weight
addresses are stored, or it can be a list of such indices.
A heavy weight address contains a pointer
to a forwarding address, and this pointer need not itself be an address.

There may be different kinds of pointer, and the pointer
functions used to translate pointers into a block bit addresses will
be different for each kind of pointer.  The only arguments to a pointer
function are the elements of the pointer (viewed as a list),
but the pointer function may reference other
data structures implicitly.

An entire heavy weight block can be
addressed by a pointer pointing at the
forwarding address of the block,
but an addressing indirection is required to address
the block.  Such a pointer, which is just a heavy weight address
without any offset part, is called a \key{middle weight address}.
A heavy weight block that is never addressed by a heavy
weight address, but only by middle weight addresses, is called a
\key{middle weight block}.

A \key{forwarding address} has the same format as a light weight address.
Forwarding addresses can all be elements of a vector or other data
structure, in which case an index can be used in a heavy weight or
middle weight address as a pointer to a forwarding address.

Alternatively forwarding addresses can be allocated as the first thing in
their movable heavyweight block, in which case they must be addressed
and not identified by index.  In this case, the entire heavy weight
block must be allocated as a fixed block, so the location
of the forwarding address will be fixed.  When the heavy weight block
is moved, the forwarding address in its old location is changed to point
to the new location.  At this point the block has two forwarding addresses,
one in its old location, and one at the beginning of its new location,
but both contain the same value.  Standard garbage collection techniques
can be used to remove all references to the old forwarding address
so that the old forwarding address and the old block containing it
can be deallocated.  In greater generality, any forwarding
address may be moved by allocating a duplicate and then eliminating
all references to the original forwarding address.

A heavy weight block can be moved almost anytime by
changing its forwarding address.  The block can be deallocated
almost anytime by moving it to unimplemented virtual memory.

A heavy weight block can be a stack that grows up or down, and
subblocks can be allocated to or deallocated from the growing
boundary of a stack.  Fixed blocks can also be stacks; but they
must have RAM above or below them reserved for allocation
of subblocks.

\subsection{Block Graphs}
\label{BLOCK-GRAPHS}

A \key{block graph} is:

\begin{itemize}

\item A set of blocks.

\item For each of these blocks, a (possibly empty) set of pointers to blocks
      in the graph.  These are the pointers of the graph.
      Each block is the \key{source} of the pointers in its set.
      Each block that is a target of one of these pointers must be stored
      in RAM, but blocks that are not targets can be free.

\item A particular subset of the blocks of the graph whose elements
      are called the \ikey{roots}{root!of block graph} of the
      block graph.

\end{itemize}

A \key{object} is a subgraph of a block graph that consists of
a single root block and all blocks reachable from this root
by following graph pointers whose targets are not root blocks.
Thus there is only one root block in a object, which is called
the \ikey{root}{of object} of the object.

A object points at another object if some block of the first
object sources a graph pointer whose target is the root of
the second object.  A set of objects is \ikey{closed}{set of objects}
if no object in the set points at a object outside the set.

Block graph blocks that are not stored in RAM cannot have bit
addresses or be targets of pointers.  Roots of objects can be such blocks.
All non-root blocks must be have bit addresses and be stored in RAM,
as they are all pointer targets.

An object whose root is stored in RAM is called a \key{RAM object}.
The \mkey{bit address}{of RAM object}
of a RAM object is the bit address of its root.

A block graph is really a way of viewing the contents of memory.  The same
memory contents can be viewed in many ways using different block graphs:
in general there is no one unique block graph in a computational process.
Blocks of memory may also be outside any block graph of the process.

A block graph is said to be \key{non-overlapping} if no two blocks
in the graph overlap and no two objects in the graph overlap.
The latter statement is equivalent to saying that
no non-root block in the graph is reachable from
two different roots by following pointers whose targets are not roots.

A \key{constituent} of a block is a subblock.  The larger
block is the \ikey{container}{of constituent} of the constitutent.
Note that elsewhere we define a \key{component} of a block to
be a value that can be computed from the block, which is a much
more general concept.

An object is said to be a \mkey{constant}{object}\label{CONSTANT-OBJECT}
if no part of it changes
except for bits in its graph pointer constituents, and if its graph does
not change.

A block graph is said to be \key{exportable} if it is non-overlapping,
if the pointers sourced at each graph block are non-overlapping constituents
of the block, if all its objects are RAM objects (all roots have bit addresses),
and if a closed set of objects can be copied to another
memory, or to a disjoint part of the memory holding the graph,
by copying the blocks of the
set and then changing only the bits of the graph pointer constituents
of these copies so the graph structure of the set of copies is the same
as the graph structure of the copied closed set of objects.

Intuitively a copy of a closed set of objects of an
exportable block graph should have the same information content as the
copied original.  But this cannot be rigorously defined for the general
case.

An \key{exportable object} is an object in an exportable block graph.

It is possible to copy a non-closed set of objects of an exportable
block graph using stubs.  A \key{stub} for an object
is a copy of some of the information in the object, with enough information
being present to identify the object.  When copying a non-closed set of
objects, stubs can be created for any non-copied objects pointed at by
copied objects.  If stubs already exist for these objects, they can be
located and used instead of creating new stubs.  If an object is copied
that already has a stub copy, that stub can be replaced by the copied object.
Note that in order to be able to replace a stub by an object, the root
of the object must be heavy weight or middle weight (\secref{ADDRESSES}),
or the stub must be as big as the root of the object.

\subsection{Values and Types}
\label{VALUES}

A \key{value} is a sequence of objects.  The last object is
the \key{value proper}.
The initial subsequence of all objects except the last
is the \key{type} of the value, and is itself a value.

The block graph of each object in a value can be \underline{different}.
The objects in a value need not be RAM objects.

The type of a value descibes the format of the value proper, and is used
to select function code when the value is given as an argument
to a function.  Some example types are 
`32-bit signed integer' and `64-bit floating point number'.
The type must specify the attributes of the root of the value proper; e.g.,
the size of the root.  The type must also specify the block graph
to which the value proper belongs.

Since a type is a value, its last object is a value proper for the type,
and the sequence of preceding objects is the type of the type.
As a special case, the empty sequence is taken as equal to
a particular value that is a type that specifies that the
associated value proper is a type and is constant.

A type defines a view of a value proper.  A value proper can
have several different views, i.e., can be the value proper
of several different values with different types.

Values are partially ordered according
how much information they contain.  This is called the
\key{information ordering}.  Functions that input and output
values may be \key{information monotonic}, meaning that
if they are given inputs with more information they will
produce outputs with more information.  These concepts are
very important for values that are types.

Values can exist at both compile-time and at run-time.
There is an operation called \key{linking} that copies
a value from compile time to run time and must be information
monotonic.

A value whose value proper is unknown is represented for practical
purposes by its type.  Such values are common at compile-time.
A value whose value proper is completely known at compile-time
is called a \key{compile-time constant}.

At run-time a type is required to consist of exportable objects
that can only change in information increasing ways.
They may, however, contain pointers to variable objects.

Types are computed at compile time with
the help of type inference functions that are associated with run-time
functions whose calls are being compiled.  Here operators are considered
to be names of such functions.
The \skey{type inference function}s take a set of values as input and produce
the same set as output, thus replacing their inputs by their outputs.
Type inference functions are required to be information monotonic.

The \key{type inference algorithm} takes as input
a set of compile-time values (typically with unknown value propers)
and a set of calls to type inference functions, each with a subset
of the values as input and output.  The algorithm executes the calls
repeatedly until all calls make no changes.  The resulting values
are called a \ikey{stable point}{of type inference}.  It can be shown
that this is uniquely determined by the original inputs and type
inference functions, because the latter are monotonic.
It is a compile-time error if this algorithm
takes to long to run, as determined in a machine independent manner.
(It is possible of the algorithm to run indefinitely only if values
can be indefinitely large).

In actual practice there are several possible sets of type inference
functions, due to overloading, and the compiler must select which
one to use.  Each type inference function may decide that its inputs
are not consistent for its purposes, in which case the set of inference
functions being tested cannot be used.  By this means a single set
of inference functions, corresponding to a single mapping of function
names to overloaded actual functions, is selected.

Normally at run-time a value is a sequence of just two objects,
the first being a constant type, and the second the value proper,
which may or may not be constant, as is indicated by the type.
Normally during compile-time a value is a sequence of three
objects, the first begin a constant type that indicates that the
second is a type that is not constant and can be changed, and the
third block being the value proper, which is often missing.
The compile-time value is translated to the run-time value by
discarding its first object and making its second into a constant
object.

TBD

However, there is no prohibition
against values having types that are not constant at run time, though
the available operations on such values are few and slow.  Such
values are well suited to compute inputs to delayed compilation
(see~\secref{EXPRESSIONS}).

\subsection{TBD}
\label{TBD}

A component of a value V is a value C that can be computed from V
by a designated `read function'.  The read function may take
agruments in addition to V, and these are called indices.
The components of a variable are just a components of the variable's value.
A component of a variable can itself be a variable if there is a
designated `write function' that changes the value returned for the
component.  Read and write functions for a component must obey
some rules: for example, two successive reads must return the same
value, and a read immediately following a write must return the value
written, when the value and index arguments given to successive functions
calls are the same.  In addition to read and write functions a component
may have update, accumulate, and test functions.

The different components of a variable obey different protocols.
The {\tt rw} protocol says that a component is readable and writable
at any time.  {\tt w/r} says that all writes occur before all reads.
{\tt w/a/r} says that writes occur first, then accumulates, then reads,
where accumulates are operations that may be exchanged, such as incrementing
the variable component by various amounts.

Types and type inference are not described in detail until
\secref{EXPRESSIONS}.
Components and protocols are not describe in detail until
\secref{PROTOCOLS}.  The other concepts just introduced are
defined in detail in the following subsections.

\section{Expressions}
\label{EXPRESSIONS}

The task of an expression is to compute values from other values.
This is straightforward except for issues in parsing, overloading,
implicit variable creation, implicit typing, implicit conversion,
and protection.

Parsing involves inserting implied parentheses and then rewriting
expressions, particularly those involving operators.  Overloading
involves picking which function to call given many functions with the
same name.  Implicit variable creation involves creating variables to
hold values output by one function and input by another.\footnote{
What we call `implicit variables' are usually called `temporary variables'.
We make the implicit variable creation process explicit for two reasons:
first, we have a more complex situation in which one function may
output many results, as is explained a bit further
in the text, and second, M-language debuggers have a mode of
operation in which instead of displaying location within the program
code, which can be a problematic concept given all the rewriting that
is done, they display the status of ordinary and implicit variables, making
it important to explain implicit variables carefully to novice programmers.}
Implicit typing involves assigning types to variables, such as implicit
variables, that were given no type or only a partial type in the code.
Implicit conversion involves inserting conversion operators to change the
type of a value.
Protection involves ensuring that code in a restricted protection domain
cannot be called or referenced from another protection domain without
explicit permission.

Parsing begins by identifying operators in an expression.  By using
a precedence level assigned to each operator/fixity pair (operators
can have three fixities: prefix, infix, or postfix), implied parentheses
are inserted.  The M-language differs from other computer languages
in not using associativity rules in doing this, so that, for
instance, `\verb|(0 < x <= y)|' has \underline{no} implied parentheses
inserted, and can be rewritten later to be the equivalent of
`\verb|(0 < x AND x <= y)|'.  After inserting implied parentheses,
expressions are rewritten using macros.  In the example just given,
`\verb|<|' and `\verb|<=|' have the same precedence level and invoke
the same rewrite macro.

Program code begins as text, a sequence of characters.  It is first
scanned to become a sequence of lexemes (words and numbers and
punctuation marks), and then parsed.  The lexemes are tagged with
their location within the text, and with the subexpression they
belong to after insertion of implied parentheses.  These tags
are carried through the rewriting of expressions, and may be copied
to new lexemes introduced during rewriting, so that eventually
a debugger can associate operations executed when the program runs with
particular lexemes and subexpressions in the program code text.

Parsing and rewriting use data constructs and operations provided
by the H-Language, the higher level layered language.  This language
provides, for example, for strings used to encode lexemes and
lists used to encode expressions.  Macros are user written H-language code
executed by the M-language compiler.

Overloading, implicit variable creation, implicit typing, and implicit
conversion are all
done together.  Each function definition has a component function called
the typer which takes as input a partial description of the types
and movement modes of the arguments of a call to the function
and any available values of compile time constant arguments
input to the function,
and produces as output possible completions of this information which
would allow the function definition to be used.  The movement modes
of an argument are `in' for an input argument, `out' for an output
argument, and `inout' for an argument that is both input and output.
The result of a function is treated here as an argument.
The typer succeeds or fails when it is given enough information to
ensure that the function definition can or cannot be used, and if it
is not given enough information, the typer proposes additions to the
information it is given, if it can.  Note that if all inputs to a function
are compile time constants, it may happen that the typer can produce
compile time constants for all outputs, and thus eliminate the need to execute
the function at run time.

Specification of argument movement modes determines what implicit
variables are needed.  Specification of argument types determines
implicit types of both implicit variables and of explicit variables
that are declared without types (or with incomplete types).

In addition to typers for explicitly named functions,
typers for implicitly named conversion functions are executed.
Conversion functions convert values of one type to values of a
different type: e.g., convert 32-bit integers to 64-bit floating
point numbers.  Implicit conversions are inserted into the value
flow wherever necessary and possible so that all the needed
types can be computed.

All this is a search that specifies implicit type information,
movement modes, implicit conversions, and compile time constants.
This search may have
multiple successful answers.  If it does the result is deemed
ambiguous and a compile error.  The user who provides multiple
definitions for a given function may elect to given them a priority
order, so that lower priority definitions will not be examined by
the search if a higher priority definition can be used, and ambiguity
may be avoided.  Similarly implied conversions have priorities, with not using
an implied conversion being given higher priority
than using any implicit conversion.

If the search fails to find any answer within a reasonable time, compilation
fails.  If this happens, the coder can explicitly specify as many more types,
variables, and conversions as is necessary to get the search to complete within
a reasonable time.  Here time is measured by counting certain operations
performed by the search, so the measure of reasonable time will be hardware
independent.

Typer functions can be implicitly declared using M-language function
declarations, or can be explicitly coded by the user.  The later option
is of course more flexible.

After all types are computed and all overloaded function definitions
have been selected, the function definitions generate function code.
The generated code may just be a fixed block of M-language code,
or it may be generated by calling a generation component function of the
function definition.  Such a generation function is written in the H-language
and returns the code for the function, given the types and movement modes
of its arguments, and any compile time constant argument values.

The M-language also supports delayed compilation of both code blocks and
functions.   In either case some set of arguments or variables or types
is specified to be the set of parameters input to a code generator
that generates the block or function code.
The generated code may be a constant
block of M-Language code that has implied types or uses parameters as
compile time constants, or the generated code may be
M-Language code output by a user supplied H-Language code generator function.
Every time the function or block executes the current values of its
parameters may be input to the code generator to generate the code to be
executed.  The result is cached according to the
parameter values, so that when the function or block is invoked again,
if the parameters are the same the compiled code may be taken from the
cache and not recompiled.

\section{Protocols}
\label{PROTOCOLS}

\subsection{Block Movement Protocol}
\label{BLOCK-MOVEMENT-PROTOCOL}


\bibliographystyle{plain}
\bibliography{layered-m-language}

\printindex

\end{document}

