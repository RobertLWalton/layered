% The Layered Middle (M) Programming Language
%
% File:         layered-m-language.tex
% Author:       Bob Walton (walton@deas.harvard.edu)
% Version:      1a
  
\documentclass[12pt]{article}

\usepackage{makeidx}
\usepackage{pictex}

\makeindex

\setlength{\oddsidemargin}{0in}
\setlength{\evensidemargin}{0in}
\setlength{\textwidth}{6.5in}
\raggedbottom

\setlength{\unitlength}{1in}

\pagestyle{headings}
\setlength{\parindent}{0.0in}
\setlength{\parskip}{1ex}

\setcounter{secnumdepth}{5}
\setcounter{tocdepth}{5}
\newcommand{\subsubsubsection}[1]{\paragraph[#1]{#1.}}
\newcommand{\subsubsubsubsection}[1]{\subparagraph[#1]{#1.}}

% Begin \tableofcontents surgery.

\newcount\AtCatcode
\AtCatcode=\catcode`@
\catcode `@=11	% @ is now a letter

\renewcommand{\contentsname}{}
\renewcommand\l@section{\@dottedtocline{1}{0.1em}{1.4em}}
\renewcommand\l@table{\@dottedtocline{1}{0.1em}{1.4em}}
\renewcommand\tableofcontents{%
    \begin{list}{}%
	     {\setlength{\itemsep}{0in}%
	      \setlength{\topsep}{0in}%
	      \setlength{\parsep}{1ex}%
	      \setlength{\labelwidth}{0in}%
	      \setlength{\baselineskip}{1.5ex}%
	      \setlength{\leftmargin}{1.0in}%
	      \setlength{\rightmargin}{1.0in}}%
    \item\@starttoc{toc}%
    \end{list}}
\renewcommand\listoftables{%
    \begin{list}{}%
	     {\setlength{\itemsep}{0in}%
	      \setlength{\topsep}{0in}%
	      \setlength{\parsep}{1ex}%
	      \setlength{\labelwidth}{0in}%
	      \setlength{\baselineskip}{1.5ex}%
	      \setlength{\leftmargin}{1.0in}%
	      \setlength{\rightmargin}{1.0in}%
	      }%
    \item\@starttoc{lot}%
    \end{list}}

\catcode `@=\AtCatcode	% @ is now restored

% End \tableofcontents surgery.

\newcommand{\CN}[2]%	Change Notice.
    {\hspace*{0in}\marginpar{\sloppy \raggedright \it \footnotesize
     $^{\mbox{#1}}$#2}}
    % Change notice.

\newcommand{\key}[1]{{\bf \em #1}\index{#1}}
\newcommand{\mkey}[2]{{\bf \em #1}\index{#1!#2}}
\newcommand{\skey}[2]{{\bf \em #1#2}\index{#1}}
\newcommand{\ikey}[2]{{\bf \em #1}\index{#2}}
\newcommand{\ttkey}[1]{{\tt \bf #1}\index{#1@{\tt #1}}}
% < and > do not work for \tt \bf, hence:
\newcommand{\ttnbkey}[1]{{\tt #1}\index{#1@{\tt #1}}}
\newcommand{\ttmkey}[2]{{\tt \bf #1}\index{#1@{\tt #1}!#2}}
\newcommand{\ttmnbkey}[2]{{\tt #1}\index{#1@{\tt #1}!#2}}
\newcommand{\ttfkey}[2]{{\tt \bf #1}\index{#1@{\tt #1}!for #2@for {\tt #2}}}
\newcommand{\ttakey}[2]{{\tt \bf #1}\index{#2@{\tt #1}}}
\newcommand{\ttamkey}[3]{{\tt \bf #1}\index{#2@{\tt #1}!#3}}
\newcommand{\ttdkey}[1]{{\tt \bf .#1}\index{#1@{\tt .#1}}}
\newcommand{\ttdmkey}[2]{{\tt \bf .#1}\index{#1@{\tt .#1}!#2}}
\newcommand{\ttindex}[1]{\index{#1@{\tt #1}}}
\newcommand{\ttmindex}[2]{\index{#1@{\tt #1}!#2}}
\newcommand{\emkey}[1]{{\bf \em #1}\index{#1@{\em #1}}}
\newcommand{\emindex}[1]{\index{#1@{\em #1}}}

\newcommand{\secref}[1]{\ref{#1}$^{p\pageref{#1}}$}
\newcommand{\stepref}[1]{\ref{#1}$^{p\pageref{#1}}$}
\newcommand{\appref}[1]{\ref{#1}$^{p\pageref{#1}}$}
\newcommand{\figref}[1]{\ref{#1}$^{p\pageref{#1}}$}
\newcommand{\pagref}[1]{p\pageref{#1}}

\newcommand{\EOL}{\penalty \exhyphenpenalty}

\newcount\TildeCatcode
\TildeCatcode=\catcode`\~
\catcode`~=12
\newcommand{\Tilde}{~}
\catcode`~=\TildeCatcode

\newcount\CircumflexCatcode
\CircumflexCatcode=\catcode`\^
\catcode`^=12
\newcommand{\Circumflex}{^}
\catcode`^=\CircumflexCatcode

\newcount\CurlyBraCatcode
\newcount\CurlyKetCatcode
\newcount\SquareBraCatcode
\newcount\SquareKetCatcode
\CurlyBraCatcode=\catcode`{
\CurlyKetCatcode=\catcode`}
\SquareBraCatcode=\catcode`[
\SquareKetCatcode=\catcode`]

\catcode`{=\SquareBraCatcode
\catcode`}=\SquareKetCatcode
\catcode`[=\CurlyBraCatcode
\catcode`]=\CurlyKetCatcode

\newcommand[\CurlyBra][{]
\newcommand[\CurlyKet][}]

\catcode`{=\CurlyBraCatcode
\catcode`}=\CurlyKetCatcode
\catcode`[=\SquareBraCatcode
\catcode`]=\SquareKetCatcode

\newcommand{\ttbrackets}{%
    \renewcommand{\{}{\CurlyBra}%
    \renewcommand{\}}{\CurlyKet}}

\newsavebox{\TILDEBOX}
\begin{lrbox}{\TILDEBOX}
\verb|~|
\end{lrbox}
\newcommand{\TILDE}{\usebox{\TILDEBOX}}

\newsavebox{\BACKSLASHBOX}
\begin{lrbox}{\BACKSLASHBOX}
\verb|\|
\end{lrbox}
\newcommand{\BACKSLASH}{\usebox{\BACKSLASHBOX}}

\newsavebox{\LEFTBRACKETBOX}
\begin{lrbox}{\LEFTBRACKETBOX}
\verb|{|
\end{lrbox}
\newcommand{\LEFTBRACKET}{\usebox{\LEFTBRACKETBOX}}

\newsavebox{\RIGHTBRACKETBOX}
\begin{lrbox}{\RIGHTBRACKETBOX}
\verb|}|
\end{lrbox}
\newcommand{\RIGHTBRACKET}{\usebox{\RIGHTBRACKETBOX}}

\newsavebox{\UNDERLINEBOX}
\begin{lrbox}{\UNDERLINEBOX}
\verb|_|
\end{lrbox}
\newcommand{\UNDERLINE}{\usebox{\UNDERLINEBOX}}

\newsavebox{\CIRCUMFLEXBOX}
\begin{lrbox}{\CIRCUMFLEXBOX}
\verb|^|
\end{lrbox}
\newcommand{\CIRCUMFLEX}{\usebox{\CIRCUMFLEXBOX}}

\newsavebox{\BARBOX}
\begin{lrbox}{\BARBOX}
\verb/|/
\end{lrbox}
\newcommand{\BAR}{\usebox{\BARBOX}}

\newsavebox{\LESSTHANBOX}
\begin{lrbox}{\LESSTHANBOX}
\verb/</
\end{lrbox}
\newcommand{\LESSTHAN}{\usebox{\LESSTHANBOX}}

\newsavebox{\GREATERTHANBOX}
\begin{lrbox}{\GREATERTHANBOX}
\verb/>/
\end{lrbox}
\newcommand{\GREATERTHAN}{\usebox{\GREATERTHANBOX}}

\newlength{\figurewidth}
\setlength{\figurewidth}{\textwidth}
\addtolength{\figurewidth}{-0.40in}

\newsavebox{\figurebox}

\newenvironment{boxedfigure}[1][!btp]%
	{\begin{figure*}[#1]
	 \begin{lrbox}{\figurebox}
	 \begin{minipage}{\figurewidth}

	 \vspace*{1ex}}%
	{
	 \vspace*{1ex}

	 \end{minipage}
	 \end{lrbox}
	 \begin{center}
	 \fbox{\hspace*{0.1in}\usebox{\figurebox}\hspace*{0.1in}}
	 \end{center}
	 \end{figure*}}

\newenvironment{indpar}[1][0.3in]%
	{\begin{list}{}%
		     {\setlength{\itemsep}{0in}%
		      \setlength{\topsep}{0in}%
		      \setlength{\parsep}{1ex}%
		      \setlength{\labelwidth}{#1}%
		      \setlength{\leftmargin}{#1}%
		      \addtolength{\leftmargin}{\labelsep}}%
	 \item}%
	{\end{list}}

\begin{document}
        
\begin{center}

{\Large
The Layered Middle (M) Programming Language \\[0.5ex]
(Draft 1a)}

\medskip

Robert L. Walton\footnote{This document is dedicated to the memory
of Professor Thomas Cheatham of Harvard University.}

November 24, 2006
 
\end{center}

{\small
\tableofcontents 
}

\newpage

\section{Introduction}

This describes the Middle Layer Programming Language, or
M-Language.  See the Introduction to the Layered
Programming Languages for an overview of the related
Lower Layer L and Higher Layer H languages.


\section{Memory}
\label{MEMORY}

We begin with an overview of M-language memory, and then provide
details in the following sections.

M-language memory consists of blocks which are sequences of bits.
Blocks can be contained in other blocks or can overlap.  Blocks
have the following attributes:

\begin{center}
\begin{tabular}{ll}
{\tt size}		& number of bits \\
{\tt growth}		& {\tt up}, {\tt down}, {\tt none} \\
{\tt alignment}		& exact divisor of bit address \\
{\tt offset}		& offset in bits of start of block from
			  bit address of block \\
{\tt mobility}		& {\tt free}, {\tt fixed}, {\tt movable} \\
{\tt operability}	& {\tt integer}, {\tt float}, {\tt address},
			  {\tt container} \\
{\tt status}		& {\tt constant}, {\tt variable} \\
\end{tabular}
\end{center}

This is just enough information to allow the M-language implementation
to allocate blocks to memory and move blocks in memory.  A {\tt growth}
of `{\tt up}' means the {\tt size} is the minimum size and the actual size
is larger with the block extending above its address.  Similarly for
`{\tt down}' with the block extending below its address.  The {\tt alignment}
and {\tt offset} only affects blocks when they are stored in memory.
{\tt Free} blocks have no fixed address and can be moved around in
memory; {\tt fixed} blocks have a fixed unchanging address; and
{\tt movable} blocks have an address that can be changed in a restricted
way (see below).  {\tt Operability} determines the kind of register
a block is copied into, when a block is placed in a register.  {\tt Status}
allows a block to be marked as {\tt constant} permitting its to be moved
around with greater freedom.

Both blocks and addresses are classified as `light weight'
or `heavy weight'.  A heavy weight movable block has a reference
pointer stored at a fixed location that points at the block and
that changes when the block moves.  A heavy weight address has
two parts: the address of a reference pointer, and an offset within
the heavy weight block that is pointed at by the reference pointer.
A light weight address is just the address of a fixed, non-movable
block.

A lightweight block is fixed block allocated to a fixed address,
or is a subblock of another block.  It can be a subblock of a
heavy weight block.
A heavy weight address is used to address a light weight subblock
of a heavy weight block.

The entire heavy weight block can be
addressed by a light weight pointer pointing at the reference pointer
of the block, but an addresssing indirection is required to address
the block, and such a pointer is called a `middle weight'
pointer.  A heavy weight block that is never addressed by a heavy
weight pointer, but only by middle weight pointers, is called a
`middle weight block'.

Reference pointers can also be held in a vector so that
sometimes a vector index can be used in place of an address of 
a reference pointer.  Sometimes reference pointers are allocated
as the first thing in their movable heavyweight block, in which
case they must be addressed and not identified by index.\footnote{
This means the entire heavy weight block must be allocated as a
subblock of a fixed block, so the reference pointer will be fixed.
However, reference pointers can be garbage collected by keeping
two copies, the old and the new, with identical contents, and
then replacing all addresses or indices of the old pointer by
the address or index of the new pointer.}

A heavy weight movable block can be moved almost anytime by
changing its reference pointer.  The block can be deallocated
almost anytime by moving it to unimplemented virtual memory.

A heavy weight movable block can be a stack that grows up or down, and
subblocks can be allocated to or deallocated from the growing
boundary of a stack.  Other blocks can also be stacks; they
must have memory above or below them reserved for allocation
of subblocks.

A value is a sequence of constant free blocks.  The initial subsequence
of all blocks up until the last is called the `type' of the value,
and the blocks in this type are called `descriptors'.  The first descriptor
contains an encoding that specifies
the `major type' of the value.  Some major types are
32-bit signed integers, 64-bit floating point numbers, structure with
a given list of components, etc.  If a second descriptor exists it
generally encodes subtype information, such as bounds for an integer
or subscript bounds for an array.

The value that is an empty sequence is a special case that is equivalent
to a major type descriptor that specifies that a value is a major type
descriptor.

A variable is a place that holds a value that may be changed.
However, when changing the value, only its last block may be changed:
the type may not be.  A variable that has an address in memory will
be either a movable or fixed block.  A variable that has no particular
location in memory is free.

At compile time the compiler must know a non-empty initial segment
of the type of each value.  From this the block attributes of the
missing part of the value must be computable.  For example, the
compiler may know the first block of a value, the one that holds
the major type, and this might specify that the value is a 32-bit integer.

A component of a value V is a value C that can be computed from V
by a function that takes only V as its single argument.  The components
of a variable is just a components of the variable's value.
Components of variables can themselves be variable if there is a
function that changes the value that will be returned for the component.

The different components of a variable obey different protocols.
The {\tt rw} protocol says that a component is readable and writable
at any time.  {\tt w/r} says that all writes occur before all reads.
{\tt w/a/r} says that writes occur first, then accumulates, then reads,
where accumulates are operations that may be exchanged, such as incrementing
the variable component by various amounts.

Components and protocols are not describe in detail until
\secref{EXPRESSIONS}.  The other concepts just introduced are
defined in detail in the following sections.

\section{Expressions}
\label{EXPRESSIONS}




\bibliographystyle{plain}
\bibliography{layered-introduction}

\printindex

\end{document}

